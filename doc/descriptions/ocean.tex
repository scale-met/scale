%\section{Land Physics}
{\bf \Large 
\begin{tabular}{ccc}
\hline
  Corresponding author & : & Tsuyoshi Yamaura\\
\hline
\end{tabular}
}


\subsection{Ocean physics: slab model}

The ocean slab model estimates sea temperature tendencies using a single-layered model.
The governing equations of the internal energy $E$ (J/m$^2$) and mass of ocean $M$ (kg/m$^2$) are
\begin{align}
  \frac{\partial E}{\partial t} &= G + e_{prec} - e_{evap} + Q_{ext}, \label{eq:Ocean-Tdt} \\
  \frac{\partial M}{\partial t} &= F_{prec} - F_{evap},
\end{align}
where
$G$ is the downward surface heat flux (J/m$^2$/s);
$e_{prec}$ and $e_{evap}$ are the downward surface internal energy flux (J/m$^2$/s) of the precipitation and evaporation, respectively;
$Q_{ext}$ is external heat source (J/m$^2$/s);
and $F_{prec}$ and $F_{evap}$ are the surface mass flux (kg/m$^2$/s) of the precipitation and evaporation, respectively.

The internal energy E (J/m$^2$) is
\begin{align}
 E = c_l M T,
\end{align}
where $M$ and $T$ are total mass of water (kg/m$^2$) and temperature, and
$c_l$ is the specific heat capacity of water (J/K/kg),


The surface preciptaion flux is
\begin{align}
 F_{prec} &= F_{rain} + F_{snow},
\end{align}
where $F_{rain}$ and $F_{snow}$ are the surface flux of rain and snow, respectively.
The internal energy fluxes are
\begin{align}
 e_{prec} &= c_lT_{rain}F_{rain} + ( c_iT_{snow} - L_f ) F_{snow}, \\
 e_{evap} &= c_lT_{evap}F_{evap},
\end{align}
where $c_i$ is the specific heat capacity of ice (J/K/kg);
$T_{rain}, T_{snow}$ and $T_{evap}$ are the temperature of rain, snow, and evaporated water, respectively;
Note that the fluxes of the rain and snow are positive for the downward direction and that of the evapolation is positive for the upward.
These ground surface fluxes are calculated in the surface scheme.


In the calculation of change of temperature, the mass change is taken into the account.
However, the mass change is ignored after the calculation and then $M = \rho_w D$,
where
$\rho_w$ is the water density (kg/m$^3$),
and $D$ is the water depth of the slab model.

Eq. (\ref{eq:Ocean-Tdt}) is discretized as follows:
\begin{align}
  T^{n+1}
 &= \frac{ C_w T^n + \Delta t ( G + e_{prec} - e_{evap} + Q_{ext} ) }{ C_w + \Delta t c_l ( F_{prec} - F_{evap} ) }, \nonumber\\
 &= T^n + \Delta t \frac{ G + e_{prec} - e_{evap} + Q_{ext} - c_l ( F_{prec} - F_{evap} ) T^n }{ C_w + \Delta t c_l ( F_{prec} - F_{evap} ) },
\end{align}
where
$C_w$ is the heat capacity of the slab layer (J/K/m$^2$) and $C_w = \rho_w c_l D$.

Note that the internal energy is not conserved, since the mass change is ignored.


\subsection{Sea ice}
\subsubsection{Governing equation}
The equations of budget of the mass (kg/m$^2$) and internal energy (J/m$^2$) in the ocean and sea ice are
\begin{align}
 \frac{\partial M_i}{\partial t} &= f_i ( F_{prec}- F_{subl} ) - m_{mlt} + m_{frz}, \\
 \frac{\partial E_i}{\partial t} &= f_i ( G_i - G_{oi} + e_{prec} - e_{subl} ) - c_lT_0m_{mlt} + ( c_iT_0 - L_f ) m_{frz}, \\
 \frac{\partial M}{\partial t} &= (1-f_i) ( F_{prec} - F_{evap} ) + m_{mlt} - m_{frz}, \\
 \frac{\partial E}{\partial t} &= (1-f_i) ( G_o + e_{prec} - e_{evap} ) + f_i G_{oi} + c_lT_0m_{mlt} - ( c_iT_0 - L_f )m_{frz},
\end{align}
where
$f_i$ is the fraction of the sea ice;
$m_{mlt}$ and $m_{frz}$ are the mass change (kg/m$^2$/s) by melting of ice and freezing of sea water;
and $G_i, G_o$ and $G_{oi}$ are the heat flux (J/m$^2$/s) at the ice-atmosphere, ocean-atmosphere, and ice-ocean surfaces, respectively;
$F_{subl}$ is the upward mass flux due to the sublimation of ice (kg/m$^2$/s);
and $e_{subl}$ is the upward internal energy flux at the surface (J/m$^2$/s) of the sublimated ice as
\begin{align}
 e_{subl} &= (c_iT_{subl}-L_f)F_{subl}.
\end{align}

As noted in the previous section, the mass change in the sea water is ignored after the calculation of change in the ocean temperature.

The $G_{oi}$ is estimated by the diffusion equation
\begin{align}
 G_{oi} &= \nu_i \frac{T_i - T}{D_i/2}, \\
 D_i &= \frac{M_i}{\rho_i f_i},
\end{align}
where $T_i$ is the temperature of the sea ice;
$\nu_i$ and $D_i$ are the thermal conductivity of ice (J/K/m$^3$/s) and depth of the sea ice (m).

The fraction is estimated as
\begin{align}
 f_i &= \sqrt{ \frac{M_i}{M_c} },
\end{align}
where $M_c$ is the critical ice mass (kg/m$^2$).

The amount of the melting during a time step is estimated to  satisfy the conservation of mass and internal energy of the sea ice as
\begin{align}
 M_{mlt} &= \int_t^{t+\Delta t} m_{mlt} dt \nonumber\\
 &= \min\left\{ \max\left\{ \frac{ c_i(T_i-T_0) }{ (c_w-c_i) T_0 + L_f } M_i, 0\right\}, M_i\right\}.
\end{align}

The amount of the freezing is estimated to satisfy the conservation of mass and internal energy of the ocean as
\begin{align}
 M_{frz} &= \int_t^{t+\Delta t} m_{frz} dt \nonumber\\
 &= \min\left\{ \max\left\{ \frac{ c_l \rho_w D ( T_0 - T ) }{ (c_l - c_i)T_0 + L_f }, 0\right\}, \rho_w D\right\},
\end{align}
where $T_0$ is the freezing tempearture.


\subsubsection{Time integration}
The governing equation is solved by a spliting method.
In the first step, the mass and internal energy budgets of the ice without the phase change is solved.
In the second step, the melting of sea ice is estimated by the mass and internal energy conservation.
In the next step, then the temperature change of ocean is calculated in the ocean scheme.
In the last step, the freezing ocean water is estimated and the mass and temperature of ice and ocean is updated.

The followings are the summary of the sequence of calculation.
Here, the superscript ``$n$'' indicates the quantites at the time step $n$, and ``$n_1$'', ``$n_2$'', ``$n_3$'', and ``$n+1$'' are those after calculation of the first, seccond, third and the last step, respectively.


\begin{description}

\item[First step]
\begin{align}
 \Delta M_i^{n_1} &= \Delta t f_i^n ( F_{prec} - F_{subl} ), \\
 \Delta E_i^{n_1} &= \Delta t f_i ( G_i - G_{oi} + e_{prec} - e_{subl} ), \\
 M_i^{n_1} &= M_i^n + \Delta M_i^{n_1}, \\
 T_i^{n_1} &= T_i^n + \frac{ \Delta E_i^{n_1} - ( c_iT_i^n - L_f ) \Delta M_i^{n_1} }{c_i M_i^{n_1}}.
\end{align}

\item[Second step]
\begin{align}
 M_{mlt} &= \min\left\{ \max\left\{ \frac{ c_i(T_i^{n_1}-T_0)M_i^{n_1} }{ (c_w-c_i) T_0 + L_f }, 0\right\}, M_i^{n_1}\right\}, \\
 M_i^{n_2} &= M_i^{n_1} - M_{mlt}, \\
 T_i^{n_2} &= T_i^{n_1} + \frac{ - c_l T_0  + ( c_iT_i^{n_1} - L_f )}{c_i M_i^{n_2}}M_{mlt}.
\end{align}

\item[Third step] (Ocean model)
\begin{align}
 \Delta E^{n_3} &= \Delta t \{ (1-f_i)(G_o + e_{prec} - e_{evap}) + f_i G_{oi} \} + c_l T_0 M_{mlt}, \\
 \Delta M^{n_3} &= \Delta t (1-f_i)(F_{prec}-F_{evap}) + M_{mlt}, \\
 T^{n_3} &= T^n + \frac{ \Delta E^{n_3} - c_l T^n \Delta M^{n_3} }{c_l \{\rho_w D + \Delta M^{n_3} \}}.
\end{align}

\item[Forth step]
\begin{align}
 M_{frz} &= \min\left\{ \max\left\{ \frac{ c_l \rho_w D ( T_0 - T^{n_3} ) }{ (c_l - c_i)T_0 + L_f }, 0\right\}, \rho_w D\right\}, \\
 M_i^{n+1} &= M_i^{n_2} + M_{frz}, \\
 T_i^{n+1} &= T_i^{n_3} + ( T_0 - T_i^{n_3} ) \frac{M_{frz}}{M_i^{n+1}}, \\
 T^{n+1} &= T^{n_3} + \frac{ - ( c_i T_0 - L_f ) + c_l T^{n_3} }{c_l (\rho_w D - M_{frz}) } M_{frz}.
\end{align}

\end{description}

\subsection{Sea surface albedo}
\subsubsection{Nakajima et al. (2000) model}
\citet{nakajima_2000} provided the albedo for the short wave on the sea surface $A$:
\begin{align}
  A = \exp \left[ \Sigma_{i=1}^{3} \Sigma_{j=1}^{5} C_{ij} t^{j-1} \mu_0^{i-1} \right],
\end{align}
where
$C_{ij}$ is the empirical optical parameters,
$t$ is the flux transmissivity for short-wave radiation,
and $\mu_0$ is cosine of the solar zenith angle.


\subsection{Roughness length}
\subsubsection{Miller et al. (1992) model}
\citet{miller_1992} provides the roughness length over the tropical ocean,
based on numerical calculations by combining smooth surface values
with the Charnock relation for aerodynamic roughness length
and constant values for heat and moisture in accordance with \citet{Smith_1988,Smith_1989} suggestions:
\begin{align}
  z_0 &= 0.11u/\nu_* + 0.018u_*^2/g, \label{eq: z_0} \\
  z_t &= 0.40u/\nu_* + 1.4 \times 10^{-5}, \label{eq: z_t} \\
  z_q &= 0.62u/\nu_* + 1.3 \times 10^{-4}, \label{eq: z_q}
\end{align}
where $\nu_*$ is the kinematic viscosity of air ($\sim 1.5 \times 10^{-5}$), and $z_0, z_t$,
and $z_q$ are the roughness length for momentum, heat, and vapor, respectively.

\subsubsection{Moon et al. (2007) model}
\citet{moon_2007} provides the air--sea momentum flux at high wind speeds
based on the coupled wave--wind model simulations for hurricanes.
At first, the wind speed $U$ at 10-m height is estimated from the previous roughness length $z_0$, as follows:
\begin{align}
  U =\frac{u_{*}}{\kappa} \ln \frac{10}{z_0},
\end{align}
where
$u_{*}$ is friction velocity (m/s)
and $\kappa$ is von Kalman constant.
And then, new roughness length $z_0$ is iteratively estimated from the wind speed:
\begin{equation}
  z_0   = \left\{
  \begin{array}{lll}
    \frac{0.0185}{g} u_{*}^2 & \mathrm{for} & U < 12.5, \\
    \left[ 0.085 \left( -0.56 u_{*}^2 + 20.255 u_{*} + 2.458 \right) - 0.58 \right] \times 10^{-3} & \mathrm{for} & U \ge 12.5.
   \end{array} \right.
\end{equation}

Furthermore, \citet{Fairall_2003} provides the roughness length for the heat and vapor
using that for momentum, as follows:
\begin{align}
  z_t &= \frac{ 5.5 \times 10^{-5} }{ ( z_0 u_{*} / \nu_{*} )^{0.6} }, \\
  z_q &= z_t.
\end{align}

