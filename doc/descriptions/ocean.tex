%\section{Land Physics}
{\bf \Large 
\begin{tabular}{ccc}
\hline
  Corresponding author & : & Tsuyoshi Yamaura\\
\hline
\end{tabular}
}


\subsection{Ocean physics: slab model}

The ocean slab model estimates sea temperature tendencies using a single-layered model.
The temperature tendency equation is estimated from simple heat equation, as follows:
\begin{align}
  \frac{\partial T}{\partial t} = \frac{1}{\rho_{w}c_{l}D} \left( G_{w} + Q \right),
  \label{eq:Ocean-Tdt}
\end{align}
where
$T$ is sea temperature ($K$),
$\rho_{w}$ is water density ($kg/m^3$),
$D$ is water depth of the slab model,
$G_{w}$ is upward heat flux between the Earth's surface and subsurface ($J/m^2/s$),
and $Q$ is external heat source ($J/m^2/s$).

Eq. (\ref{eq:Ocean-Tdt}) is discretized as follows:
\begin{align}
  \frac{\Delta T}{\Delta t} = \frac{1}{C_{w}} \left( G_{w} + Q \right),
\end{align}
where
$C_{w}$ is heat capacity of water ($J/K/m^2$).

\subsection{Sea surface albedo}
Nakajima et al. (2000) provide albedo on the sea surface $A$:
\begin{align}
  A = \exp \left[ \Sigma_{i=1}^{3} \Sigma_{j=1}^{5} C_{ij} t^{j-1} \mu_0^{i-1} \right],
\end{align}
where
$C_{ij}$ is optical parameter table,
$t$ is the flux transmissivity for short-wave radiation,
and $\mu_0$ is the cosine of solar zenith angle.


\subsection{Roughness length}
Miller et al. (1992) provide the roughness length over the tropical ocean,
based on numerical calculations by combining smooth surface values
with the Charnock relation for aerodynamic roughness length
and constant values for heat and moisture in accordance with Smith et al. (1988,1989) suggestions:
\begin{align}
  z_0 &= 0.11u/\nu_* + 0.018u_*^2/g, \label{eq: z_0} \\
  z_t &= 0.40u/\nu_* + 1.4 \times 10^{-5}, \label{eq: z_t} \\
  z_q &= 0.62u/\nu_* + 1.3 \times 10^{-4}, \label{eq: z_q}
\end{align}
where $\nu_*$ is the kinematic viscosity of air ($\sim 1.5 \times 10^{-5}$), and $z_0, z_t$,
and $z_q$ are the roughness length for momentum, heat, and vapor, respectively.

Moon et al. (2007) provide the air--sea momentum flux at high wind speeds
based on the coupled wave--wind model simulations for hurricanes.
At first, the wind speed $U$ at 10-m height is estimated from the previous roughness length $z_0$, as follows:
\begin{align}
  U =\frac{u_{*}}{\kappa} \ln \frac{10}{z_0},
\end{align}
where
$u_{*}$ is friction velocity ($m/s$)
and $\kappa$ is von Kalman constant.
And then, new roughness length $z_0$ is iteratively estimated from the wind spped:
\begin{align}
  z_0 &= \frac{0.0185}{g} u_{*}^2 ~ ( U < 12.5 ), \\
  z_0 &= \left[ 0.085 \left( -0.56 u_{*}^2 + 20.255 u_{*} + 2.458 \right) - 0.58 \right] \times 10^{-3} ~ ( U \ge 12.5 ).
\end{align}

Furthermore, Fairall et al. (2003) provide the roughness length for heat and vapor
using the roughness length for momentum, as follows:
\begin{align}
  z_t &= \frac{ 5.5 \times 10^{-5} }{ ( z_0 u_{*} / \nu_{*} )^{0.6} }, \\
  z_q &= z_t.
\end{align}

