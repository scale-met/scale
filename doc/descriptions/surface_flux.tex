\section{Surface flux}
{\bf \Large 
\begin{tabular}{ccc}
\hline
  Corresponding author & : & Seiya Nishizawa\\
\hline
\end{tabular}
}


\subsection{Monin-Obukhov similarity}
The first of the all,
we assume that in the boundary layer
1. fluxes are constant,
and 2. variables are horizontally uniform.


Relations between flux and vertical gradient are
\begin{align}
  \frac{kz}{u_*} \frac{\partial u}{\partial z} &= \phi_m\left(\frac{z}{L}\right), \label{eq: flux-gradient u} \\
  \frac{kz}{\theta_*} \frac{\partial \theta}{\partial z} &= \phi_h\left(\frac{z}{L}\right), \label{eq: flux-gradient t} \\
  \frac{kz}{q_*} \frac{\partial q}{\partial z} &= \phi_q\left(\frac{z}{L}\right),
\end{align}
where $k$ is the Von Karman constant.
$L$ is the Monin-Obukhov scale height, which is
\begin{equation}
  L = \frac{\theta u_*^2}{kg\theta_*},
\end{equation}
where $g$ is the gravity.
The scaling velocity, $u_*$, temperature, $\theta_*$,
and water vapor, $q_*$, are defined
from the vertical eddy fluxes of momentum, sensible heat and water vapor:
\begin{align}
  \overline{u'w'} &= -u_*u_*, \\
  \overline{w'\theta'} &= - u_*\theta_*, \\
  \overline{w'q'} &= - u_*q_*.
\end{align}

The integration between the roughness length $z_0$ to the height $z$ of the lowest model level, eqs. (\ref{eq: flux-gradient u}) and (\ref{eq: flux-gradient t}) become
\begin{align}
  u(z) &= \frac{u_*}{k} \left\{\ln(z/z_0)-\Phi_m(z/L)+\Phi_m(z_0/L)\right\}, \\
  \Delta\theta &= R\frac{\theta_*}{k} \left\{\ln(z/z_0)-\Phi_h(z/L)+\Phi_h(z_0/L)\right\},
\end{align}
where $\Delta\theta = \theta-\theta_0$,
and
\begin{align}
  \Phi_m(z) = \int^z \frac{1-\phi_m(z')}{z'} dz', \\
  \Phi_h(z) = \int^z \frac{R-\phi_h(z')}{R z'} dz'.
\end{align}


\subsection{Louis (1979) Model}
Louis (1979) introduced a parametric model of vertical eddy fluxes.

The $L$ becomes
\begin{equation}
  L = \frac{\theta u^2}{g\Delta\theta}
    \frac{\ln(z/z_0)-\Phi_h(z/L)+\Phi_h(z_0/L)}{\left\{\ln(z/z_0)-\Phi_m(z/L)+\Phi_m(z/L)\right\}^2}.
\end{equation}
The bulk Richardson number for the layer $Ri_B$ is
\begin{equation}
  Ri_B = \frac{gz\Delta\theta}{\theta u^2},
\end{equation}
and its form implies relationship with the Monin-Obukhov scale height $L$.
Then the fluxes could be written as
\begin{align}
  u_*^2 &= a^2 u^2 F_m\left(\frac{z}{z_0},Ri_B\right), \label{eq: u_*^2} \\
  u_*\theta_* &= \frac{a^2}{R} u \Delta \theta F_h\left(\frac{z}{z_0},Ri_B\right), \label{eq: u_*t_*}
\end{align}
where
$R$ is ratio of the drag coefficients for momentum and heat in the neutral limit, and
\begin{equation}
  a^2 = \frac{k^2}{\left\{\ln\left(z/z_0\right)\right\}^2}
\end{equation}
is the drag coefficient in neutral conditions.

For the unstable condition ($Ri_B<0$),
$F_i$s ($i=m,h$) could be
\begin{equation}
  F_i = 1 - \frac{b Ri_B}{1 + c_i \sqrt{|Ri_B|}},
\end{equation}
under the consideration that
$F_i$ must behave as $1/u$ (i.e. $\sqrt{|Ri_B|}$) in the free convection limit ($u \to 0$),
and becomes $1$ in neutral conditions ($Ri_B \to 0$).
In the stable conditions ($Ri_b$), on the other hand,
Louis (1979) adopted the following form for $F_i$:
\begin{equation}
  F_i = \frac{1}{(1 + b' Ri_B)^2}.
  \label{eq: F_i stable}
\end{equation}

The constants are estimated as
$R=0.74$ by Businger et al. (1971),
and $b=2b'=9.4$ by Louis (1979).
By the dimensional analysis,
\begin{equation}
  c_i = C^*_i a^2 b \sqrt{\frac{z}{z_0}},
\end{equation}
and $C^*_m = 7.4, C^*_h = 5.3$, which result best fit curves.


\subsection{Uno et al. (1995) Model}
Uno et al. (1995) extended the Louis Model,
which considers difference of the roughness length
between for momentum, $z_0$, and temperature, $z_t$.

The potential temperature difference between $z=z$ and $z=z_t$,
$\Delta\theta_t$, is
\begin{align}
  \Delta\theta_t
  &= R\frac{\theta_*}{k}\left\{\ln(z_0/z_t) - \Phi_h(z_0/L) + \Phi_h(z_t/L)\right\} + \Delta\theta_0, \nonumber \\
  &= R\frac{\theta_*}{k}{\ln(z_0/z_t)} + \Delta\theta_0, \nonumber \\
  &= \Delta\theta_0 \left\{\frac{R\ln(z_0/z_t)}{\Psi_h} + 1\right\},
\end{align}
where $\Delta\theta_0 = \theta_z - \theta_{z_0} (=\Delta\theta)$,
\begin{equation}
  \Psi_h = \int_{z_0}^z\frac{\phi_h}{z'}dz', \label{eq: Psi_h}
\end{equation}
and $\phi_h$ is assumed to be $R$ in the range $z_t < z < z_0$.
Thus
\begin{equation}
  \Delta\theta_0 = \Delta\theta_t \left\{\frac{R\ln(z_0/z_t)}{\Psi_h}+1\right\}^{-1},
  \label{eq: Delta t_0}
\end{equation}
or equivalently,
\begin{equation}
  Ri_{B0} = Ri_{Bt} \left\{\frac{R\ln(z_0/z_t)}{\Psi_h}+1\right\}^{-1}.
  \label{eq: Ri_B0}
\end{equation}

From the eqs. (\ref{eq: u_*^2}) and (\ref{eq: u_*t_*}),
\begin{equation}
  \Delta\theta_0 = \frac{R\theta_*}{k}\ln\left(\frac{z}{z_0}\right)\frac{\sqrt{F_m}}{F_h},
\end{equation}
while
\begin{equation}
  \Delta\theta_0 = \frac{\theta_*}{k}\Psi_h,
\end{equation}
from eqs. (\ref{eq: flux-gradient t}) and (\ref{eq: Psi_h}).
Therefore
\begin{equation}
  \Psi_h = R\ln\left(\frac{z}{z_0}\right)\frac{\sqrt{F_m}}{F_h}.
\end{equation}

Because $\Psi_h$ depends on $Ri_{B0}$,
$Ri_{B0}$ cannot be calculated from $Ri_{Bt}$ with eq. (\ref{eq: Ri_B0})
directly,
so numerical iteration is required to obtain $Ri_{B0}$
\footnote{In the stable case, it can be solved analytically
with eq. (\ref{eq: F_i stable}),
but the solution is too complicated.}.
Starting from $Ri_{Bt}$ as the first estimation of $Ri_{B0}$,
the second estimate by the Newton-Raphson iteration becomes
\begin{equation}
  \hat{Ri}_{B0} = Ri_{Bt} - \frac{Ri_{Bt}R\ln(z_0/z_t)}{\ln(z_0/z_t) + \hat{\Psi_h}},
\end{equation}
where $\hat{\Psi_h}$ is the estimate of $\Psi_h$ using $Ri_{Bt}$ instead of $Ri_{B0}$.
Approximate values for $F_m, F_h$, and $\Psi_h$ are re-calculated
based on the $\hat{Ri}_{B0}$,
and then
$\Delta\theta_0$, and the surface fluxes $u_*^2$ and $u_*\theta_*$
are calculated from eqs. (\ref{eq: Delta t_0}), (\ref{eq: u_*^2}),
and (\ref{eq: u_*t_*}), respectively.


\subsection{Roughness length}
Miller et al. (1992) provides the roughness length over the tropical ocean,
based on the numerical calculations
by combining the smooth surface values with the Charnock relation
for the aerodynamic roughness length
and the constant values for heat and moisture in accordance with Smith (1988,1989) suggestions:
\begin{align}
  z_{0M} &= 0.11u/\nu_* + 0.018u_*^2/g, \\
  z_{0H} &= 0.40u/\nu_* + 1.4 \times 10^{-5}, \\
  z_{0Q} &= 0.62u/\nu_* + 1.3 \times 10^{-4},
\end{align}
where $z_{0M}, z_{0H}$, and $z_{0Q}$ are $z_0$ for the momentum, heat, and vapor,
respectively.
