{\bf \Large
\begin{tabular}{ccc}
\hline
  Corresponding author & : & Sachiho A. Adachi\\
\hline
\end{tabular}
}
\\


Urban models calculate energy exchanges between the urban surface (urban canopy) and atmosphere.
SCALE-RM has two options for calculation of urban areas: LAND and KUSAKA01.
In LAND option, fluxes over urban subtiles are calculated by the land model you chose for your examination.
Please refer to Section \ref{sec:land}.

\subsection{KUSAKA01: Single layer urban canopy model}

KUSAKA01 is the single layer urban canopy model by \citet{kusaka_2001} and \citet{kusaka_2004} (henceforth, UCM).
The UCM of KUSAKA01 assumes street canyons as the urban geometry.
The model estimates prognostic variables of the surface temperatures and heat fluxes from three surfaces, i.e., roofs, walls, and roads,
and those from urban canopy.
The model assumes that the urban canopy layer is located lower than the 1st layer of atmospheric model.
Therefore, there is a restriction that a sum of displacement height and roughness length must be two meters lower than the face level of 1st layer of atmosphere.
In this version, single urban type can be considered in the urban model.
Several parameters are prepared to specify the urban geometry and anthropogenic heat. 
