%%%%%%%%%%%%%%%%%%%%%%%%%%%%%%%%%%%%%%%%%%%%%%%%%%%%%%%%%%%%%%%%%%%%%
%  File 31_ideal_exp.tex
%%%%%%%%%%%%%%%%%%%%%%%%%%%%%%%%%%%%%%%%%%%%%%%%%%%%%%%%%%%%%%%%%%%%%
\section{概要} \label{sec:ideal_exp_intro}

本章では、SCALE-RMを用いて一連の実験を行うための基本的な操作について、
チュートリアル用に準備した理想実験のテストケースを題材に説明する。\\
第\ref{chap:install}章で実行したSCALEのコンパイルが正常に完了しているかどうかの
チェックも兼ねているのでぜひ実施してもらいたい。

本章では、SCALEのコンパイルが正常に終了し、
すでに下記のファイルが生成されているものとして説明を行う。
\begin{alltt}
  scale-{\version}/bin/scale-rm
  scale-{\version}/bin/scale-rm_init
  scale-{\version}/scale-rm/util/netcdf2grads_h/net2g
\end{alltt}
これらに加えて、描画ツールとして\grads を使用する。
オプションとして、gpviewは、結果の確認用に利用することができる。
\grads およびgpview(GPhys)との詳細やインストール方法については、
付録 \ref{sec:env_vis_tools}節を参照のこと。




