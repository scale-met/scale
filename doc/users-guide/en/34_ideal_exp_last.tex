\section{Guideline for further study} \label{sec:ideal_exp_last}

In this chapter,
the method for the execution of \scalerm was explained by using a simple ideal experiment. We recommend studying methods of changing the model resolution, the calculation domain, and the number of MPI processes for further study.  With regard to the ideal experiment, several files of other configurations,  e.g., to increase resolution, the number of domains, and the physical scheme, are prepared in the directory ``sample'' under the same directory as was used in this experiment.  These configuration files are useful to change such configurations.  Moreover, various ideal experimental settings  have been prepared in the directory ``\verb|scale-rm/test/case|.'' For some ideal experiments,  it may be necessary to carry out the ``make'' command again in the same directory as in the configuration file  because some test cases need special source codes according to their experimental settings. The procedures for the generation of the initial conditions and those for simulation execution are the same as in the tutorial in this chapter.

It is important to study the method for the configuration of physical processes, such as cloud microphysics, radiation, and turbulence schemes. Methods to alter them in detail are described in Chapter \ref{chap:basic_usel}.

