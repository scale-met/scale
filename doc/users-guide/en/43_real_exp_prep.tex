%-------------------------------------------------------%
\section{Preparation for experiment set} \label{sec:tutrial_real_prep}
%-------------------------------------------------------%

In the real atmospheric experiment,
many execution procedures and a large number of files are needed in comparison with the ideal experiment.
In addition, it is necessary to maintain consistency between settings in the configuration file (\verb|***.conf|)
between pre-processing \verb|pp|, initial value making \verb|init|, and simulation execution \verb|run|.
The inconsistency between these settings and the lack of files in the preparation step may cause an abnormal model run.
To avoid such situations, the tool "the making tool for the complete settings of the experiment"
is prepared for the generation of a set of necessary files.
You first move to the following directory and prepare a series of files for the tutorial for the real atmospheric experiment using the next procedure:
\begin{verbatim}
 $ cd ${Tutorial_DIR}/real/
 $ ls
    Makefile : Makefile for generation of a set of necessary files.
    README   : README related to use of the script
    USER.sh  : Description of experimental setting.
    config/  : Each of configurations for generation of a set of files
               ( basically, unnecessary for users to be rewritten)
    sample/  : sample script of USERS.sh
    data/    : tools for the tutorial
    tools/   : tools for initial condition used in the tutorial 
               (basically, users do it themselves except for the tutorial case)
 $ make
 $ ls experiment/    : directories added by the above make command
    init/
    net2g/
    pp/
    run/
\end{verbatim}

According to the settings described in \verb|USER.sh|,
an experiment set is generated under the directory \verb|experiment| when the make command is executed.
Refer to Section \ref{sec:basic_makeconf} for a detailed explanation of "the making tool for the complete settings of the experiment."
%Further, in case of nesting, the available files are prepared for the directory \verb|sample|. Referred to them if needed.

