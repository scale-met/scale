\section{How to restart run}\label{sec:restart}
%=======================================================================
The restart function is beneficial in limiting the job execution time of a computer system and in cases of unexpected termination in a long simulation.
Such run can usually be executed by being divided into multiple sequential runs using this function. In the restart run, it is also possible to output restart files for the next restart simulation. The file outputs for restart are configured in \namelist{PARAM_RESTART} and \namelist{PARAM_TIME} in \verb|run.conf|. The example below indicates that the restart run starts using the initial file \verb|restart1_***|, and generates restart files called \verb|restart2_***| every six hours.
\editboxtwo{
\verb|&PARAM_RESTART| & \\
\multicolumn{2}{l}{\verb| RESTART_IN_BASENAME  = "restart1_d01_20070715-000000.000",|} \\
                                                                             & File name of input initial (restart) file.\\
                                                                             & \\ 
\verb| RESTART_IN_POSTFIX_TIMELABEL  = .false.                             | & Whether initial date and time are added after \verb|RESTART_IN_BASENAME| or not.\\
\verb| RESTART_OUTPUT       = .true.,                                      | & Setting of restart file output\\
                                                                             & \verb|.true.|: Create restart files, \verb|.false.|: Not create\\
\verb| RESTART_OUT_BASENAME = "restart2_d01",                              | & Head of output restart file name\\
                                                                             & After the head name, the output date is added.\\
\verb| RESTART_OUT_POSTFIX_TIMELABEL = .true.                              | & Whether initial date and time are added after \verb|RESTART_OUT_BASENAME| or not.\\
\verb|/| & \\
\\
\verb|&PARAM_TIME| & \\
\verb| TIME_STARTDATE             = 2007, 7, 15, 00, 0, 0,| & Start date of restart run\\
\verb| TIME_STARTMS               = 0.D0,                 | & Start date [ms]\\
\verb| TIME_DURATION              = 12.0D0,               | & Integration time [Unit is defined by \verb|TIME_DURATION_UNIT|]\\
\verb| TIME_DURATION_UNIT         = "HOUR",               | & Unit for \verb|TIME_DURATION|\\
\verb|  ..... *snip* .....                                | & \\
\verb| TIME_DT_ATMOS_RESTART      = 21600.D0,             | & Output interval of restart files for atmospheric variables\\
\verb| TIME_DT_ATMOS_RESTART_UNIT = "SEC",                | & Unit for \verb|TIME_DT_ATMOS_RESTART|\\
\verb| TIME_DT_OCEAN_RESTART      = 21600.D0,             | & Output interval of restart files for ocean variables\\
\verb| TIME_DT_OCEAN_RESTART_UNIT = "SEC",                | & Unit for \verb|TIME_DT_OCEAN_RESTART|\\
\verb| TIME_DT_LAND_RESTART       = 21600.D0,             | & Output interval of restart files for land variables\\
\verb| TIME_DT_LAND_RESTART_UNIT  = "SEC",                | & Unit for \verb|TIME_DT_LAND_RESTART|\\
\verb| TIME_DT_URBAN_RESTART      = 21600.D0,             | & Output interval of restart files for urban variables\\
\verb| TIME_DT_URBAN_RESTART_UNIT = "SEC",                | & Unit for \verb|TIME_DT_URBAN_RESTART|\\
\verb|/| & \\
}
The time intervals for output of restart files are specifided by \nmitem{TIME_DT_ATMOS_RESTART}, \\
\nmitem{TIME_DT_OCEAN_RESTART},  \nmitem{TIME_DT_LAND_RESTART}, and \nmitem{TIME_DT_URBAN_RESTART}. When they are not specified, the restart files are generated at the end of the simulation, i.e., \nmitem{TIME_DURATION}.

The normal run is performed by specifying \nmitem{RESTART_IN_BASENAME} $=$ \verb|init_***|, which are prepared by \verb|scale-rm_init|.
The restart run is performed by assigning the restart file name to \nmitem{RESTART_IN_BASENAME}.

The file names of input and output restart files are specified by \nmitem{RESTART_IN_BASENAME} and \nmitem{RESTART_OUT_BASENAME}, respectively.
\nmitem{RESTART_IN_POSTFIX_TIMELABEL}, \\ \nmitem{RESTART_OUT_POSTFIX_TIMELABEL} indicate
whether date and time at input and output are automatically added to the file name after \nmitem{RESTART_IN_BASENAME}, \nmitem{RESTART_OUT_BASENAME}, respectively.
The default setting is \nmitem{RESTART_IN_POSTFIX_TIMELABEL = .false.} \\ and \nmitem{RESTART_OUT_POSTFIX_TIMELABEL=.true.}.
In avobe example, setting \nmitem{RESTART_IN_BASENAME} \verb|="restart1_d01_20070715-000000.000"| is equivalent to
setting \nmitem{RESTART_IN_POSTFIX_TIMELABEL = .true.} and \nmitem{RESTART_IN_BASENAME} \verb|="restart1_d01"|.

The restart run is independent of the calculation that generates a restart file for the run.
Note that the output variables in the restart file must be sufficient for the next restart run because the prognostic variables depend on the scheme used, especially in case of physical schemes.
The simplest way for preparing the sufficient variables in the restart file is
to use the same dynamic and physics schemeses, that would be used in the restart simulation, in the simulation creating the restart file.
\nmitem{TIME_STARTDATE} and \nmitem{TIME_DURATION} represent the start date and the integration time for the restart simulation.

For a realistic atmospheric experiment, the boundary data prepared by \verb|scale-rm_init| is needed in addition to the initial data. An example is as follows:
\editboxtwo{
\verb|&PARAM_ATMOS_BOUNDARY| & \\
\verb| ATMOS_BOUNDARY_TYPE           = "REAL",                            | & \verb|"REAL"|: Real case experiment\\
\verb| ATMOS_BOUNDARY_IN_BASENAME    = "../init/output/boundary_d01",     | & Head of file name of boundary data\\
\verb| ATMOS_BOUNDARY_START_DATE     = 2010, 7, 14, 18, 0, 0,             | & Initial date of boundary file\\
\verb| ATMOS_BOUNDARY_UPDATE_DT      = 21600.D0,                          | & Time interval of boundary data\\
\verb|/| & \\
}
The boundary data at appropriate date are read in a restart simulation by specifying the first date of the boundary data in \verb|boundary_***.nc|  at \nmitem{ATMOS_BOUNDARY_START_DATE} in \namelist{PARAM_ATMOS_BOUNDARY}. When \nmitem{ATMOS_BOUNDARY_START_DATE} is not given, \scalerm treats the first data in \verb|boundary_***.nc|  as the boundary condition at the initial date of the restart simulation.

