\section{How to restart run}\label{sec:restart}
%=======================================================================
The restart function is beneficial to avoid unexpected termination of a simulation because of limitation of the job execution time controlled by a computer system.
In such cases, the simulation can be executed by being divided into multiple sequential runs using this function.
It is also possible to output restart files for the next restart simulation during restart run.
The settings for restart files are configured in \namelist{PARAM_RESTART} and \namelist{PARAM_TIME} in \verb|run.conf|.
The example below indicates that the restart run starts using the initial file \verb|restart1_***|, and generates restart files called \verb|restart2_***| every six hours.
\editboxtwo{
\verb|&PARAM_RESTART| & \\
\multicolumn{2}{l}{\verb| RESTART_IN_BASENAME           = "restart1_d01_20070715-000000.000",   |}\\
                                                                  & Basename of input initial (restart) file.\\
%\verb| RESTART_IN_AGGREGATE          = .false.,                 | & Please refer to Section \ref{sec:netcdf}\\
\verb| RESTART_IN_POSTFIX_TIMELABEL  = .false.                  | & Whether or not to add initial date and time after \verb|RESTART_IN_BASENAME|.\\
\verb| RESTART_OUTPUT                = .false.,                 | & Whether or not to output restart file \\
                                                                  & \verb|.true.|: Output, \verb|.false.|: Not output\\
                                                                  & Default is \verb|.false.|\\
\verb| RESTART_OUT_BASENAME          = "restart2_d01",          | & Basename of output restart file.\\
%\verb| RESTART_IN_AGGREGATE          = .false.,                 | & Please refer to Section \ref{sec:netcdf}\\
\verb| RESTART_OUT_POSTFIX_TIMELABEL = .true.                   | & Whether or not to add output date and time after \verb|RESTART_OUT_BASENAME|.\\
\verb| RESTART_OUT_TITLE             = "",                      | & Title written in the restart file.\\
\verb| RESTART_OUT_DTYPE             = "DEFAULT",               | & \verb|REAL4| or \verb|REAL8| or \verb|DEFAULT|\\
\verb|/| & \\
\\
\verb|&PARAM_TIME| & \\
\verb| TIME_STARTDATE             = 2007, 7, 15, 00, 0, 0,| & Start date of restart run\\
\verb| TIME_STARTMS               = 0.D0,                 | & Start date [ms]\\
\verb| TIME_DURATION              = 12.0D0,               | & Integration time\\
\verb| TIME_DURATION_UNIT         = "HOUR",               | & Unit for \verb|TIME_DURATION|\\
\verb|  ..... *snip* .....                                | & \\
\verb| TIME_DT_ATMOS_RESTART      = 21600.D0,             | & Output interval of restart files for atmospheric variables\\
\verb| TIME_DT_ATMOS_RESTART_UNIT = "SEC",                | & Unit for \verb|TIME_DT_ATMOS_RESTART|\\
\verb| TIME_DT_OCEAN_RESTART      = 21600.D0,             | & Output interval of restart files for ocean variables\\
\verb| TIME_DT_OCEAN_RESTART_UNIT = "SEC",                | & Unit for \verb|TIME_DT_OCEAN_RESTART|\\
\verb| TIME_DT_LAND_RESTART       = 21600.D0,             | & Output interval of restart files for land variables\\
\verb| TIME_DT_LAND_RESTART_UNIT  = "SEC",                | & Unit for \verb|TIME_DT_LAND_RESTART|\\
\verb| TIME_DT_URBAN_RESTART      = 21600.D0,             | & Output interval of restart files for urban variables\\
\verb| TIME_DT_URBAN_RESTART_UNIT = "SEC",                | & Unit for \verb|TIME_DT_URBAN_RESTART|\\
\verb|/| & \\
}

The time intervals for output of restart files are specifided by \nmitem{TIME_DT_ATMOS_RESTART}, \\
\nmitem{TIME_DT_OCEAN_RESTART},  \nmitem{TIME_DT_LAND_RESTART}, and \nmitem{TIME_DT_URBAN_RESTART}.
When they are not specified, the restart files are generated at the end of the simulation, i.e., \nmitem{TIME_DURATION}.
The file names of output restart files are specified by \nmitem{RESTART_OUT_BASENAME}.\\
\nmitem{RESTART_OUT_POSTFIX_TIMELABEL} indicate
whether date and time at output are automatically added to the file name after \nmitem{RESTART_OUT_BASENAME}.\\
The default setting is \nmitem{RESTART_OUT_POSTFIX_TIMELABEL=.true.}.

The restart run is independent of the calculation that generated a restart file for the restart run.
Note that the output variables in the restart file must be consistent for the restart run
because the prognostic variables depend on the used scheme, especially for physical schemes.
The simplest way for preparing the restart file including variables required in next restart run is
to use the same dynamic and physics schemeses, that would be used in the restart simulation, in the simulation creating the restart file.

The other settings are basically the same as the normal run.
\nmitem{RESTART_IN_BASENAME} is input file name including initial data.
The normal run usually uses \verb|init_***| prepared by \verb|scale-rm_init|,
while the restart run uses a restart file output by the previous run.
%
\nmitem{RESTART_IN_POSTFIX_TIMELABEL} is the same as \nmitem{RESTART_OUT_POSTFIX_TIMELABEL},
but for \nmitem{RESTART_IN_BASENAME}.
The default setting is \nmitem{RESTART_IN_POSTFIX_TIMELABEL = .false.}.\\
In avobe example, setting \nmitem{RESTART_IN_BASENAME} \verb|="restart1_d01_20070715-000000.000"| is equivalent to
setting \nmitem{RESTART_IN_POSTFIX_TIMELABEL = .true.} and \nmitem{RESTART_IN_BASENAME} \verb|="restart1_d01"|.
\nmitem{TIME_STARTDATE} and \nmitem{TIME_DURATION} represent the start date and the integration time for the restart simulation.



For a realistic atmospheric experiment, the boundary data prepared by \verb|scale-rm_init|
is needed in addition to the initial data. An example is as follows:
\editboxtwo{
\verb|&PARAM_ATMOS_BOUNDARY| & \\
\verb| ATMOS_BOUNDARY_TYPE           = "REAL",                            | & \verb|"REAL"|: Real case experiment\\
\verb| ATMOS_BOUNDARY_IN_BASENAME    = "../init/output/boundary_d01",     | & Head of file name of boundary data\\
\verb| ATMOS_BOUNDARY_START_DATE     = 2010, 7, 14, 18, 0, 0,             | & Initial date of boundary file\\
\verb| ATMOS_BOUNDARY_UPDATE_DT      = 21600.D0,                          | & Time interval of boundary data\\
\verb|/| & \\
}
The boundary data at appropriate date are read in a restart simulation by specifying the first date of the boundary data in \verb|boundary_***.nc|  at \nmitem{ATMOS_BOUNDARY_START_DATE} in \namelist{PARAM_ATMOS_BOUNDARY}. When \nmitem{ATMOS_BOUNDARY_START_DATE} is not given, \scalerm treats the first data in \verb|boundary_***.nc|  as the boundary condition at the initial date of the restart simulation.

