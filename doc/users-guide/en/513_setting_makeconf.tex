\section{Supporting tool for preparation of configuration files} \label{sec:basic_makeconf}
%------------------------------------------------------

In the current version, the configuration files \verb|***.conf| for the experiment are individually prepared for \verb|pp, init, run|. Some items are required to be the same among these configure files. If there is an inconsistency among them, the model does not run properly. To avoid such mistakes, a convenient supporting tool for the preparation of configuration files is provided as follows:
\begin{verbatim}
 $ cd ${Tutorial_DIR}/real/
 $ ls
    Makefile 
      : Makefile for generation of necessary files for experiment
    README
      : README file
    USER.sh
      : Shell script for specifying the settings
    config/  
      : Configuration files for each setting (not necessary for user to rewrite it)
    sample
      : Sample script of USER.sh
\end{verbatim}
Although this tool is intended for the tutorial experiment of real atmosphere as initial setting, users can change the settings in \verb|USER.sh|.

Under the directory \verb|sample/|,   several sample scripts are provided for the typical configuration. Use them as necessary by copying them to \verb|USER.sh|.
\begin{verbatim}
 $ ls sample/
   USER.default.sh                 
      : USER.sh same as for the tutorial experiment of real atmosphere
       ( single domain )
   USER.offline-nesting-child.sh   
      : for the child domain for off-line nesting experiment
   USER.offline-nesting-parent.sh  
      : for the parent domain for off-line nesting experiment
   USER.online-nesting.sh          : for the on-line nesting experiment
\end{verbatim}


\subsubsection{How to use the tool}

The use of this tool is described as follows (or refer to README):
\begin{enumerate}
  \item Edit \verb|USER.sh| according to experimental settings intended by the users.
  \item Execute \verb|make| command.
\end{enumerate}
 All the necessary files for the experiment are generated under the directory \verb|experiment|.

Since the configuration in \verb|USER.sh| is that for \verb|tutorial| of real atmosphere,
it is recommended to leave the tutrial setting as a different file as follows: 
\begin{verbatim}
 $ mv experiment/ tutorial/    
        : (In the case of already existing experiment directory)
 $ cp USER.sh USER_tutorial.sh
 ... edit USER.sh ...
 $ make
 $ cp -rL experiment/ arbitrary_directory/
        : ``arbitrary_place'' means  directory name of arbitrary place
\end{verbatim}


\subsubsection{Editing \texttt{USER.sh}}

%Copy the script in the sample programs closest to the intended experiment to \verb|USER.sh|.
\verb|NUMB_DOMAIN| specifies the number of domains appearing in the script. Change the items below appropriately; as many items following \verb|"# require  parameters for each domain"|  are described as the number of domains. Note that if the number of nesting domains according to \verb|NUM_DOMAIN| and the number of variables are different, the set of files for the experiment cannot be generated. When items do not appear in \verb|USER.sh|, edit directly the configuration file generated under the directory \verb|experiment|.
