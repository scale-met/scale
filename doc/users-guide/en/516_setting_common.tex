%\section{Setting the common components} \label{sec:common}
%------------------------------------------------------

%-------------------------------------------------------------------------------
\section{Setting Physical Constants} \label{subsec:const}
%-------------------------------------------------------------------------------

The changeable physical constants are specified in \namelist{PARAM_CONST} in files \verb|init.conf| and \verb|run.conf|.

\editboxtwo{
\verb|&PARAM_CONST                           | & \\
\verb| CONST_RADIUS            = 6.37122D+6, | & ; Radius of the planet [m] \\
\verb| CONST_OHM               = 7.2920D-5,  | & ; Angular velocity of the planet [1/s] \\
\verb| CONST_GRAV              = 9.80665D0,  | & ; Standard acceleration of gravity [m/s2] \\
\verb| CONST_Rdry              = 287.04D0,   | & ; Specific gas constant (dry air) [J/kg/K] \\
\verb| CONST_CPdry             = 1004.64D0,  | & ; Specific heat (dry air,constant pressure) [J/kg/K] \\
\verb| CONST_LAPS              = 6.5D-3,     | & ; Lapse rate of ISA [K/m] \\
\verb| CONST_Pstd              = 101325.D0,  | & ; Standard pressure [Pa] \\
\verb| CONST_PRE00             = 100000.D0,  | & ; Pressure reference [Pa] \\
\verb| CONST_Tstd              = 288.15D0,   | & ; Standard temperature (15C) [K] \\
\verb| CONST_THERMODYN_TYPE    = 'EXACT',    | & ; Internal energy type \\
\verb| CONST_SmallPlanetFactor = 1.D0,       | & ; Factor for small planet [1] \\
\verb|/                                      | & \\
}

\noindent
When \nmitem{CONST_THERMODYN_TYPE} is 'EXACT', temperature dependencies of the latent heat are considered.
%
When \nmitem{CONST_THERMODYN_TYPE} is 'SIMPLE', specific heat of water categories are set to that of dry air,
and temperature dependencies of the latent heat are ignored.
%
When \nmitem{CONST_SmallPlanetFactor} is set, both radius and angular velocity of the planet are changed as follows;

\begin{eqnarray}
   && \nmitemeq{CONST_RADIUS} = \nmitemeq{CONST_RADIUS} \times \nmitemeq{CONST_SmallPlanetFactor}, \\
   && \nmitemeq{CONST_OHM}    = \frac{\nmitemeq{CONST_OHM}}{\nmitemeq{CONST_SmallPlanetFactor}}
\end{eqnarray}



%-------------------------------------------------------------------------------
\section{Setting Calendar} \label{subsec:calendar}
%-------------------------------------------------------------------------------

The calendar is specified in \namelist{PARAM_CALENDAR} in files \verb|init.conf| and \verb|run.conf|.
Gregorian calendar is used in default.

\editboxtwo{
\verb|&PARAM_CALENDAR             | & \\
\verb| CALENDAR_360DAYS = .false. | & ; Whether 12x30 days calendar is used? \\
\verb| CALENDAR_365DAYS = .false. | & ; Whether leap year is considered? \\
\verb|/                           | & \\
}

\noindent
The setting of the calendar affects calculation of solar zenith angle.
It is calculated so that length of one year and the full circle of the ecliptic match.
Note that the external data in the different calendar should not be read.

When \nmitem{CALENDAR_360DAYS} is \verb|.true.|,
the calendar that one year is 12 months and one month is 30 days, is set.
%
When \nmitem{CALENDAR_365DAYS} is \verb|.true.|,
the Gregorian calendar is used without leap year.



%-------------------------------------------------------------------------------
\section{Setting Random Number Generator} \label{subsec:random}
%-------------------------------------------------------------------------------

The random number generator is specified in \namelist{PARAM_RANDOM} in files \verb|init.conf| and \verb|run.conf|.

\editboxtwo{
\verb|&PARAM_RANDOM         | & \\
\verb| RANDOM_FIX = .false. | & ; Random seed is fixed? \\
\verb|/                     | & \\
}

\noindent
The intrinsic function of random number generator is used in the scale library. Note that the generated number is pseudorandom.
The seed of the random number is determined by current datetime, cpu time, and the process id.
%
When \nmitem{RANDOM_FIX} is \verb|.true.|, the seed is fixed by specific number.
This option is useful to reproduce the simulation results, which use the random perturbation for the initial field.
