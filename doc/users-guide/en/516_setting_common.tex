%\section{Setting the common components} \label{sec:common}
%------------------------------------------------------

%-------------------------------------------------------------------------------
\section{Setting Physical Constants} \label{subsec:const}
%-------------------------------------------------------------------------------

The some physical constants are able to modify in \namelist{PARAM_CONST} in the configuration file.

\editboxtwo{
\verb|&PARAM_CONST                           | & \\
\verb| CONST_RADIUS            = 6.37122D+6, | & ; Radius of the planet [m] \\
\verb| CONST_OHM               = 7.2920D-5,  | & ; Angular velocity of the planet [1/s] \\
\verb| CONST_GRAV              = 9.80665D0,  | & ; Standard acceleration of gravity [m/s2] \\
\verb| CONST_Rdry              = 287.04D0,   | & ; Specific gas constant (dry air) [J/kg/K] \\
\verb| CONST_CPdry             = 1004.64D0,  | & ; Specific heat (dry air,constant pressure) [J/kg/K] \\
\verb| CONST_LAPS              = 6.5D-3,     | & ; Lapse rate of ISA [K/m] \\
\verb| CONST_Pstd              = 101325.D0,  | & ; Standard pressure [Pa] \\
\verb| CONST_PRE00             = 100000.D0,  | & ; Pressure reference [Pa] \\
\verb| CONST_Tstd              = 288.15D0,   | & ; Standard temperature (15C) [K] \\
\verb| CONST_THERMODYN_TYPE    = 'EXACT',    | & ; Internal energy type \\
\verb| CONST_SmallPlanetFactor = 1.D0,       | & ; Factor for small planet [1] \\
\verb|/                                      | & \\
}

\noindent
When \nmitem{CONST_THERMODYN_TYPE} is 'EXACT', temperature dependencies of the latent heat are considered.
%
When \nmitem{CONST_THERMODYN_TYPE} is 'SIMPLE', specific heat of water categories are set to that of dry air,
and temperature dependencies of the latent heat are ignored.
%
The \nmitem{CONST_RADIUS} is multiplied by the \nmitem{CONST_SmallPlanetFactor}.
At the same time, \nmitem{CONST_OHM} is multiplied by the inverse of the \nmitem{CONST_SmallPlanetFactor}.



%-------------------------------------------------------------------------------
\section{Setting Calendar} \label{subsec:calendar}
%-------------------------------------------------------------------------------

The calendar type can be specified in \namelist{PARAM_CALENDAR} in the configuration file.
Gregorian calendar is used in default.

\editboxtwo{
\verb|&PARAM_CALENDAR             | & \\
\verb| CALENDAR_360DAYS = .false. | & ; Whether 12x30 days calendar is used? \\
\verb| CALENDAR_365DAYS = .false. | & ; Whether leap year is considered? \\
\verb|/                           | & \\
}

\noindent
The setting of the calendar affects calculation of solar zenith angle.
It is calculated so that length of one year and the full circle of the ecliptic match.
Note that the external data in the different calendar should not be read.

When \nmitem{CALENDAR_360DAYS} is \verb|.true.|,
the calendar that one year is 12 months and one month is 30 days, is set.
%
When \nmitem{CALENDAR_365DAYS} is \verb|.true.|,
the Gregorian calendar is used without leap year.



%-------------------------------------------------------------------------------
\section{Setting Random Number Generator} \label{subsec:random}
%-------------------------------------------------------------------------------

A parameter of the random number generator is set in \namelist{PARAM_RANDOM} in the configuration file.

\editboxtwo{
\verb|&PARAM_RANDOM         | & \\
\verb| RANDOM_FIX = .false. | & ; Random seed is fixed? \\
\verb|/                     | & \\
}

\noindent
The intrinsic function of random number generator is used in the scale library. Note that the generated number is pseudorandom.
The seed of the random number is determined by current datetime, cpu time, and the process id.
%
When \nmitem{RANDOM_FIX} is \verb|.true.|, the seed is fixed by specific number.
This option is useful to reproduce the simulation results, which use the random perturbation for the initial field.



%-------------------------------------------------------------------------------
\section{Setting Performance Profiler} \label{subsec:prof}
%-------------------------------------------------------------------------------

Parameters of the performance profiler are set in \namelist{PARAM_PROF} in the configuration file.

\editboxtwo{
\verb|&PARAM_PROF                 | & \\
\verb| PROF_rap_level   = 2       | & ; Rap output level \\
\verb| PROF_mpi_barrier = .false. | & ; Add barrier command of MPI in every rap? \\
\verb|/                           | & \\
}

\noindent
To measure the elapse time, utility functions (PROF\_rapstart,PROF\_rapend) are inserted in the source code.
These measurement sections are also used for detailed performance profiling.
%
The results of the rap time are displayed at the end of the log file.
If \nmitem{IO_LOG_ALLNODE} in  \namelist{PARAM_IO} is \verb|.true.|, the result of each process is reported to each log file individually.
If \nmitem{IO_LOG_SUPPRESS} in \namelist{PARAM_IO} is \verb|.true.|, the result is sent to the standard output.
%
Each measurement section has the output level. The section, which has the output level larger than the \nmitem{PROF_rap_level} are not measured.

When \nmitem{PROF_mpi_barrier} is \verb|.true.|, barrier command of MPI is called before and after getting current time.
This option is useful to separate the computation time and communication time.
The computation time often shows large imbalance among the processes.



%-------------------------------------------------------------------------------
\section{Setting Statistics Monitor} \label{subsec:statistics}
%-------------------------------------------------------------------------------

Parameters of the statistics monitor are set in \namelist{PARAM_STATISTICS} in the configuration file.

\editboxtwo{
\verb|&PARAM_STATISTICS                    | & \\
\verb| STATISTICS_checktotal     = .false. | & ; Calculate and report variable totals to logfile? \\
\verb| STATISTICS_use_globalcomm = .false. | & ; Calculate total with global communication? \\
\verb|/                                    | & \\
}

\noindent
When \nmitem{STATISTICS_checktotal} is \verb|.true.|, domain total of some variables are calculated and reported to the log file, for debugging.
%
When \nmitem{STATISTICS_use_globalcomm} is \verb|.true.|, global domain total is calculated by using global communication.
This option can slow the simulation time.
When the option is \verb|.false.|, the total is calculated in the spatial region allocated for each process.



