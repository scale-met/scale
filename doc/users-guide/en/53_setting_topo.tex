\section{Setting the topography} \label{subsec:basic_usel_topo}
%-----------------------------------------------------------------------

\scalerm employs the terrain-following coordinates to represent topography.
In these coordinates, the bottom face of the lowest grid is given such that it can follow the surface altitude. The maximum allowable angle of inclination, $\theta_{\max}$[radian] is calculated as follows:
\begin{eqnarray}
  && \theta_{\max} = \arctan( \mathrm{RATIO} \times \mathrm{DZ}/\mathrm{DX} )\nonumber,
\end{eqnarray}
where $\mathrm{DZ}$ and $\mathrm{DX}$ are the horizontal and vertical grid intervals, respectively.
As shown in the above equation, $\theta_{\max}$ depends on spatial resolution.
If $\mathrm{RATIO}$ is greater than unity, the fine topography is expressed, and vice versa.
Note that if $\mathrm{RATIO}$ is set to a greatly large value, the risk of numerical instability increases.
The default value of $\mathrm{RATIO}$ is 5.0.

The program \verb|scale-rm_pp| converts external topography data into \scalelib format.
The detailed configurations are specified by \namelist{PARAM_CNVTOPO} in \verb|pp.conf|.
\editboxtwo{
\verb|&PARAM_CNVTOPO                               | & \\
\verb|CNVTOPO_name                  = 'NONE',      | & ; "\verb|NONE|","\verb|GTOPO30|","\verb|DEM50M|","\verb|USERFILE|"\\ %"COMBINE"
\verb|CNVTOPO_UseGTOPO30            = .false.,     | & ; Use GTOPO30 dataset? \\
\verb|CNVTOPO_UseDEM50M             = .false.,     | & ; Use DEM50M dataset? \\
\verb|CNVTOPO_UseUSERFILE           = .false.,     | & ; Use user-defined dataset? \\
\verb|CNVTOPO_smooth_trim_ocean     = .true.       | & ; \\
\verb|CNVTOPO_smooth_type           = 'LAPLACIAN', | & ; Type of smoothing filter \\
                                                     & ~~~ ("\verb|OFF|", "\verb|LAPLACIAN|", "\verb|GAUSSIAN|") \\
\verb|CNVTOPO_smooth_maxslope_ratio =  5.D0,       | & ; Maximum allowable ratio of inclination to $\mathrm{DZ}$/$\mathrm{DX}$ \\
\verb|CNVTOPO_smooth_maxslope       = -1.D0,       | & ; Maximum allowable angle of inclination [deg] \\
\verb|CNVTOPO_smooth_local          = .true.,      | & ; Try to continue smoothing, for only grids whose angles of inclination exceed the maximum value? \\
\verb|CNVTOPO_smooth_itelim         = 10000,       | & ; Number limit of the smoothing iteration \\
\verb|CNVTOPO_smooth_hypdiff_niter  = 20,          | & ; Number of the smoothing iteration by hyperdiffusion \\
\verb|CNVTOPO_smooth_hypdiff_order  = 4,           | & ; \\
%\verb|CNVTOPO_interp_level          = 5,           | & ; Number of the neighbor grid points for interpolation \\
\verb|CNVTOPO_copy_parent           = .false.,     | & ; The topography in the buffer region of child domain is copied from parent domain? \\
\verb|/                                            | \\
}



The default settings of \nmitem{CNVTOPO_(UseGTOPO30|UseDEM50M|UseUSERFILE)} are \verb|.false.|.
The default setting of \nmitem{CNVTOPO_name} is \verb|NONE|.
If you specified something to \nmitem{CNVTOPO_name},
each of \nmitem{CNVTOPO_(UseGTOPO30|UseDEM50M|UseUSERFILE)} is automatically set as shown in Table \ref{tab:cvntopo_name}.
The mark of asterisk ($\ast$) in Table \ref{tab:cvntopo_name} indicates that
if \nmitem{CNVTOPO_(UseGTOPO30|UseDEM50M|UseUSERFILE)} are specified in \verb|pp.conf|,
they are adopted instead of the default setting.

\begin{table}[tbh]
\begin{center}
\caption{The relationship between \nmitem{CNVTOPO_name} and \nmitem{CNVTOPO_UseGTOPO30}, \nmitem{CNVTOPO_UseDEM50M}, and \nmitem{CNVTOPO_UseUSERFILE}.}
\begin{tabularx}{150mm}{X|l|l|l} \hline
  \rowcolor[gray]{0.9} \verb|CNVTOPO_name| & \verb|CNVTOPO_UseGTOPO30| & \verb|CNVTOPO_UseDEM50M| & \verb|CNVTOPO_UseUSERFILE| \\ \hline
                       \verb|NONE|           & $\ast$         & $\ast$         & $\ast$          \\ \hline
                       \verb|GTOPO30|        & \verb|.true.|  & \verb|.false.| & \verb|.false.|  \\ \hline
                       \verb|DEM50M|         & \verb|.false.| & \verb|.true.|  & \verb|.false.|  \\ \hline
                       \verb|USERFILE|       & $\ast$         & $\ast$         & \verb|.true.|   \\ \hline
\end{tabularx}
\label{tab:cvntopo_name}
\end{center}
\end{table}


\scalerm supports GTOPO30 and DEM50M provided by the Geospatial Information Authority of Japan as the topography data.
The topographic data prepared by user can be used, when \nmitem{CNVTOPO_UseUSERFILE} = \verb|.true.| or \nmitem{CNVTOPO_name} = \verb|USERFILE| (please refer to Sec. \ref{subsec:topo_userfile} for details).
The combination of these datasets is also available.
When two or more out of \nmitem{CNVTOPO_UseGTOPO30}, \nmitem{CNVTOPO_UseDEM50M}, and \nmitem{CNVTOPO_UseUSERFILE} are set to \verb|.true.|,
the program makes the data as follows:
\begin{enumerate}[1)]
 \item If \nmitem{CNVTOPO_UseGTOPO30}=\verb|.true.|,  interpolate GTOPO30 dataset to the simulation grid point.
 \item If \nmitem{CNVTOPO_UseDEM50M}=\verb|.true.|,   interpolate DEM50M dataset to the simulation grid and overwrite the data in the region covered by DEM50M.
 \item If \nmitem{CNVTOPO_UseUSERFILE}=\verb|.true.|, interpolate the user-defined data to the simulation grid and overwrite the data in the region covered by the user-defined data.
 \item Apply smoothing.
\end{enumerate}


%In default, the nearest five grid points of input data around the target grid point are used for the interpolation.
%The number of using grid point is determined by \nmitem{CNVTOPO_interp_level}.
There are two types of filter for smoothing the elevation with the steep slope in re-gridded topography: Laplacian and Gaussian filters.
The type can be chosen by \nmitem{CNVTOPO_smooth_type}. The Laplacian filter is used in default.
In the smoothing operation, the selected filter is applied multiple times until the angle is below the maximum allowable angle $\theta_{\max}$.
By specifying \nmitem{CNVTOPO_smooth_maxslope_ratio}, you can set $\mathrm{RATIO}$ described above directly.
Instead of \nmitem{CNVTOPO_smooth_maxslope_ratio}, you can also use the parameter \nmitem{CNVTOPO_smooth_maxslope}, which determines the maximum angle in degree.
The number limit of the smoothing iteration is 10000 times in default. You can set larger number by setting \nmitem{CNVTOPO_smooth_itelim}.
When \nmitem{CNVTOPO_smooth_local} is set to \verb|.true.|, the operation for smoothing is continued only at the grid point where the smoothing is not completed.

Additional hyperdiffusion is prepared to smooth the rugged terrain on a small spatial scale
We recommend applying this filtering to reduce the numerical noise induced by the rugged terrain in the simulation.
If \nmitem{CNVTOPO_smooth_hypdiff_niter} is set to negative, the filter is not applied.


\nmitem{CNVTOPO_copy_parent} is the item used for the nesting computation.
Since the spatial resolution of the child domain is higher than the parent domain in general,
the topography in the child domain is finer than that in the parent domain.
At this time, problems often occurs due to an inconsistency between the atmospheric data in the buffer region of the child domain and the boundary data (the atmospheric data of the parent domain).
To avoid this problem, the topography of the parent domain can be copied to the buffer region of the child domain by specifying \nmitem{CNVTOPO_copy_parent}$=$\verb|.true.| If there is no parent domain, \nmitem{CNVTOPO_copy_parent} must be \verb|.false.|. Section \ref{subsec:nest_topo} provides a more detailed explanation of the case that involves the use of \nmitem{CNVTOPO_copy_parent}.



\subsection{Preparation of user-defined topography} \label{subsec:topo_userfile}

When \nmitem{CNVTOPO_UseUSERFILE} is set to \verb|.true.|,
the \verb|scale-rm_pp| creats the \scale topography data from the user-defined data according to \namelist{PARAM_CNVTOPO_USERFILE}.
The input data supports the type of ``GrADS'' and ``TILE''; they are specified by \nmitem{USERFILE_TYPE}.
The details for these file types are described in Section \ref{sec:userdata}.
The avairable items for \namelist{PARAM_CNVTOPO_USERFILE} as follows:
\editboxtwo{
\verb|&PARAM_CNVTOPO_USERFILE              | & \\
\verb| USERFILE_TYPE           = '',       | & ; "GrADS" or "TILE" \\
\verb| USERFILE_DTYPE          = 'REAL4',  | & ; (for TILE) Type of the data \\
                                             & ~~~("\verb|INT2|","\verb|INT4|","\verb|REAL4|","\verb|REAL8|") \\
\verb| USERFILE_DLAT           = -1.0,     | & ; (for TILE) Interval of the grid (latitude,degree) \\
\verb| USERFILE_DLON           = -1.0,     | & ; (for TILE) Interval of the grid (longitude,degree) \\
\verb| USERFILE_CATALOGUE      = '',       | & ; (for TILE) Name of the catalogue file   \\
\verb| USERFILE_DIR            = '.',      | & ; (for TILE) Directory path of input file \\
\verb| USERFILE_yrevers        = .false.,  | & ; (for TILE) If data is stored from north to south, set \verb|.true.| \\
\verb| USERFILE_MINVAL         = 0.0,      | & ; (for TILE) The data lower than \verb|MINVAL| is recognized as missing value \\
\verb| USERFILE_GrADS_FILENAME = '',       | & ; (for GrADS) Name of the namelist for the \grads data \\
\verb| USERFILE_GrADS_VARNAME  = 'topo',   | & ; (for GrADS) Name of the target variable in the namelist \\
\verb| USERFILE_GrADS_LATNAME  = 'lat',    | & ; (for GrADS) Name of latitude in the namelist \\
\verb| USERFILE_GrADS_LONNAME  = 'lon',    | & ; (for GrADS) Name of longitude in the namelist \\
\verb| USERFILE_INTERP_TYPE    = 'LINEAR', | & ; (for GrADS) Type for the horizontal interpolation \\
\verb| USERFILE_INTERP_LEVEL   = 5,        | & ; (for GrADS) Level of the interpolation \\
\verb|/                                    | \\
}


For the ``GrADS'' type, the namelist file specified by \nmitem{USERFILE_GrADS_FILENAME} is required to set the information of the \grads binary data.
Please see Section \ref{sec:datainput_grads} for the details of the namelist file.
As the default, the variable name for the topography, latitude, and longitude are ``topo'', ``lat'', and ``lon'', respectively.
If the name is different from the default value, it can be specified with \nmitem{USERFILE_GrADS_VARNAME}, \nmitem{USERFILE_GrADS_LATNAME}, and \nmitem{USERFILE_GrADS_LONNAME}, respectively.


For the ``TILE'' type, the catalogue file specified by \nmitem{USERFILE_CATALOGUE} is required;
it contains information of the name of individual tiled data files and the area where the tiled data covers.
Refer the catalogue files of the \verb|$SCALE_DB/topo/DEM50M/Products/DEM50M_catalogue.txt| and \\
\verb|$SCALE_DB/topo/GTOPO30/Products/GTOPO30_catalogue.txt| as samples.
The sample setting of \namelist{PARAM_CNVTOPO_USERFILE} for the ``TILE'' data is as follows.
In this sample, the catalogue file named \verb|catalogue.txt| is located in the directory \verb|./input_topo|.
The data is stored with a 2-byte integer.
\editboxtwo{
\verb|&PARAM_CNVTOPO_USERFILE                     | & \\
\verb|USERFILE_CATALOGUE  = "catalogue.txt",      | & ; Name of the catalogue file \\
\verb|USERFILE_DIR        = "./input_topo",       | & ; Directory path of input file \\
\verb|USERFILE_DLAT       = 0.0083333333333333D0, | & ; Interval of the grid (latitude,  degree) \\
\verb|USERFILE_DLON       = 0.0083333333333333D0, | & ; Interval of the grid (longitude, degree) \\
\verb|USERFILE_DTYPE      = "INT2",               | & ; Type of the data ("\verb|INT2|", "\verb|INT4|", "\verb|REAL4|", "\verb|REAL8|") \\
\verb|USERFILE_yrevers    = .true.,               | & ; Data is stored from north to south in latitudinal direction? \\
\verb|/                                           | \\
}
