\section{How to create initial and boundary data} \label{sec:adv_datainput}
%====================================================================================

\begin{table}[tbh]
\begin{center}
\caption{External input data supported in \scalelib}
\begin{tabularx}{150mm}{l|l|X} \hline
 \rowcolor[gray]{0.9} Data type   & \verb|FILETYPE_ORG|  & Note \\ \hline
 SCALE format   & \verb|SCALE-RM|     & Only history files are supported. The latitude-longitude catalog is needed. \\ \hline
 Binary format  & \verb|GrADS|        & Another namelist for data input is required.    \\ \hline
 WRF format     & \verb|WRF-ARW|      & Both ``wrfout''  and``wrfrst'' are supported.\\ \hline
\end{tabularx}
\label{tab:inputdata_format}
\end{center}
\end{table}

\scalerm can generate initial and boundary data by entering various types of external data, as shown in Table \ref{tab:inputdata_format}. The program \verb|scale-rm_init| converts external data into boundary and initial data by configuring the file \verb|init.conf|. The input data format is specified at \nmitem{FILETYPE_ORG} in \namelist{PARAM_MKINIT_REAL_***}. The SCALE format is mainly used for offline nesting. Refer to Section \ref{subsec:nest_offline} for details. The WRF data format is available; WRF model output data can be used directly. Note that the file should contain all data required for the generation of the boundary data of \scalerm. The ``binary format'' in this documentation is defined as binary data with single-precision floating points that FORTRAN can directly access. GRIB/GRIB2 data are available by converting them to binary format; this procedure is explained in Section \ref{sec:tutrial_real_data}. Other arbitrary data can also be used if it is converted into binary format.

%%%---------------------------------------------------------------------------------%%%%
\subsubsection{Input from binary format data} \label{sec:datainput_grads}

The input data format is specified in \namelist{PARAM_MKINIT_REAL_***} in the configuration file \verb|init.conf|
as follows:
\editbox{
\verb|&PARAM_RESTART|\\
\verb| RESTART_OUTPUT       = .true.,|\\
\verb| RESTART_OUT_BASENAME = "init_d01",|\\
\verb|/|\\
\\
\verb|&PARAM_MKINIT_REAL_ATMOS|\\
\verb| NUMBER_OF_FILES      = 2,|\\
\verb| FILETYPE_ORG         = "GrADS",|\\
\verb| BASENAME_ORG         = "namelist.grads_boundary.FNL.grib1",|\\
\verb| BASENAME_BOUNDARY    = "boundary_d01",|\\
\verb| BOUNDARY_UPDATE_DT   = 21600.0,|\\
\verb| PARENT_MP_TYPE       = 3,|\\
\verb| USE_FILE_DENSITY     = .false.,|\\
\verb|/|\\
\verb|&PARAM_MKINIT_REAL_OCEAN|\\
\verb| NUMBER_OF_FILES      = 2,|\\
\verb| FILETYPE_ORG         = "GrADS",|\\
\verb| BASENAME_ORG         = "namelist.grads_boundary.FNL.grib1",|\\
\verb| INTRP_OCEAN_SFC_TEMP = "mask",|\\
\verb| INTRP_OCEAN_TEMP     = "mask",|\\
\verb|/|\\
\verb|&PARAM_MKINIT_REAL_LAND|\\
\verb| NUMBER_OF_FILES      = 2,|\\
\verb| FILETYPE_ORG         = "GrADS",|\\
\verb| BASENAME_ORG         = "namelist.grads_boundary.FNL.grib1",|\\
\verb| USE_FILE_LANDWATER   = .true.,|\\
\verb| INTRP_LAND_TEMP      = "mask",|\\
\verb| INTRP_LAND_WATER     = "fill",|\\
\verb| INTRP_LAND_SFC_TEMP  = "fill",|\\
\verb|/|\\
}

If binary data is entered, \verb|"GrADS"| is given to \nmitem{FILETYPE_ORG}.
In \scalerm, the namelist file \verb|namelist.grads_boundary**|, which contains the file name and the structure of binary data, is prepared instead of the ``ctl'' file. Give its path at \nmitem{BASENAME_ORG}.

\nmitem{NUMBER_OF_FILES} is the number of input files.
In case of a single input file, prepare only file \verb|"filename.grd"|.
In the case of multiple input files, prepare the files numbered as \verb|"filename.XXXXX.grd"| in the forward direction.
The program \verb|scale-rm_init| reads these files enumerated from \verb|00000| to the given number \nmitem{NUMBER_OF_FILES}-1.
The header name of input files, i.e., \verb|"filename"|, is specified in the namelist file and explained later.

\nmitem{BOUNDARY_UPDATE_DT} is the time step of input data.
\nmitem{RESTART_OUT_BASENAME} in \namelist{PARAM_RESTART} is the header name of the initial file converted.
\nmitem{BASENAME_BOUNDARY} is the header name of the boundary files converted.

The above configurations are the common among \namelist{PARAM_MKINIT_REAL_ATMOS},\\ \namelist{PARAM_MKINIT_REAL_OCEAN}, and \namelist{PARAM_MKINIT_REAL_LAND}. Unless otherwise specified in\\
\namelist{PARAM_MKINIT_REAL_OCEAN} and \namelist{PARAM_MKINIT_REAL_LAND}, these information are inherited. 


\nmitem{USE_FILE_DENSITY} is an option in case of \verb|FILETYPE_ORG="SCALE-RM"|.
If binary data is selected, provide \verb|.false.| to \nmitem{USE_FILE_DENSITY}.
\nmitem{PARENT_MP_TYPE} is the category type of the water substance in the parent model.
If binary data format is entered, give \verb| 3 | to \nmitem{PARENT_MP_TYPE}.

There are two options in preparation of soil moisture; one is a method to provide the data from the parent model and the other a method to provide it as a constant value in the entire region.
In the former case, 3D soil moisture data are required. In the latter, configure \verb|USE_FILE_LANDWATER = .false.| in \namelist{PARAM_MKINIT_REAL_LAND} in \verb|init.conf|.
The soil water condition is specified in \verb|INIT_LANDWATER_RATIO| as the ratio of occupation of water to the void in the soil per unit volume (degree of saturation). The default value is 0.5. The size of void in the soil per unit volume (void ratio) depends on land use.
\editboxtwo{
\verb|&PARAM_MKINIT_REAL_LAND| &\\
\verb| USE_FILE_LANDWATER   = .false.| & whether or not soil moisture is given by file. The default is \verb|.true.| \\
\verb| INIT_LANDWATER_RATIO = 0.5    | & in the case of \verb|USE_FILE_LANDWATER=.false.| \\
                                       & the ratio of occupation of water to void (degree of saturation)\\
\verb|  ..........                 | & \\
\verb|/| & \\
}

If binary data ( \grads format ) is used as input file, prepare them yourself. Refer to \grads Web page (\url{http://cola.gmu.edu/grads/gadoc/aboutgriddeddata.html#structure}) for the format.\\
The following example is the namelist file \verb|namelist.grads_boundary**| to provide information pertaining to the data file name and data structure in \scalerm in stead of ``ctl'' file.

\editbox{
\verb|#| \\
\verb|# Dimension    |  \\
\verb|#|                \\
\verb|&nml_grads_grid|  \\
\verb| outer_nx     = 360,|~~~   ; the number of grids of the atmosphere along the x direction \\
\verb| outer_ny     = 181,|~~~   ; the number of grids of the atmosphere along the y direction \\
\verb| outer_nz     = 26, |~~~~~ ; the number of layers for the atmosphere\\
\verb| outer_nl     = 4,  |~~~~~~ ; the number of layers for soil data\\
\verb|/|                \\
\\
\verb|#              |  \\
\verb|# Variables    |  \\
\verb|#              |  \\
\verb|&grdvar  item='lon',     dtype='linear',  swpoint=0.0d0,   dd=1.0d0 /  |  \\
\verb|&grdvar  item='lat',     dtype='linear',  swpoint=90.0d0,  dd=-1.0d0 / |  \\
\verb|&grdvar  item='plev',    dtype='levels',  lnum=26,| \\
~~~\verb|      lvars=100000,97500,.........,2000,1000, /     |  \\
\verb|&grdvar  item='MSLP',    dtype='map',     fname='FNLsfc', startrec=1,  totalrec=6   / |  \\
\verb|&grdvar  item='PSFC',    dtype='map',     fname='FNLsfc', startrec=2,  totalrec=6   / |  \\
\verb|&grdvar  item='U10',     dtype='map',     fname='FNLsfc', startrec=3,  totalrec=6   / |  \\
\verb|&grdvar  item='V10',     dtype='map',     fname='FNLsfc', startrec=4,  totalrec=6   / |  \\
\verb|&grdvar  item='T2',      dtype='map',     fname='FNLsfc', startrec=5,  totalrec=6   / |  \\
\verb|&grdvar  item='RH2',     dtype='map',     fname='FNLsfc', startrec=6,  totalrec=6   / |  \\
\verb|&grdvar  item='HGT',     dtype='map',     fname='FNLatm', startrec=1,  totalrec=125 / |  \\
\verb|&grdvar  item='U',       dtype='map',     fname='FNLatm', startrec=27, totalrec=125 / |  \\
\verb|&grdvar  item='V',       dtype='map',     fname='FNLatm', startrec=53, totalrec=125 / |  \\
\verb|&grdvar  item='T',       dtype='map',     fname='FNLatm', startrec=79, totalrec=125 / |  \\
\verb|&grdvar  item='RH',      dtype='map',     fname='FNLatm', startrec=105,totalrec=125, knum=21 /  |  \\
\verb|&grdvar  item='llev',    dtype='levels',  lnum=4, lvars=0.05,0.25,0.70,1.50, /        |  \\
\verb|&grdvar  item='lsmask',  dtype='map',     fname='FNLland', startrec=1, totalrec=10 /  |  \\
\verb|&grdvar  item='SKINT',   dtype='map',     fname='FNLland', startrec=2, totalrec=10 /  |  \\
\verb|&grdvar  item='STEMP',   dtype='map',     fname='FNLland', startrec=3, totalrec=10,|\\
~~~~~~~~\verb| missval=9.999e+20 /|  \\
\verb|&grdvar  item='SMOISVC', dtype='map',     fname='FNLland', startrec=7, totalrec=10,|\\
~~~~~~~~\verb| missval=9.999e+20 /|  \\
}

The number of grids in the atmosphere is specified as \verb|outer_nx, outer_ny, outer_nz|, and the number of layers of soil data (\verb|STEMP、SMOISVC|) is specified as \verb|outer_nl|.\\ 

The input data to \verb|QV| and \verb|RH| is not often provided in the upper layers.
In such cases, the number of layers where the data exist is specified as \verb|knum|. Two methods of giving values to the upper layers are prepared. As default, \verb| upper_qv_type = "ZERO"| as
\editboxtwo{
\verb|&PARAM_MKINIT_REAL_GrADS| & \\
\verb| upper_qv_type = "ZERO"| & \verb|"ZERO"|: QV=0 \\
                               & \verb|"COPY"|: copy the RH at the top layer where input humidity data exists to the upper layers without the data\\
\verb|/|\\
}

The configuration of \namelist{grdvar} is different by data, as shown in Table \ref{tab:namelist_grdvar}.
The list of \namelist{grdvar} is shown in Table \ref{tab:grdvar_item}. In Table \ref{tab:grdvar_item}, soil moisture (fraction of volume) is the ratio of water volume ($V_w$) to soil volume ($V$), i.e., $V_w / V$. The saturation ratio is the ratio of water volume $V_w$ to void volume in $V$, i.e., $V_w / V_v$. If \nmitem{USE_FILE_LANDWATER}\verb|=.true.| in \namelist{PARAM_MKINIT_REAL_LAND}, prepare data either for \verb|SMOISVC| or of \verb|SMOISDS|. 


{\small
\begin{table}[tbh]
\begin{center}
\caption{Variables of \namelist{grdvar}}
\label{tab:namelist_grdvar}
\begin{tabularx}{150mm}{llX} \hline
\rowcolor[gray]{0.9} 
item of \verb|grdvar|      & Explanation    & Note \\ \hline
item                        & Variable name  & Select from Table \ref{tab:grdvar_item}   \\
dtype                       & Data type      & \verb|"linear" or "levels" or "map"| \\\hline
\multicolumn{3}{X}{namelist at \nmitem{dtype}\verb|="linear"| (Specific use of \verb|"lon", "lat"| )} \\ \hline
swpoint                     & Value of start point &  \\
dd                          & Increment            &  \\ \hline
\multicolumn{3}{X}{namelist at \nmitem{dtype}\verb|"=levels"| (Specific use of \verb|"plev", "llev"|)} \\ \hline
lnum      & Number of levels (layers )     &  \\
lvars     & Values of each layer           &  \\ \hline
\multicolumn{3}{X}{namelist at \nmitem{dtype}\verb|="map"|}           \\ \hline
fname     & Header name of files           &  \\
startrec  & Recorded number of variables \nmitem{item}     &  time at t=1\\
totalrec  & Recorded length of all variables per time  &  \\
knum      & Number of layers of 3D data & (option) in the case of specifying \\
                             &                      &  value that differs from \verb|outer_nz|\\
                             &                      &  available for RH and QV\\
missval  & missing value        & (option) \\ \hline
\end{tabularx}
\end{center}
\end{table}
}

{
\begin{table}[bth]
\begin{center}
\caption{Variable list of \nmitem{item} in \namelist{grdvar}. $\ast$ means ``required.''}
\label{tab:grdvar_item}
\small
\begin{tabularx}{150mm}{rl|l|l|X} \hline
 \rowcolor[gray]{0.9}  & Variable name            & Explanation &  Unit & \nmitem{dtype} \\ \hline
$\ast$ &\verb|lon|     & longitude data           & [deg.]      & \verb|linear, map| \\
$\ast$ &\verb|lat|     & latitude data            & [deg.]      & \verb|linear, map| \\
$\ast$ &\verb|plev|    & pressure data            & [Pa]        & \verb|levels, map| \\
$\ast$ &\verb|HGT|     & geopotential height data & [m]         & \verb|map| \\
$\ast$ &\verb|U|       & eastward wind speed      & [m/s]       & \verb|map| \\
$\ast$ &\verb|V|       & northward wind speed     & [m/s]       & \verb|map| \\
$\ast$ &\verb|T|       & temperature              & [K]         & \verb|map| \\
$\ast$ &\verb|QV|      & specific humidity (optional if RH is given) & [kg/kg] & \verb|map| \\
$\ast$ &\verb|RH|      & relative humidity (optional if QV is given) & [\%] & \verb|map| \\
       &\verb|MSLP|    & sea level pressure       & [Pa]        & \verb|map| \\
       &\verb|PSFC|    & surface pressure         & [Pa]        & \verb|map| \\
       &\verb|U10|     & eastward 10m wind speed  & [m/s]       & \verb|map| \\
       &\verb|V10|     & northward 10m wind speed & [m/s]       & \verb|map| \\
       &\verb|T2|      & 2m temperature           & [K]         & \verb|map| \\
       &\verb|Q2|      & 2m specific humidity  (optional if RH2 is given)   &[kg/kg] & \verb|map| \\
       &\verb|RH2|     & 2m relative humidity (optional if Q2 is given) & [\%]  & \verb|map| \\
       &\verb|TOPO|    & topography of GCM        & [m]         & \verb|map| \\
       &\verb|lsmask|  & ocean--land distribution of GCM & 0:ocean1:land & \verb|map| \\
$\ast$ &\verb|SKINT|   & surface temperature      & [K]         & \verb|map| \\
$\ast$ &\verb|llev|    & soil depth  & [m]        & \verb|levels| \\
$\ast$ &\verb|STEMP|   & soil temperature         & [K]         & \verb|map| \\
($\ast$) &\verb|SMOISVC| & soil moisture (volume fraction)  & [-] & \verb|map| \\
($\ast$) &\verb|SMOISDS| & soil moisture (saturation ratio) & [-] & \verb|map| \\
$\ast$ &\verb|SST|     & sea surface temperature (optional if SKINT is given) & [K] & \verb|map|\\\hline
\end{tabularx}
\end{center}
\end{table}
}
