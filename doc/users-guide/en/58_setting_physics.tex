\section{Setting the physical process} \label{sec:basic_usel_physics}
%------------------------------------------------------

\subsection{Cloud micro-physics} \label{subsec:basic_usel_microphys}
%------------------------------------------------------
The cloud micro-physics scheme is configured in \nmitem{ATMOS_PHY_MP_TYPE} in \namelist{PARAM_ATMOS} in files \verb|init.conf| and \verb|run.conf|, respectively. 
{\color{blue} Note that it is necessary to specify the same scheme for \nmitem{ATMOS_PHY_MP_TYPE} in both \texttt{init.conf} and \texttt{run.conf}}. 
The update interval for the cloud micro-physics scheme is specified in \namelist{PARAM_TIME}. Refer to Section \ref{sec:timeintiv} for the detailed configuration of calling timing. The following example shows the configuration for cases involving a six-class one-moment bulk scheme that contains ice phase clouds:

\editboxtwo{
\verb|&PARAM_ATMOS  | & \\
\verb| ATMOS_PHY_MP_TYPE = "TOMITA08", | & ; Choose from Table \ref{tab:nml_atm_mp}.\\
\verb|/             | & \\
}

\begin{table}[tbh]
\begin{center}
  \caption{Choices of cloud micro-physics scheme}
  \label{tab:nml_atm_mp}
  \begin{tabularx}{150mm}{lXX} \hline
    \rowcolor[gray]{0.9}  Scheme name & Description of scheme & Reference\\ \hline
     \verb|OFF|      & Do not calculate phase change of water by cloud micro-physics. &  \\
     \verb|KESSLER|  & Three-class one-moment bulk scheme & \citet{kessler_1969} \\
     \verb|TOMITA08| & Six-class one-moment bulk scheme & \citet{tomita_2008} \\
     \verb|SN14|     & Six-class two-moment bulk scheme & \citet{sn_2014} \\
     \verb|SUZUKI10| & Spectral bin scheme (consideration of ice cloud can be specified as option) & \citet{suzuki_etal_2010} \\
%    \verb|XX|       & Super droplet scheme              & \citer{Shima_etal_2009} \\
    \hline
  \end{tabularx}
\end{center}
\end{table}

Four typical schemes are prepared:
\begin{enumerate}
\item {\bf One-moment bulk scheme without ice \cite{kessler_1969}}\\ This scheme assumes that the particle size distribution function is expressed only by mass concentration. Considering two categories of water in cloud and rain, the ratios of the densities of cloud and rain to total air density are prognostically predicted.
\item {\bf One-moment bulk scheme with ice \cite{tomita_2008}}\\
This scheme makes the same assumption as that in \cite{kessler_1969} for the particle size distribution function, but with five categories of water: cloud, rain, ice, snow, and graupel.
\item {\bf Two-moment bulk scheme with ice \cite{sn_2014}}\\
In this scheme, the particle size distribution is expressed by the numerical concentration of particles and their mass concentration. 
\item {\bf One-moment bin scheme \cite{suzuki_etal_2010}}\\
This scheme explicitly expresses particle size distribution by discretizing it using an appropriate number of degrees of freedom for each category. There are six categories: cloud, rain, ice, snow, graupel, and hail. The accuracy of expressing the size distribution depends on the degrees of freedom.

\end{enumerate}
The degrees of sophistication increases from 1 to 4, as does computational cost.

If \verb|SUZUKI10| is selected, in addition to the specification of \nmitem{ATMOS_PHY_MP_TYPE}, the following configuration needs to be added to both files of \verb|init.conf| and \verb|run.conf|:
\editboxtwo{
\verb|&PARAM_BIN|   &  \\
\verb| nbin   = 33, & : The number of bins| \\
\verb| ICEFLG =  1, & : Option for consideration of ice cloud: 0(not considered), 1(considered)| \\
\verb|/|            & \\
}
In this case, {\color{blue}{\namelist{PARAM_BIN} in \texttt{init.conf} must also be same as in \texttt{run.conf}}}. A necessary file \verb|micpara.dat| is automatically generated. If file \verb|micpara.dat| already exists, it is used for the calculation. When changing \verb|nbin| as described in the first line, this file is regenerated. If \verb|nbin| in file \verb|run.conf| is different from that in file \verb|micpara.dat|, the following error message is output and the simulation program is terminated instantaneously without calculation:
\msgbox{
\verb|xxx nbin in inc_tracer and nbin in micpara.dat is different check!| \\
}
To avoid this error, it is necessary to delete the old \verb|micpara.dat| beforehand and regenerate it. The regeneration is automatically done at the execution of \scalerm with \verb|SUZUKI10|.


\subsection{Turbulence scheme} \label{subsec:basic_usel_turbulence}
%------------------------------------------------------

The turbulence scheme is specified in \nmitem{ATMOS_PHY_TB_TYPE} in \namelist{PARAM_ATMOS} in files \verb|init.conf| and \verb|run.conf|. The timing of the calling of the turbulence scheme is specified in \namelist{PARAM_TIME}. Refer to Section \ref{sec:timeintiv} for the detailed configuration of the calling timing.

\editboxtwo{
\verb|&PARAM_ATMOS  | & \\
\verb| ATMOS_PHY_TB_TYPE = "MYNN", | & ; Select the scheme shown in Table \ref{tab:nml_atm_tb}\\
\verb|/             | & \\
}
\begin{table}[h]
\begin{center}
  \caption{Choices of turbulent scheme}
  \label{tab:nml_atm_tb}
  \begin{tabularx}{150mm}{lXX} \hline
    \rowcolor[gray]{0.9}  Value & Description of scheme & Reference\\ \hline
      \verb|OFF|          & Do not calculate the turbulence process &  \\
      \verb|SMAGORINSKY|  & Smagorinsky—Lily-type sub-grid model & \citet{smagorinsky_1963,lilly_1962,Brown_etal_1994,Scotti_1993} \\
      \verb|D1980|        &  Deardorff-type sub-grid model & \citet{Deardorff_1980} \\
      \verb|MYNN|         &  MYNN Level 2.5 boundary scheme & \citet{my_1982,nakanishi_2004} \\
    \hline
  \end{tabularx}
\end{center}
\end{table}

\subsection{Radiation scheme} \label{subsec:basic_usel_radiation}
%-------------------------------------------------------------------------------
The radiation scheme is specified in \nmitem{ATMOS_PHY_RD_TYPE} in \namelist{PARAM_ATMOS} in files \verb|init.conf| and \verb|run.conf|. The timing of the calling of the radiation scheme is specified in \namelist{PARAM_TIME}.  Refer to Section \ref{sec:timeintiv} for the detailed configuration of calling timing.

\editboxtwo{
\verb|&PARAM_ATMOS  | & \\
\verb| ATMOS_PHY_RD_TYPE = "MSTRNX", | & ; Select the radiation scheme shown in Table \ref{tab:nml_atm_rd}\\
\verb|/             | & \\
}
\begin{table}[h]
\begin{center}
  \caption{Choices of radiation scheme}
  \label{tab:nml_atm_rd}
  \begin{tabularx}{150mm}{lXX} \hline
    \rowcolor[gray]{0.9}  Value & Explanation of scheme & Reference\\ \hline
      \verb|OFF| or \verb|NONE| & Do not calculate the radiation scheme & \\
      \verb|OFFLINE|      & use prescrived radiative data given from a file & \\
      \verb|MSTRNX|       & mstrnX (A k-distribution-based broadband radiation transfer model) & \citet{sekiguchi_2008} \\
%      \verb|WRF|          & mstrnX(Long wave) + Dudhia (shortwave) & \citet{dudhia_1989} \\
    \hline
  \end{tabularx}
\end{center}
\end{table}

\subsubsection{Configuration for \texttt{OFFLINE}}

When \nmitem{ATMOS_PHY_RD_TYPE} is \verb|OFFLINE| in \namelist{PARAM_ATMOS},
the file name and information of the data are specified in \namelist{PARAM_ATMOS_PHY_RD_OFFLINE}.

\editboxtwo{
\verb|&PARAM_ATMOS_PHY_RD_OFFLINE        | & \\
\verb| ATMOS_PHY_RD_OFFLINE_BASENAME              = "",          | & ; base name of external data file \\
\verb| ATMOS_PHY_RD_OFFLINE_AXISTYPE              = "XYZ",       | & ; order of spatial dimensions of the data. 'XYZ' or 'ZXY' \\
\verb| ATMOS_PHY_RD_OFFLINE_ENABLE_PERIODIC_YEAR  = .false.,     | & ; whether annually cyclic data \\
\verb| ATMOS_PHY_RD_OFFLINE_ENABLE_PERIODIC_MONTH = .false.,     | & ; whether monthly cyclic data \\
\verb| ATMOS_PHY_RD_OFFLINE_ENABLE_PERIODIC_DAY   = .false.,     | & ; whether dayly cyclic data \\
\verb| ATMOS_PHY_RD_OFFLINE_STEP_FIXED            = 0,           | & ; step number when data at a cirtain time step is used. Set the value less than 1 for temporal varied data. \\
\verb| ATMOS_PHY_RD_OFFLINE_CHECK_COORDINATES     = .true.,      | & ; whether coordinate variables are to be checked \\
\verb| ATMOS_PHY_RD_OFFLINE_STEP_LIMIT            = 0,           | & ; maximum limit of steps. The data at the time step exceed this limit would not be read. 0 for no limit \\
\verb| ATMOS_PHY_RD_OFFLINE_DIFFUSE_RATE          = 0.5D0,       | & ; diffuse rate (diffuse solar radiation/global solar radiation) of short wave used when short-wave direct flux data is not given \\
\verb|/|            & \\
}

\noindent
The file format of external data file is \netcdf format 
with the same corrdinate variables as that of initial/boundary data files.
Variables as the external data is shown in Table \ref{tab:var_list_atm_rd_offline}.
\begin{table}[h]
\begin{center}
  \caption{Radiative data as external file input}
  \label{tab:var_list_atm_rd_offline}
  \begin{tabularx}{150mm}{lXll} \hline
    \rowcolor[gray]{0.9}  Variable name & Description & \# of dimensions & \\ \hline
      \verb|RFLX_LW_up|     & upward long-wave radiative flux & 3D (spatial) + 1D (time) \\
      \verb|RFLX_LW_dn|     & downward long-wave radiative flux & 3D (spatial) + 1D (time) \\
      \verb|RFLX_SW_up|     & upward short-wave radiative flux & 3D (spatial) + 1D (time) \\
      \verb|RFLX_SW_dn|     & downward short-wave radiative flux & 3D (spatial) + 1D (time) \\
      \verb|SFLX_LW_up|     & upward long-wave radiative flux at the surface & 2D (spatial) + 1D (time) \\
      \verb|SFLX_LW_dn|     & downward long-wave radiative flux at the surface & 2D (spatial) + 1D (time) \\
      \verb|SFLX_SW_up|     & upward short-wave radiative flux at the surface & 2D (spatial) + 1D (time) \\
      \verb|SFLX_SW_dn|     & downward short-wave radiative flux at the surface & 2D (spatial) + 1D (time) \\
      \verb|SFLX_SW_dn_dir| & downward short-wave direct radiative flux at the surface & 2D (spatial) + 1D (time) & optional \\
    \hline
  \end{tabularx}
\end{center}
\end{table}



\subsubsection{Configuration for \texttt{MSTRNX}}


The solar radiation is calculated by using date and time, and latitude and longitude, in the model domain. For the ideal experiment, they can be arbitrarily given as fixed values of time and location over the domain. The solar constant can also be changed. These are configured in \namelist{PARAM_ATMOS_SOLARINS} as follows:

\editboxtwo{
\verb|&PARAM_ATMOS_SOLARINS        | & \\
\verb| ATMOS_SOLARINS_constant    = 1360.250117   | & Solar constant [W/m2] \\
\verb| ATMOS_SOLARINS_fixedlatlon = .false.       | & Whether values are fixed for latitude and longitude at radiation calculation\\
\verb| ATMOS_SOLARINS_fixeddate   = .false.       | & Whether values are fixed for date and time at radiation calculation\\
\verb| ATMOS_SOLARINS_lon         = 135.221       | & Longitude in case that \verb|ATMOS_SOLARINS_fixedlatlon=.true.| [deg.] \\
\verb| ATMOS_SOLARINS_lat         =  34.653       | & Latitude in case that \verb|ATMOS_SOLARINS_fixedlatlon=.true.| [deg.]\\
\verb| ATMOS_SOLARINS_date = -1,-1,-1,-1,-1,-1,   | & Year, month, day, and time in the case that \verb|ATMOS_SOLARINS_fixeddate=.true.|[Y,M,D,H,M,S]\\
\verb|/|            & \\
}
If \nmitem{MPRJ_basepoint_lon,MPRJ_basepoint_lat} in \namelist{PARAM_MAPPROJ} are given, \\
\nmitem{ATMOS_SOLARINS_lon, ATMOS_SOLARINS_lat} are set identical to the values of 
\nmitem{MPRJ_basepoint_lon, MPRJ_basepoint_lat}.
Refer to Section \ref{subsec:adv_mapproj} for the explanation of \namelist{PARAM_MAPPROJ}.


Depending on experimental design, the top of the model is often too low, such as 10$\sim$ 20km, compared to the height of atmosphere. To remedy this situation, another top height used only for radiation calculation is set. The number of layers to be placed at levels higher than the top of the model can be configured. The top for radiation depends on the parameter file of the radiation scheme. For example, when \verb|MSTRNX| is used, the default parameter table in \verb|MSTRNX| assumes that it is 100 km. The additional layers added are 10 by default; if the top of the model is 22 km, 10 additional layers with a grid spacing of 7.8 km are added for radiation calculation. These are configured in \namelist{PARAM_ATMOS_PHY_RD_MSTRN} as follows:

\editboxtwo{
\verb|&PARAM_ATMOS_PHY_RD_MSTRN | & \\
\verb| ATMOS_PHY_RD_MSTRN_KADD                  = 10             | & Number of layers between model top and TOA for radiation\\
\verb| ATMOS_PHY_RD_MSTRN_TOA                   = 100.0          | & Height of TOA for radiation [km]( depending on parameter file used)\\
\verb| ATMOS_PHY_RD_MSTRN_nband                 = 29             | & Number of bins for wavelength (depending on parameter file used)\\
\verb| ATMOS_PHY_RD_MSTRN_nptype                = 11             | & Number of aerosol species (depending on parameter file used)\\
\verb| ATMOS_PHY_RD_MSTRN_nradius               = 6              | & Number of particle bins for aerosol (depending on parameter file used)\\
\verb| ATMOS_PHY_RD_MSTRN_GASPARA_IN_FILENAME   = "PARAG.29"     | & Input file for absorption parameter by gas\\
\verb| ATMOS_PHY_RD_MSTRN_AEROPARA_IN_FILENAME  = "PARAPC.29"    | & Input file for absorption and scattering parameter by aerosol\\
\verb| ATMOS_PHY_RD_MSTRN_HYGROPARA_IN_FILENAME = "VARDATA.RM29" | & Input file for aerosol particle parameter\\
\verb| ATMOS_PHY_RD_MSTRN_ONLY_QCI              = .false.        | & Whether rain, snow, and graupel are considered\\
\verb|/| & \\
}

\verb|MSTRNX| requires a parameter table for radiation calculation. By default, the wavelength from solar radiation to infrared radiation is divided into 29 bands/111 channels; 11 types of cloud and aerosol particles with six particle bins are prepared in the table. Three kinds of parameter files are prepared in the directory \verb|scale-rm/test/data/rad/|. The file and directory are specified in \namelist{PARAM_ATMOS_PHY_RD_MSTRN}.

It is necessary to provide vertical profiles in additional layers for temperature, pressure, and gas concentration, such as CO2 and ozone. There are two methods for this. The profiles are input as climatologies or prepared by users in ASCII format. In the case of providing climatologies, \scalerm provides the database for CIRA86\footnote{http://catalogue.ceda.ac.uk/uuid/4996e5b2f53ce0b1f2072adadaeda262}\citep{CSR_2006} for temperature and pressure, and MIPAS2001\citep{Remedios_2007} for gas species. The climatology profiles are calculated according to date and time, and latitude and longitude are calculated in the model domain. These input files are also provided in the directory \verb|scale-rm/test/data/rad/|. The file and directory names are specified in \namelist{PARAM_ATMOS_PHY_RD_PROFILE}.

In case of using user-defined profiles, users must prepare height [m], pressure [Pa], temperature [K], water vapor [kg/kg], and ozone concentration [kg/kg] in ASCII format. The concentrations of the other gases are set to zero and temporal variation is not considered. An example of user-defined files is given for the directory \verb|scale-rm/test/data/rad/|. Given as \nmitem{ATMOS_PHY_RD_PROFILE_use_climatology} $=$ \verb|.false.| in \namelist{PARAM_ATMOS_PHY_RD_PROFILE}, the file and directory names are required to be specified. 

\editboxtwo{
\verb|&PARAM_ATMOS_PHY_RD_PROFILE | & \\
\verb| ATMOS_PHY_RD_PROFILE_use_climatology       = .true.    | & Whether climatologies of CIRA86 and MIPAS2001 are used \\
\verb| ATMOS_PHY_RD_PROFILE_CIRA86_IN_FILENAME    = "cira.nc" | & File name of \verb|CIRA86 (NetCDF format)|\\
\verb| ATMOS_PHY_RD_PROFILE_MIPAS2001_IN_BASENAME = "."       | & Directory name of \verb|MIPAS2001 (ASCII format)|\\
\verb| ATMOS_PHY_RD_PROFILE_USER_IN_FILENAME      = ""        | & User-defined file in the case of not using climatorogies (ASCII format)\\
\verb| ATMOS_PHY_RD_PROFILE_USE_CO2               = .true.    | & When false, CO2 concentration is zero.\\
\verb| ATMOS_PHY_RD_PROFILE_USE_O3                = .true.    | & When false, O3 concentration is zero.\\
\verb| ATMOS_PHY_RD_PROFILE_USE_N2O               = .true.    | & When false, N2O concentration is zero.\\
\verb| ATMOS_PHY_RD_PROFILE_USE_CO                = .true.    | & When false, CO concentration is zero.\\
\verb| ATMOS_PHY_RD_PROFILE_USE_CH4               = .true.    | & When false, CH4 concentration is zero.\\
\verb| ATMOS_PHY_RD_PROFILE_USE_O2                = .true.    | & When false, O2 concentration is zero.\\
\verb| ATMOS_PHY_RD_PROFILE_USE_CFC               = .true.    | & When false, CFC concentration is zero.\\
\verb|/| & \\
}
The number of layers and their heights in this user-defined file can be given independently of the default model configuration. During execution, the values in the model layers are interpolated from the given profiles. Note that if the top height considered TOA in the radiation calculation is higher than in the input profile, an extrapolation is adopted there.




\subsection{Surface flux at the bottom of atmospheric boundary } \label{subsec:basic_usel_surface}
%------------------------------------------------------
The flux scheme at the bottom atmospheric boundary is configured in \nmitem{ATMOS_PHY_SF_TYPE} in \namelist{PARAM_ATMOS} as follows: 

\editboxtwo{
\verb|&PARAM_ATMOS  | & \\
\verb| ATMOS_PHY_SF_TYPE = "COUPLE", | &  ; Select the surface flux scheme shown in Table \ref{tab:nml_atm_sf}\\
\verb|/             | & \\
}
If ocean, land, and urban models are not used, the bottom atmospheric boundary is assumed to be a virtual surface used in an ideal experiment. The timing of the calling of the surface flux scheme is configured in \namelist{PARAM_TIME}. Refer to Section \ref{sec:timeintiv} for the detailed configuration of the calling timing. If ocean, land, and urban models are used, ``COUPLE'' is given to \nmitem{ATMOS_PHY_SF_TYPE}:

\begin{table}[htb]
\begin{center}
  \caption{Choices for the atmospheric bottom boundary }
  \label{tab:nml_atm_sf}
  \begin{tabularx}{150mm}{lX} \hline
    \rowcolor[gray]{0.9}  Value & Description of scheme\\ \hline
      \verb|NONE|         & Do not calculate surface flux, but \verb|"NONE"| is replaced to \verb|"COUPLE"| according to settings of ocean, land, and urban models) \\
      \verb|OFF|          & Do not calculate surface flux\\
      \verb|CONST|      & Fix the constant value of surface flux \\
      \verb|BULK|       & Calculate the surface flux in bulk mode \\
      \verb|COUPLE|     & Receive surface flux from ocean, land, and urban models \\
    \hline
  \end{tabularx}
\end{center}
\end{table}

%-------------------------------------------------------------------------------
\subsubsection{Configuration of constant}

If \nmitem{ATMOS_PHY_SF_TYPE} $=$ \verb|"CONST"|, the surface flux can be kept to a value specified in file \verb|run.conf| as follows. The values below are the default ones.

\editboxtwo{
 \verb|&PARAM_ATMOS_PHY_SF_CONST                | & \\
 \verb| ATMOS_PHY_SF_FLG_MOM_FLUX   =    0      | & 0: Bulk coefficient is constant \\
                                                  & 1: Frictional velocity is constant  \\
 \verb| ATMOS_PHY_SF_U_minM         =    0.0E0  | & Lower limit of absolute velocity  [m/s] \\
 \verb| ATMOS_PHY_SF_Const_Cm       = 0.0011E0  | & Constant bulk coefficient for momentum \\
                                                  &  (Active at \verb|ATMOS_PHY_SF_FLG_MOM_FLUX = 0|) \\
 \verb| ATMOS_PHY_SF_CM_min         =    1.0E-5 | & Lower limit of bulk coefficient for momentum \\
                                                  &  (Active at \verb|ATMOS_PHY_SF_FLG_MOM_FLUX = 1|) \\
 \verb| ATMOS_PHY_SF_Const_Ustar    =   0.25E0  | & Constant fictional velocity [m/s] \\
                                                  &  (Active at \verb|ATMOS_PHY_SF_FLG_MOM_FLUX = 1|) \\
 \verb| ATMOS_PHY_SF_Const_SH       =    15.E0  | & Constant sensible heat flux at the surface [W/m2] \\
 \verb| ATMOS_PHY_SF_FLG_SH_DIURNAL =   .false. | & Whether diurnal variation is enabled for sensible heat flux [logical]\\
 \verb| ATMOS_PHY_SF_Const_FREQ     =    24.E0  | & Daily cycle if diurnal variation is enabled [hour]\\
 \verb| ATMOS_PHY_SF_Const_LH       =   115.E0  | & Constant latent heat flux at the surface [W/m2] \\
 \verb|/|            & \\
}

\subsubsection{Bulk configuration}
%-------------------------------------------------------------------------------
If \nmitem{ATMOS_PHY_SF_TYPE} $=$ \verb|"BULK"|, the surface flux is calculated by the bulk model using the surface temperature. The roughness is calculated by the scheme for ocean roughness as default. On the contrary, ocean roughness can be given as a constant value, and evaporation efficiency can be given arbitrarily. This flexibility enables the ideal experiment not only for ocean surfaces but also for land. The scheme for ocean roughness is specified in \nmitem{ROUGHNESS_TYPE} in \namelist{PARAM_ROUGHNESS} in the file \verb|run.conf| as follows:

\editboxtwo{
\verb|&PARAM_ROUGHNESS  | & \\
\verb| ROUGHNESS_TYPE = "MOON07", | & ; Select the radiation scheme shown in Table \ref{tab:nml_roughness}\\
\verb|/             | & \\
}
\begin{table}[hbt]
\begin{center}
  \caption{Choices of ocean roughness}
  \label{tab:nml_roughness}
  \begin{tabularx}{150mm}{lXX} \hline
    \rowcolor[gray]{0.9}  Value & Description of scheme & Reference \\ \hline
      \verb|MOON07|   & Based on empirical formula with time development as default & \citet{moon_2007} \\
      \verb|MILLER92| & Based on empirical formula without time development          & \citet{miller_1992} \\
      \verb|CONST|    & Constant value & \\
    \hline
  \end{tabularx}
\end{center}
\end{table}

The scheme for the bulk coefficient is configured in \nmitem{BULKFLUX_TYPE} in \namelist{PARAM_BULKFLUX} in the file \verb|run.conf| as follows:

\editboxtwo{
\verb|&PARAM_BULKFLUX  | & \\
\verb| BULKFLUX_TYPE = "B91W01", | & ; Select the radiation scheme shown in Table \ref{tab:nml_bulk}\\
\verb|/                | & \\
}
\begin{table}[h]
\begin{center}
  \caption{Choices of bulk coefficient scheme}
  \label{tab:nml_bulk}
  \begin{tabularx}{150mm}{llX} \hline
    \rowcolor[gray]{0.9}  Value & Description of scheme & Reference\\ \hline
      \verb|B91W01| & Bulk method by the universal function (Default) & \citet{beljaars_1991,wilson_2001} \\
      \verb|U95|    & Louis-type bulk method  (improved version of Louis (1979) & \citet{uno_1995} \\
    \hline
  \end{tabularx}
\end{center}
\end{table}

\subsection{Ocean model} \label{subsecp:basic_usel_ocean}
%-------------------------------------------------------------------------------
The ocean process consists of two main parts, i.e., update of sea surface state and flux calculation at the interface of atmosphere and ocean. The timing of calling these scheme is configured in \namelist{PARAM_TIME}. Refer to Section \ref{sec:timeintiv} for the detailed configuration of the calling timing.

\subsubsection{Sea surface scheme}
%-------------------------------------------------------------------------------
The manner of updating sea surface state, such as sea surface temperature, is configured in \nmitem{OCEAN_TYPE} in \namelist{PARAM_OCEAN} in the files \verb|init.conf| and \verb|run.conf|:

\editboxtwo{
\verb|&PARAM_OCEAN           | & \\
\verb| OCEAN_TYPE = "CONST", | & ; Select the sea surface type shown in Table \ref{tab:nml_ocean}\\
\verb|/                      | & \\
}
\begin{table}[h]
\begin{center}
  \caption{Choices of sea surface schemes}
  \label{tab:nml_ocean}
  \begin{tabularx}{150mm}{lX} \hline
    \rowcolor[gray]{0.9}  Values & Description of scheme \\ \hline
      \verb|NONE or OFF| & Not using the sea surface model         \\
      \verb|CONST|        & Fixed as initial condition           \\
      \verb|FILE|         & Input from external data (temporal variation is available) \\
      \verb|SLAB|         & Slab ocean model                   \\
    \hline
  \end{tabularx}
\end{center}
\end{table}

If the ocean category is included in land-use setup by \namelist{PARAM_LANDUSE}, neither \verb|"NONE"| nor \verb|"OFF"| can be given to \nmitem{OCEAN_TYPE}. If this condition is not satisfied, the program immediately stops without the computation, and outputs the following message to LOG file:
\msgbox{
\verb|xxx Ocean fraction exists, but ocean components never called. STOP.| \\
}

If \nmitem{OCEAN_TYPE} $=$ \verb|"FILE"|, it is necessary to specify an external input file name in files \verb|init.conf| and \verb|run.conf|. In this case, the sea surface temperature varies with time according to spatial distribution and the temporal history of the external file.
\editboxtwo{
 \verb|&PARAM_OCEAN_PHY_FILE                                    | & \\
 \verb| OCEAN_PHY_FILE_basename   = "",                         | & ; base name of input data \\
 \verb| OCEAN_PHY_FILE_ENABLE_PERIODIC_YEAR  = .false.,         | & ; whether annually cyclic data \\
 \verb| OCEAN_PHY_FILE_ENABLE_PERIODIC_MONTH = .false.,         | & ; whether monthly cyclic data \\
 \verb| OCEAN_PHY_FILE_ENABLE_PERIODIC_DAY   = .false.,         | & ; whether dayly cyclic data \\
 \verb| OCEAN_PHY_FILE_STEP_FIXED            = 0,               | & ; step number when data at a cirtain time step is used. Set the value less than 1 for temporal varied data. \\
 \verb| OCEAN_PHY_FILE_CHECK_COORDINATES     = .true.,          | & ; whether coordinate variables are to be checked \\
 \verb| OCEAN_PHY_FILE_STEP_LIMIT            = 0,               | & ; maximum limit of steps. The data at the time step exceed this limit would not be read. 0 for no limit \\
 \verb|/                                         | & \\
}

If \nmitem{OCEAN_TYPE} $=$ \verb|"SLAB"|, the depth of the mixed slab layer can be specified in files \verb|init.conf| and \verb|run.conf|. In this case, the temperature in the mixed slab layer develops over time through heat flux exchange between the atmosphere and the ocean:
\editboxtwo{
 \verb|&PARAM_OCEAN_PHY_SLAB           | & \\
 \verb| OCEAN_PHY_SLAB_DEPTH = 10.0D0, | & ; Default [m] \\
 \verb|/                               | & \\
}
The albedo on the sea surface does not depend on the scheme selected, and
is calculated by using only the solar zenith angle.

\subsubsection{Flux in atmosphere and ocean}
%-------------------------------------------------------------------------------
Once a sea surface scheme except for \verb|NONE| or \verb|OFF| is specified, flux is exchanged at the interface of atmosphere and ocean. The fluxes between the atmosphere and the ocean are calculated by some kind of bulk schemes contained in \scalerm. The scheme for the length of the roughness on the sea surface is also selected from several schemes prepared. The scheme of roughness is specified in \nmitem{ROUGHNESS_TYPE} in \namelist{PARAM_ROUGHNESS} in file \verb|run.conf|. The scheme for the bulk transfer coefficient is specified in \nmitem{BULKFLUX_TYPE} in \namelist{PARAM_BULKFLUX} in file \verb|run.conf|. Refer to Section \ref{subsec:basic_usel_surface} for more details.


\subsection{Land model} \label{subsec:basic_usel_land}
%-------------------------------------------------------------------------------
Similar to the ocean model, the land model consists of two main parts, i.e., an update of the state of the land surface, and calculation of flux at the interface of atmosphere and land. The timing of the calling of these scheme is configured in \namelist{PARAM_TIME}. Refer to Section \ref{sec:timeintiv} for the detailed configuration of calling timing.

\subsubsection{Land surface scheme}
%-------------------------------------------------------------------------------
The land model scheme that updates the state of land, e.g., land surface temperature, soil temperature, and soil moisture, is configured as in \nmitem{LAND_TYPE} in \namelist{PARAM_LAND} in files \verb|init.conf| and \verb|run.conf|:

\editboxtwo{
\verb|&PARAM_LAND  | & \\
\verb| LAND_TYPE = "SLAB", | & ; Select the land type shown in Table \ref{tab:nml_land}\\
\verb|/             | & \\
}
\begin{table}[hbt]
\begin{center}
  \caption{Choices of land surface scheme}
  \label{tab:nml_land}
  \begin{tabularx}{150mm}{lX} \hline
    \rowcolor[gray]{0.9}  Values & Description of scheme \\ \hline
      \verb|NONE or OFF| & Do not use land surface model              \\
      \verb|SLAB|          & Heat diffusion/bucket model                   \\
      \verb|CONST|         & Do not update soil temperature, soil moister, and land surface temperature in \verb|SLAB| \\
    \hline
  \end{tabularx}
\end{center}
\end{table}

If the land surface type is included in the land use setup by \namelist{PARAM_LANDUSE}, neither NONE nor OFF can be given to \nmitem{LAND_TYPE}. If this condition is not satisfied, the program immediately terminates without computation, outputting the following message to the LOG file:
\msgbox{
\verb|xxx Land  fraction exists, but land  components never called. STOP.| \\
}

If \nmitem{LAND_TYPE} $=$ \verb|"SLAB"| or \nmitem{LAND_TYPE} $=$ \verb|"CONST"|,
it is necessary to prepare parameter tables for the length of roughness and the input to the land-use distribution.
A parameter table is provided 
in the file \verb|scale-rm/test/data/land/param.bucket.conf|.\\


\subsubsection{Flux in atmosphere and on land}
%-------------------------------------------------------------------------------
Once any land surface scheme except \verb|NONE| or \verb|OFF| is specified, flux is exchanged at the interface of the atmosphere and land. The flux between atmosphere and land is calculated by some kind of land scheme contained in \scalerm. If \nmitem{LAND_TYPE} $=$ \verb|"SLAB"| or \nmitem{LAND_TYPE} $=$ \verb|"CONST"|, the bulk scheme is the same as that for ocean or ideal land surface. This is specified in \nmitem{BULKFLUX_TYPE} in \namelist{PARAM_BULKFLUX} in file \verb|run.conf|. Refer to Section \ref{subsec:basic_usel_surface} for details.

\subsection{Urban model} \label{subsec:basic_usel_urban}
%------------------------------------------------------
The urban process consists of two main parts, i.e., updating the urban state and flux calculation at the interface of the atmosphere and the urban environment. The timing of calling these scheme is configured in \namelist{PARAM_TIME}. Refer to Section \ref{sec:timeintiv} for the detailed configuration of calling timing.

\subsubsection{Flux in atmosphere and urban area}
%-------------------------------------------------------------------------------

The urban scheme that updates the urban surface state, e.g., urban surface temperature and moisture, and calculates flux at the interface of atmosphere and the urban environment is configured in \nmitem{URBAN_TYPE} in \namelist{PARAM_URBAN} in \verb|init.conf| and \verb|run.conf|, as follows:

\editboxtwo{
\verb|&PARAM_URBAN         | & \\
\verb| URBAN_TYPE = "SLC", | & ; Select the urban type shown in Table \ref{tab:nml_urban}\\
\verb|/                    | & \\
}
\begin{table}[hbt]
\begin{center}
  \caption{Choices of urban scheme}
  \label{tab:nml_urban}
  \begin{tabularx}{150mm}{llX} \hline
    \rowcolor[gray]{0.9}  Value  & Description of scheme & reference \\ \hline
      \verb|NONE or OFF|  & Do not use the urban scheme            \\
      \verb|SLC|          & Single-layer canopy model  & \citet{kusaka_2001} \\
    \hline
  \end{tabularx}
\end{center}
\end{table}

If the type of urban area is included in the land-use setup by \namelist{PARAM LANDUSE}, neither \verb|NONE| nor \verb|OFF| can be given to \nmitem{URBAN_TYPE}. If this condition is not satisfied, the program immediately stops without computation, outputting the following message to LOG file:
\msgbox{
\verb|xxx Urban fraction exists, but urban components never called. STOP.| \\
}
