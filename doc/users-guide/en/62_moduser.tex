\section{Module for user defined settings} \label{sec:mod_user}

\scale prepares many options, users can specify as namelist, to meet users' demand for their calculations.
However, when there is no option you desire, you can set the desired settings in module of \verb|mod_user|.
This section describes what is \verb|mod_user| module and how to use it.


\subsection{What is \texttt{mod\_user} module}

\verb|mod_user| module is provided in \texttt{scale-{\version}/scale/scale-rm/src/user/mod\_user.F90}.\\
The module contains the following subroutines:
\begin{alltt}
  subroutine USER_tracer_setup
  subroutine USER_setup
  subroutine USER_mkinit
  subroutine USER_update
  subroutine USER_calc_tendency
\end{alltt}

\noindent These subroutines are called by \scale at the following timing.
\begin{alltt}
Initial setup
  IO setup
  MPI setup
  Grid settings
  Setup of administrator for dynamics and physics schemes 
  Tracer setup
  \textcolor{blue}{USER_tracer_setup}
  Setup topography, land
  Setup of vars and drivers for dynamics and physics schemes 
  \textcolor{blue}{USER_setup}
Main routine
  Time advance
  Ocean/Land/Urban/Atmos update
  \textcolor{blue}{User update}
  Output restart
  Calculation of tendency in Atmos/Ocean/Land/Urban
  \textcolor{blue}{Calculation of tendency}
  History output
\end{alltt}

Since the subroutines in \verb|mod_user| are basically called after handling the processes,
you can replace settings and output and add tracers and processes users desire.
%\verb|USER_tracer_setup| can add tracers to one difined according to physical schemes.
%When you change settings defined in setup process, use \verb|USER_setup|.
%\textcolor{blue}{User update}
Please refer to \verb|mod_user.F90| under test cases (\texttt{scale-{\version}/scale-rm/test/case}).

