%\chapter{Q \& A}

\clearpage
\section*{Answer}
\begin{enumerate}
\item {\bf How is the number of MPI processes changed while maintaining the computational domain?}\\
\nmitem{PRC_NUM_X, PRC_NUM_Y} in \namelist{PARAM_PRC} are changed.
If the below conditions are satisfied, your answer is correct:
\begin{eqnarray}
&& \verb|Number of MPIs parallel| = \nmitemeq{PRC_NUM_X} \times \nmitemeq{PRC_NUM_Y} = 6 \nonumber\\
&& \nmitemeq{IMAXG} \bmod \nmitemeq{PRC_NUM_X} = 0 \nonumber\\
&& \nmitemeq{JMAXG} \bmod \nmitemeq{PRC_NUM_Y} = 0 \nonumber
\end{eqnarray}


\item {\bf How is the computational domain altered while maintaining the number of MPI processes?}\\
If the number of grid points increases n times, the domain size also increases n times.
Thus, only \nmitem{IMAXG, JMAXG} in \namelist{PARAM_ATMOS_GRID_CARTESC_INDEX} are changed.
The red parts indicate the answer below:

\noindent {\small {\rm
\ovalbox{
\begin{tabularx}{140mm}{ll}
\verb|&PARAM_ATMOS_GRID_CARTESC_INDEX| & \\
\verb| KMAX = 36,|  & \\
\textcolor{red}{\verb| IMAXG = 120,|}  & (\verb|IMAXG = 90| in the original settings)\\
\textcolor{red}{\verb| JMAXG =  60,|}  & (\verb|JMAXG = 90| in the original settings)\\
\verb|/| & \\
\end{tabularx}
}}}\\

\item {\bf How is horizontal grid interval changed while maintaining the computational domain?}\\
When the number of MPI processes is not changed, \nmitem{DX, DY} in \namelist{PARAM_ATMOS_GRID_CARTESC} and \nmitem{IMAXG, JMAXG} in \namelist{PARAM_ATMOS_GRID_CARTESC_INDEX} should be changed.

\noindent {\small {\rm
\ovalbox{
\begin{tabularx}{140mm}{ll}
\verb|&PARAM_PRC_CARTESC|  & \\
\verb| PRC_NUM_X      = 2,|  & \\
\verb| PRC_NUM_Y      = 2,|  & \\
\\
\verb|&PARAM_ATMOS_GRID_CARTESC_INDEX| & \\
\verb| KMAX  = 36,|  & \\
\verb| IMAXG = 90,|  & \\
\verb| JMAXG = 90,|  & \\
\verb|/| &\\
 \\
\verb|&PARAM_ATMOS_GRID_CARTESC| & \\
\textcolor{red}{\verb| DX = 5000.D0,|} & (\verb|DX = 20000.D0| in the original settings)\\
\textcolor{red}{\verb| DY = 5000.D0,|} & (\verb|DY = 20000.D0| in the original settings)\\
\verb|/| & \\
\end{tabularx}
}}}\\

In the case that the number of MPI processes is changed, if the below conditions are satisfied in the setting of \namelist{PARAM_ATMOS_GRID_CARTESC}, your answer is correct:
\begin{eqnarray}
&& \nmitemeq{IMAXG} \bmod \nmitemeq{PRC_NUM_X} = 0 \nonumber\\
&& \nmitemeq{JMAXG} \bmod \nmitemeq{PRC_NUM_Y} = 0 \nonumber
\end{eqnarray}

Moreover, it is necessary to suitably change the time interval for the dynamics,\\
 i.e., \nmitem{TIME_DT_ATMOS_DYN} and \nmitem{TIME_DT} (refer to Section \ref{sec:timeintiv}).
 The width of the buffer region should be set between 20 and 40 times of the grid spacing.
 Below is an example of the answer, indicating the case where the buffer region was set to 20 times the grid spacing.

\noindent {\small {\rm
\ovalbox{
\begin{tabularx}{140mm}{ll}
\verb|&PARAM_PRC_CARTESC|  & \\
\verb| BUFFER_DX = 100000.D0, | & (\verb|BUFFER_DX = 400000.D0,| in the original settings) \\
\verb| BUFFER_DY = 100000.D0, | & (\verb|BUFFER_DY = 400000.D0,| in the original settings) \\
\verb|/| &\\
\end{tabularx}
}}}\\


\item {\bf How is the location of the computational domain changed?}\\

The coordinates of the center of the computational domain are changed as follows. Note that the unit is degrees. For example, 139 degrees 45.4 min = 139 + 45.4/60 degrees.

\noindent {\small {\rm
\ovalbox{
\begin{tabularx}{140mm}{ll}
\verb|&PARAM_MAPPROJECTION                   | & \\
\textcolor{red}{\verb| MAPPROJECTION_basepoint_lon = 139.7567D0,|} & (\verb|135.220404D0,| in the original setting)\\
\textcolor{red}{\verb| MPRPROJECTION_basepoint_lat =  35.6883D0,|} & (\verb|34.653396D0,| in the original setting)\\
\verb| MAPPROJECTION_type    = 'LC',         | & \\
\verb| MAPPROJECTION_LC_lat1 =  30.00D0,     | & \\
\verb| MAPPROJECTION_LC_lat2 =  40.00D0,     | & \\
\verb|/| & \\
\end{tabularx}
}}}\\


\item {\bf How is the integration time changed?}\\

\noindent {\small {\rm
\ovalbox{
\begin{tabularx}{140mm}{ll}
\verb|&PARAM_TIME| & \\
\verb| TIME_STARTDATE             = 2007, 7, 14, 18, 0, 0, | & \\
\verb| TIME_STARTMS               = 0.D0,                  | & \\
\textcolor{red}{\verb| TIME_DURATION = 12.0D0,             |}
                                                     &  (\verb| 6.0D0,| in the original settings) \\
\verb| TIME_DURATION_UNIT         = "HOUR",              | & \\
\verb|/| & \\
\end{tabularx}
}}}\\


Furthermore, it is necessary to prepare a boundary condition of more than 12 hours by \verb|scale-rm_init|. Referring to Section \ref{sec:adv_datainput}, \nmitem{NUMBER_OF_FILES} must be set greater than 3.


\item {\bf How are output variables added and their output interval changed?}\\

The output interval is set to \nmitem{FILE_HISTORY_DEFAULT_TINTERVAL} in \namelist{PARAM_FILE_HISTORY} as below. The output variables are specified at \nmitem{NAME} in \namelist{HISTORY_ITEM}. The list of the history variables is in the reference manual.
See Section \ref{sec:reference_manual} for the reference manual.\\

\noindent {\small {\rm
\ovalbox{
\begin{tabularx}{140mm}{ll}
\verb|&PARAM_FILE_HISTORY | & \\
\verb| FILE_HISTORY_DEFAULT_BASENAME  = "history_d01", | & \\
\textcolor{red}{\verb| FILE_HISTORY_DEFAULT_TINTERVAL = 1800.D0,|} & (\verb|3600.D0,| in the original settings) \\
\verb| FILE_HISTORY_DEFAULT_TUNIT     = "SEC",| & \\
\verb|/| & \\
\\
\textcolor{red}{\verb|&HISTORY_ITEM NAME="SFLX_SW_up" /|} & \textcolor{red}{added}\\
\textcolor{red}{\verb|&HISTORY_ITEM NAME="SFLX_SW_dn" /|} & \textcolor{red}{added}\\
\\
\end{tabularx}
}}}\\



\item {\bf How does the model restart?}\\

The first integration for three hours is configured in \verb|run.conf| as below.
Once the three hours for integration have elapsed, the restart file is created.\\

\noindent {\small {\rm
\ovalbox{
\begin{tabularx}{150mm}{ll}
\verb|&PARAM_TIME| & \\
\verb| TIME_STARTDATE             = 2007, 7, 14, 18, 0, 0, | & \\
\verb| TIME_STARTMS               = 0.D0, | & \\
\textcolor{red}{\verb| TIME_DURATION              = 3.0D0, |} ~~~~~~~~~necessary for more than 3 hours & \\
\verb| TIME_DURATION_UNIT         = "HOUR", | & \\
\verb| ....(omission)....| & \\
\textcolor{red}{\verb| TIME_DT_ATMOS_RESTART      = 10800.D0, |} & \\
\textcolor{red}{\verb| TIME_DT_ATMOS_RESTART_UNIT = "SEC",    |} & \\
\textcolor{red}{\verb| TIME_DT_OCEAN_RESTART      = 10800.D0, |} & \\
\textcolor{red}{\verb| TIME_DT_OCEAN_RESTART_UNIT = "SEC",    |} & \\
\textcolor{red}{\verb| TIME_DT_LAND_RESTART       = 10800.D0, |} & \\
\textcolor{red}{\verb| TIME_DT_LAND_RESTART_UNIT  = "SEC",    |} & \\
\textcolor{red}{\verb| TIME_DT_URBAN_RESTART      = 10800.D0, |} & \\
\textcolor{red}{\verb| TIME_DT_URBAN_RESTART_UNIT = "SEC",    |} & \\
\verb|/| & \\
\\
\verb|&PARAM_RESTART | & \\
\textcolor{red}{\verb| RESTART_OUTPUT      = .true.,|} ~~~~~~~~~~~~(\verb|.false.,| in the original setting) & \\
\verb| RESTART_IN_BASENAME = "../init/init_d01_20070714-180000.000",| &  \\
\textcolor{red}{\verb| RESTART_OUT_BASENAME = "restart_d01",|} ~~~~~~~~~~~~~~~(added)& \\
\verb|/| & \\
\\
\verb|&PARAM_ATMOS_BOUNDARY| & \\
\verb| ATMOS_BOUNDARY_TYPE           = "REAL",                            | & \\
\verb| ATMOS_BOUNDARY_IN_BASENAME    = "../init/output/boundary_d01",     | & \\
\verb|/| & \\
\\
\verb|&PARAM_ATMOS_REFSTAT| & \\
\verb| ATMOS_REFSTATE_TYPE      = "INIT",  | & \\
\textcolor{red}{\verb| ATMOS_REFSTART_UPDATE_DT = 10800.D0,|} &\\
\verb|/| & \\
\end{tabularx}
}}}\\


When \nmitem{TIME_DURATION} is set to three hours and \nmitem{RESTART_OUTPUT} is \verb|.true.|, the restart file is created at the end of the integration. Therefore, \nmitem{TIME_DT_ATMOS_RESTART}, \\
\nmitem{TIME_DT_OCEAN_RESTART}, \nmitem{TIME_DT_LAND_RESTART}, and \nmitem{TIME_DT_URBAN_RESTART} are not necessary.\\
When \nmitem{TIME_DURATION} is set to more than three hours, it is necessary to specify \\
\nmitem{TIME_DT_ATMOS_RESTART}, \nmitem{TIME_DT_OCEAN_RESTART}, \nmitem{TIME_DT_LAND_RESTART}, and \\
\nmitem{TIME_DT_URBAN_RESTART}.
The values for \nmitem{TIME_DT_ATMOS_RESTART}, \nmitem{TIME_DT_OCEAN_RESTART}, \nmitem{TIME_DT_LAND_RESTART}, and \nmitem{TIME_DT_URBAN_RESTART} must be a divisor of three hours (10800 s) and a multiple of \nmitem{TIME_DT}.
In order to make the reference state (See Section \ref{subsec:refstat}) in the restart calculation identical to that in the first calculation, the update interval of the reference state is specified to be a divisor of three hours.


The configuration for the restart calculation from three to six hours in integration time is as follows:\\

\noindent {\small {\rm
\ovalbox{
\begin{tabularx}{150mm}{ll}
\verb|&PARAM_TIME| & \\
\textcolor{red}{\verb| TIME_STARTDATE             = 2007, 7, 14, 21, 0, 0, |} & \\
\verb| TIME_STARTMS               = 0.D0, | & \\
\textcolor{red}{\verb| TIME_DURATION              = 3.0D0, |}    & set for longer than 3 hours\\
\verb| TIME_DURATION_UNIT         = "HOUR", | & \\
\verb|/| & \\
\\
\verb|&PARAM_RESTART | & \\
\verb| RESTART_OUTPUT      = .true.,                |                  & (not always necessary)\\
\textcolor{red}{\verb| RESTART_IN_BASENAME = "restart_d01_20070714-210000.000",|} & (\textcolor{red}{necessary})\\
\verb| RESTART_OUT_BASENAME = "restart2_d01",| & (not alway necessary)\\
\verb|/| & \\
\\
\verb|&PARAM_ATMOS_BOUNDARY| & \\
\verb| ATMOS_BOUNDARY_TYPE           = "REAL",                            | & \\
\verb| ATMOS_BOUNDARY_IN_BASENAME    = "../init/output/boundary_d01",     | & \\
\verb|/| & \\
\\
\verb|&PARAM_ATMOS_REFSTAT| & \\
\verb| ATMOS_REFSTATE_TYPE      = "INIT",  | & \\
\textcolor{red}{\verb| ATMOS_REFSTART_UPDATE_DT = 10800.D0,|} &\\
\verb|/| & \\
\end{tabularx}
}}}\\



\end{enumerate}

