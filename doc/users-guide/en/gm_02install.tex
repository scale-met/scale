%###############################################################################
\section{実行の準備}
%-------------------------------------------------------------------------------
ここではSCALE-GMのコンパイルについて簡潔に説明する。依存関係のあるライブラリの準備や
コンパイル前に必要な環境変数の設定については、``SCALE USERS GUIDE''の第2章の
2.1節、2.2節を参照されたい。
SCALE-GMに必要とされるライブラリは、SCALE-RMと同じく、HDF5、NetCDF、およびMPIである。

\section{コンパイル}
%-------------------------------------------------------------------------------
ソースコードの取得方法については ``SCALE USERS GUIDE'' の2.3.1節、
マシン環境やコンパイラ等を指定する環境変数(Makedef)の設定については
 ``SCALE USERS GUIDE'' の2.3.2節を参照されたい。
Makedefファイルは、SCALE-RMと共通のファイルを使用する。
このMakedefファイルの設定をコンパイル前に忘れないようにすること。
これらの準備が整えば、srcディレクトリへ移動する。
SCALE-GMでは、\texttt{scale-{\version}/scale-gm/src}の下でコンパイルを行い、\texttt{scale-{\version}/bin}の下に
各種実行バイナリが生成され、このバイナリファイルを用いて各種実験や後処理を行う。
各テストケースディレクトリでもmakeできるが、作業効率やバイナリの取り違えを防ぐために、
ここで説明する方法を推奨する。\\

\verb|  > cd scale-|{\version}\verb|/scale-gm/src|\\


\noindent makeコマンドを用いてコンパイルを行う。\\

\verb|  > make -j 4|\footnote{makeコマンドの -j オプションはコンパイル時に使用する並列プロセス数を指定している。   コンパイルにかかる時間を短縮するため、2以上の数を指定することで並列コンパイルを行うことができる。SCALE-GMでは、2 $\sim$ 8 プロセスの指定を推奨する。}\\

\verb|コンパイルが正常に終了したならば、以下のバイナリが|\texttt{scale-{\version}/bin}の下に生成される。
 \begin{itemize}
   \item \verb|scale-gm| (SCALE-GM本体の実行バイナリ)
   \item \verb|gm_fio_cat| (fioフォーマットのcatコマンドツール)\footnote{``fio'' とは、ヘッダー情報付きバイナリをベースにした独自ファイルフォーマットである。}
   \item \verb|gm_fio_dump| (fioフォーマットのファイルをダンプするツール)
   \item \verb|gm_fio_ico2ll| (fioフォーマットの二十面体格子データをLatLon格子データに変換するツール)
   \item \verb|gm_fio_sel| (fioフォーマットのselコマンドツール)
   \item \verb|gm_mkhgrid| (バネ格子を適用した二十面体の水平格子を作成するツール)
   \item \verb|gm_mkllmap| (LatLonの水平格子を作成するツール)
   \item \verb|gm_mkmnginfo| (MPIプロセスのマネージメントファイルを作成するツール)
   \item \verb|gm_mkrawgrid| (正二十面体の水平格子を作成するツール)
   \item \verb|gm_mkvlayer| (鉛直格子を作成するツール)
 \end{itemize}


