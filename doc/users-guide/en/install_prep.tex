This chapter explains how to compile \scalelib / \scalerm
and the minimum computational requirements for their execution.

\section{Necessary system environment} \label{sec:req_env}
%====================================================================================
\subsubsection{Recommended system environment}

\begin{table}[b]
\begin{center}
\caption{OSs already checked (the ones below are based on x86-64.)}
\begin{tabularx}{150mm}{|l|l|X|} \hline
 \rowcolor[gray]{0.9} Name of OS & Version & Note \\ \hline
 CentOS                & 6.6、6.9、7.0、7.2、7.3 &  \\ \hline
 openSUSE              & 13.2、42.1、42.2        &  \\ \hline
 SUSE Enterprise Linux & 11.3、12.1         &  \\ \hline
 fedora                & 24、25、26         &  \\ \hline
 Mac OS X              & 10.11 (El Capitan) &  \\ \hline
\end{tabularx}
\label{tab:compatible_os}
\end{center}
\end{table}


\begin{itemize}
  \item {\bf Hardware system}\\
  Although the necessary hardware depends on the experiment to be performed,
the following are the specifications for conducting a tutorial as in Chapters \ref{chap:tutorial_ideal} and \ref{chap:tutorial_real}.
  \begin{itemize}
    \item {\bf CPU} :
    The system need to have more than two and more than four physical cores
    for an ideal experiment and a real atmospheric experiment, respectively.
    \item {\bf Memory} :
    The system needs 512 MB and 2 GB memory
    for the ideal experiment and a real atmospheric experiment, respectively.
    Note that this applies to the case involving the use of double-precision floating points.
    \item {\bf HDD} : The system needs to have 3 GB of free disk space in a real atmospheric experiment.
  \end{itemize}
  \item {\bf System software}
  \begin{itemize}
  \item {\bf OS} : Linux OS, Mac OS X\\
  Refer to Table \ref{tab:compatible_os} for other OSs currently supported.
  \item {\bf Compiler} : C, Fortran\\
  The FORTRAN compiler should support FORTRAN 2003 syntax.
  Refer to Table \ref{tab:compatible_compiler} for compilers already confirmed as supported.
  \item {\bf MPI library} :
  The MPI library should support the MPI 1.0/2.0 protocol.  Refer to Table \ref{tab:compatible_compiler} for MPI libraries already confirmed as supported.
  \item {\bf File I/O library} :
  The development libraries of gzip, HDF5, and {\netcdf}4 are required.\\
  In stead of the HDF5/{\netcdf}4, {\netcdf}3 is sufficient, but the size of the output file increases in this case.
  \end{itemize}
\end{itemize}


\subsubsection{Useful tools (not always necessary)}
\begin{itemize}
  \item {\bf Data conversion tool}: wgrib, wgrib2, NCL\\
  In the tutorial for the real atmospheric experiment, wgrib is used.
  \item {\bf Viewer tool}:\grads \footnote{\url{http://cola.gmu.edu/grads/}},
  GPhys/Ruby-DCL\footnote{\url{https://www.gfd-dennou.org/arch/ruby/products/gphys/}},
  ncview\footnote{\url{http://meteora.ucsd.edu/~piece/ncview\_home\_page.html}}, and so on.
  \item {\bf Evaluation tool of computational performance}:The PAPI library\footnote{\url{http://icl.utk.edu/papi/}} is available.
\end{itemize}


\begin{table}[tb]
\begin{center}
\caption{Compilers and MPI libraries already checked}
\begin{tabularx}{150mm}{|l|X|X|} \hline
 \rowcolor[gray]{0.9} Name of compiler & Version & Note \\ \hline
 GNU (gcc/gfortran)    & 4.4.7、4.7.4、4.8.5、4.9.4、6.1.1、6.3.1、7.1.1 & Version 4.3 or earlier are not supported. The series with version 4.4.x sometimes issues a warning. OpenMP is supported. \\ \hline
 Intel (icc/ifort)     & 13.0.1、14.0.2、16.0.1、17.0.1、17.0.2  & Version of 2013 or later are recommended. OpenMP is supported. \\ \hline
 PGI (gcc/pgfortran)   & 17.1                 & OpenACC is supported. \\ \hline
 \rowcolor[gray]{0.9} Name of MPI library &  &  \\ \hline
 openMPI   & 1.7.2、1.8.1、1.10.2、1.10.3、1.10.5、1.10.7、2.0.2、2.1.0、2.1.1 & \\ \hline
 Intel MPI & 4.1.0、5.1.2、2017.2 & The version released in 2013 or later is recommended.\\ \hline
 SGI MPT   & 2.09、2.14               & A combined use with the Intel compiler is available. \\ \hline
\end{tabularx}
\label{tab:compatible_compiler}
\end{center}
\end{table}

Since the source code of \scalelib is  written in FORTRAN 2003 standard syntax, the compiler must support it. For example, GNU gfortran version 4.3 or previous versions cannot be used for \scalelib compilation because they do not follow the FORTRAN 2003 standard.

In addition to the above environment, it has been confirmed that \scalelib can be compiled and run on  machines with a SPARC-based instruction set, such as the K Computer.



\section{Installation of external libraries and drawing tools} \label{sec:inst_env}
%====================================================================================
\subsubsection{Required libraries}
The required external libraries are described below; each library is commonly used.
\begin{itemize}
\item HDF5 Library (\url{https://www.hdfgroup.org/HDF5/})
\item {\netcdf} Library (\url{http://www.unidata.ucar.edu/software/netcdf/})
\item MPI Library (openMPI version、\url{http://www.open-mpi.org/})
\item LAPACK ( \url{http://www.netlib.org/lapack/} ) (only required for \scalegm)
\end{itemize}
The interested reader can refer to the above URLs for the installation information of each library.  Since most series of Linux distributions provide them as packages, their installation is easy. These libraries have been already installed on the K Computer as AICS software. Refer to the description ``AICS software'' in the K portal site or the URL\\ \url{http://www-sys-aics.riken.jp/releasedsoftware/ksoftware/pnetcdf.html}\\ for details. The library paths are described there in ``Makedef.K.'' The tutorials in Chapters \ref{chap:tutorial_ideal} and \ref{chap:tutorial_real} assume that the above library environment has been prepared.


\subsubsection{Drawing tools}
The following are typical drawing tools that can draw
the initial conditions, boundary data, and the simulation results on \scalelib.
The GPhys and \grads are used for a quick view
and the drawing model output in the tutorial in Chapters \ref{chap:tutorial_ideal} and \ref{chap:tutorial_real}, respectively.
Other tools are of course available,
if they can be read in \netcdf format, which is the output format of \scalelib.

\begin{itemize}
\item GPhys / Ruby-DCL by GFD DENNOU Club
 \begin{itemize}
  \item URL: \url{http://ruby.gfd-dennou.org/products/gphys/}
  \item Note: \scalelib outputs the split files
  in {\netcdf} format according to domain decomposition by the MPI process.
  "gpview" and/or "gpvect" in {\gphys} can directly draw the split file without post-processing.
  \item How to install:
  On the GFD DENNOU Club webpage, the installation is explained for major OSs\\
  \url{https://www.gfd-dennou.org/library/ruby/tutorial/install/index-j.html}\\
   \end{itemize}
\item Grid Analysis and Display System (\grads) by COLA
 \begin{itemize}
  \item URL: \url{http://cola.gmu.edu/grads/}
  \item Note:This is among the most popular drawing tools,
  but the split files with {\netcdf} generated by \scalelib are not directly readable.
  The post-processing tool \verb|netcdf2grads_h| is needed to combine the output of \scalelib in one file that is readable from \grads. Refer to \verb|netcdf2grads_h| for installation instructions in Section \label{sec:source_net2g}, and to Chapters 3 and 4 and Section \label{sec:net2g} for details pertaining to its use.
  \item How to install: Refer to \url{http://cola.gmu.edu/grads/downloads}.
 \end{itemize}
\item Ncview: a {\netcdf} visual browser developed by David W. Pierce
 \begin{itemize}
  \item URL: \url{http://meteora.ucsd.edu/~pierce/ncview_home_page.html}
  \item Note: Ncview is a quick viewer for the {\netcdf} file format.
  Although it cannot combine split files in \scalelib, it is useful in drawing the result file by file.
  \item How to install: Refer to \url{http://meteora.ucsd.edu/~pierce/ncview_home_page.html}
 \end{itemize}
\end{itemize}


