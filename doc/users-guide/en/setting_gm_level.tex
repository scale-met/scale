\section{Terms of SCALE-GM}
%-------------------------------------------------------------------------------

 \begin{itemize}
   \item g-level (grid level): number of subdivision times of the grid from the original icosahedron.
         the number starts from 1, we recommend to use the number larger than 4.
   \item r-level (region level): number of subdivision times of the region(tile)
         from the original icosahedron. When r-level = 0, we have ten regions(tiles).
         At that time, the number of available maximum MPI processes is ten.
 \end{itemize}


\begin{figure}[h]
\centering
\includegraphics[width=15cm]{figure/g-level_concept.jpg}
\caption{g-levelの概念図}
\centering
\includegraphics[width=10cm]{figure/r-level_concept.jpg}
\caption{r-levelの概念図}
\end{figure}

\textcolor{red}{[g-levelとr-levelを説明するための適切な図を追加する必要がある:とりあず佐藤さんの発表資料より拝借!したけど、copyright要確認。さらにHALOを説明する図もある方が良い]}

