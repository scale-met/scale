\section{この章の最後に} \label{sec:ideal_exp_last}

本章では、簡単な理想実験を例にして{\scalerm}の実行方法を説明した。
次の段階としては、解像度や計算領域、MPIプロセス数の変更を一度試してみることを勧める。
本章の理想実験に関しては、同じディレクトリ下にある「sample」ディレクトリ内に、
解像度設定、領域設定、物理スキーム等を変更した設定ファイルを数種類用意してある。
これらは、設定を変更するときに参考となるだろう。
また、「\verb|scale-rm/test/case|」以下には、様々な理想実験に対する設定が用意されている。
幾つかの理想実験については、それらの実験設定に特化したソースコードを必要とするために、
設定ファイルのあるディレクトリでmakeコマンドを再度実行する必要がある。
初期値作成と実行の手順は、基本的に本章のチュートリアルと同じである。

雲微物理スキーム, 放射スキーム, 乱流スキームといった物理過程の設定方法について
確認することも重要である。これらの変更方法は、第\ref{part:basic_usel}章に記載されている。
