\section{\SecCommonSetting} \label{sec:common}
%------------------------------------------------------

\subsection{\SubsecCalendarSetting} \label{subsec:calendar}
%------------------------------------------------------
シミュレーション内の暦は、\verb|init.conf|と\verb|run.conf|の中の
\namelist{PARAM_CALENDAR}で設定する。デフォルトの暦はグレゴリオ暦であり、
必要に応じて異なる暦を選択できる。

\noindent {\small {\gt
\ovalbox{
\begin{tabularx}{150mm}{lX}
\verb|&PARAM_CALENDAR             | & \\
\verb| CALENDAR_360DAYS = .false. | & 12ヶ月x30日の暦を利用するかどうか \\
\verb| CALENDAR_365DAYS = .false. | & うるう年の存在しない暦を利用するかどうか \\
\verb|/                           | & \\
\end{tabularx}
}}}\\

暦の設定の影響を受けるのは、太陽放射計算と時刻情報を持った外部ファイルの読み込みである。
太陽放射は暦で設定された1年の長さと、太陽の周りを一周する長さが一致するように計算される。
読み込む外部ファイルは、同じ暦の設定で作成されたデータ以外は利用すべきでないことに注意されたい。

\nmitem{CALENDAR_360DAYS}が\verb|.true.|の場合、1年が12ヶ月、1ヶ月が30日であるような暦が設定される。
この設定は理想実験でしばしば用いられる。
%
\nmitem{CALENDAR_365DAYS}が\verb|.true.|の場合、1年が12ヶ月で、1ヶ月の長さはグレゴリオ暦と同じであるが、
うるう年がない暦が設定される。
