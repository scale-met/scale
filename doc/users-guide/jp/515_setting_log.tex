\section{ログファイル} \label{sec:log}
%====================================================================================


\subsection{ログファイルの出力}

\verb|scale-rm|、 \verb|scale-rm_init|、 \verb|scale-rm_pp|を実行するときに、
\scalerm はログファイルを出力できる。
デフォルト設定において、\verb|scale-rm|では"\verb|LOG.pe000000|"、
\verb|scale-rm_init|では"\verb|init_LOG.pe000000|"、\verb|scale-rm_pp|では"\verb|pp_LOG.pe000000|"に、番号ゼロのプロセスからのログメッセージが書き込まれる。
ユーザーは以下のように設定ファイルを編集することで、ログファイルの出力設定を変更できる。

\editboxtwo{
\verb|&PARAM_IO                       | & \\
\verb| IO_LOG_BASENAME     = 'LOG',   | & ; ログファイルのベース名 \\
\verb| IO_LOG_ALLNODE      = .false., | & ; 全てのプロセスに対するログファイルを出力するか? \\
\verb| IO_LOG_SUPPRESS     = .false., | & ; .true.であれば、 ログの出力を抑制する \\
\verb| IO_LOG_NML_SUPPRESS = .false., | & ; .true.であれば、 ネームリストのパラメータの出力を抑制する \\
\verb| IO_NML_FILENAME     = '',      | & ; 指定された場合には、ネームリストのパラメータを指定したファイルに出力する。指定がなければ、ログファイルに出力する。\\
\verb| IO_STEP_TO_STDOUT   = -1,      | & ; 正であれば、時間ステップの情報を標準出力に出力する \\
\verb|/                               | & \\
}

ログファイルの名前は\namelist{PARAM_IO}の\nmitem{IO_LOG_BASENAME}で設定する。
上記のデフォルト設定の場合は、マスタープロセスに対するログファイル名は"\verb|LOG.pe000000|"である。
全てのプロセスに対するログファイルを出力するかは、\nmitem{IO_LOG_ALLNODE}で制御する。
\nmitem{IO_LOG_ALLNODE}を\verb|.true.|とした場合は、全プロセスに対するログファイルが生成され、
そうでない場合はログファイルはマスタープロセス(すなわちランク0)からのみ出力される。

\nmitem{IO_LOG_SUPPRESS}を\verb|.true.|に設定した場合はログファイルは生成されず、
ほぼ全てのログメッセージが出力されない。
この場合でも、経過時間に関する情報だけは標準出力(STDOUT)に送られる。

ネームリストのパラメータは、\nmitem{IO_LOG_NML_SUPPRESS}を\verb|.true.|に設定した場合を除いて出力される。
デフォルトでは、パラメータはログファイルに出力される。
\nmitem{IO_NML_FILENAME}を設定することによって、異なるファイルに出力することもできる。
\nmitem{IO_NML_FILENAME}で指定したファイルは、その後の実行の入力設定ファイルとして用いることができる。

時間ステップの情報はログファイルに出力される。その情報の詳細は次節で説明する。\\
\nmitem{IO_STEP_TO_STDOUT}$>0$と設定されている場合は、時間ステップの情報は標準出力にも出力される。
全ての時間ステップの情報はログファイルに出力される。
標準出力については、出力するステップ間隔を\nmitem{IO_STEP_TO_STDOUT}に数値で指定する。


\subsection{ログファイル内の時間に関する情報}

\verb|scale-rm|を実行すると、ログファイル中に以下の形式の行を見つけることができる。
\msgbox{
  +++++ TIME: 0000/01/01 00:06:36 + 0.600 STEP:   1984/ 432000 WCLOCK:    2000.2 \\
}
この行は、以下のような計算状況に関するメッセージを意味する。
\begin{itemize}
 \item 現在、初期時刻「0000/01/01 00:00:00 + 0.000」から 6m36.6s の時間積分が行われたこと。
 \item 今回の時間ステップは、全時間ステップ数である 432000 回中の 1984 回目であること。
 \item 経過時間(cpu 時間)は 2000.2s であること。
\end{itemize}
さらに、これらの情報から本計算に必要な時間を推測できる。
この場合には、推定される所要時間は、121 時間( $= 2000.2 \times 432000 \div 1984$ )である。

\vspace{2ex}
ログファイル内のメッセージは以下の形式で出力される。
\msgbox{
\texttt{{\it type} [{\it subroutine name}] {\it message}} \\
\texttt{\hspace{2em}{\it messages}} \\
\texttt{\hspace{2em} ... }
}
\begin{description}
 \item[{\it type}]: メッセージの種類(以下の中の1つをとる)
   \begin{itemize}
    \item INFO: ジョブ実行に関する一般的な情報
    \item WARN: ジョブ実行に関する重要な出来事
    \item ERROR: 実行停止を伴う致命的なエラー
   \end{itemize}
 \item[{\it subroutine name}]: メッセージを書き込んだサブルーチンの名前
 \item[{\it message}]: メッセージの本文
\end{description}


\noindent 以下は、エラーメッセージの例である。
\msgbox{
\verb|ERROR [ATMOS_PHY_MP_negative_fixer] large negative is found. rank =            1| \\
\verb|      k,i,j,value(QHYD,QV) =           17           8           1   1.7347234759768071E-018   0.0000000000000000| \\
\verb|      k,i,j,value(QHYD,QV) =           19           8           1  -5.4717591620764856E-003   0.0000000000000000| \\
\verb|  ... |
}
