%\section{Setting the common components} \label{sec:common}
%------------------------------------------------------

%-------------------------------------------------------------------------------
\section{物理定数の設定} \label{subsec:const}
%-------------------------------------------------------------------------------

いくつかの物理定数の値は、設定ファイルの\namelist{PARAM_CONST}で変更できる。

\editboxtwo{
\verb|&PARAM_CONST                           | & \\
\verb| CONST_RADIUS            = 6.37122D+6, | & ; 惑星半径 [m] \\
\verb| CONST_OHM               = 7.2920D-5,  | & ; 惑星の自転角速度 [1/s] \\
\verb| CONST_GRAV              = 9.80665D0,  | & ; 標準重力加速度 [m/s2] \\
\verb| CONST_Rdry              = 287.04D0,   | & ; 気体定数(乾燥空気)  [J/kg/K] \\
\verb| CONST_CPdry             = 1004.64D0,  | & ; 定圧比熱(乾燥空気) [J/kg/K] \\
\verb| CONST_LAPS              = 6.5D-3,     | & ; 国際標準大気の温度減率  [K/m] \\
\verb| CONST_Pstd              = 101325.D0,  | & ; 標準圧力 [Pa] ]\\
\verb| CONST_PRE00             = 100000.D0,  | & ; 圧力の参照値 [Pa] \\
\verb| CONST_Tstd              = 288.15D0,   | & ; 標準温度 (15$^\circ$C) [K] \\
\verb| CONST_THERMODYN_TYPE    = 'EXACT',    | & ; 内部エネルギーの定式化の種類  \\
\verb| CONST_SmallPlanetFactor = 1.D0,       | & ; 小さな惑星設定に対するファクタ [1] \\
\verb|/                                      | & \\
}

\noindent

\nmitem{CONST_THERMODYN_TYPE}が'EXACT'の場合は、潜熱の温度依存性が考慮される。
%
\nmitem{CONST_THERMODYN_TYPE}が'SIMPLE'の場合は、水物質の各カテゴリーの比熱は乾燥空気の定積比熱に設定し、
潜熱の温度依存性は無視する。
%
\nmitem{CONST_RADIUS} には\nmitem{CONST_SmallPlanetFactor}が掛けられる.
同時に、\nmitem{CONST_OHM}には\nmitem{CONST_SmallPlanetFactor}の逆数が掛けられる.


%-------------------------------------------------------------------------------
\section{暦の設定} \label{subsec:calendar}
%-------------------------------------------------------------------------------

暦の種類は、設定ファイルの\namelist{PARAM_CALENDAR}において指定することができる。
デフォルトでは、グレゴリオ暦が用いられる。

\editboxtwo{
\verb|&PARAM_CALENDAR             | & \\
\verb| CALENDAR_360DAYS = .false. | & ; 12x30 days の暦を用いるか? \\
\verb| CALENDAR_365DAYS = .false. | & ; うるう年を考慮するか? \\
\verb|/                           | & \\
}

\noindent
暦の設定は太陽天頂角の計算に影響を及ぼし、
一年の長さと黄道の一周が一致するように計算される。
異なるカレンダーを用いる外部データは、読み込むべきでないことに注意されたい。

\nmitem{CALENDAR_360DAYS}が\verb|.true.|である場合は、
1年が12ヶ月、 1ヶ月が30日ある暦を設定する。
%
\nmitem{CALENDAR_365DAYS}が\verb|.true.|である場合は、
うるう年の無いグレゴリオ暦を用いる。


%-------------------------------------------------------------------------------
\section{乱数生成の設定} \label{subsec:random}
%-------------------------------------------------------------------------------

乱数生成のパラメータは、設定ファイルの\namelist{PARAM_RANDOM}において設定する。

\editboxtwo{
\verb|&PARAM_RANDOM         | & \\
\verb| RANDOM_FIX = .false. | & ; 乱数のシードを固定するか? \\
\verb|/                     | & \\
}

\noindent
scale library では、乱数生成のための組み込み関数を用いる。
生成される乱数は擬似乱数であるこちに注意が必要である。
乱数のシードは、現在の日時、cpu 時間、プロセス ID によって決定される。
%
\nmitem{RANDOM_FIX}が\verb|.true.|である場合は, シードは特定の数字に固定される。
このオプションは、初期場に対してランダムな擾乱を用いるシミュレーションの結果を再現するときに便利である。


%-------------------------------------------------------------------------------
\section{パフォーマンス測定の設定} \label{subsec:prof}
%-------------------------------------------------------------------------------

パフォーマンス測定のためのパラメータは、設定ファイルの\namelist{PARAM_PROF}で与える。

\editboxtwo{
\verb|&PARAM_PROF                 | & \\
\verb| PROF_rap_level   = 2       | & ; ラップを測定するレベル \\
\verb| PROF_mpi_barrier = .false. | & ; ラップ毎に MPI のバリア命令を追加するか? \\
\verb|/                           | & \\
}

\noindent
経過時間を測定するために、ソースコードにはユーティリティ関数(PROF\_rapstart,PROF\_rapend)が埋め込まれている。
これらの測定区間は、詳細なパフォーマンス測定のためにも用いられる。
%
ラップ時間の結果は、ログファイルの終わりに表示される。
\namelist{PARAM_IO}の\nmitem{IO_LOG_ALLNODE}が\verb|.true.|であるならば、
各プロセスの結果が個々のログファイルに報告される。
\namelist{PARAM_IO}の\nmitem{IO_LOG_SUPPRESS}が\verb|.true.|であるならば、
結果は標準出力に送られる。
%
それぞれの測定区間は、出力に関するレベルを持つ。
\nmitem{PROF_rap_level}よりも大きな出力のレベルを持つ区間は、経過時間が測定されない。

\nmitem{PROF_mpi_barrier}が\verb|.true.|である場合は、現在時刻を取得する前後において MPI のバリア命令を呼ぶ。
このオプションは、計算時間と通信時間を分離するために役立つ。
計算時間は、プロセス間の大きな不均衡をしばしば示す。


%-------------------------------------------------------------------------------
\section{統計量のモニターの設定} \label{subsec:statistics}
%-------------------------------------------------------------------------------

統計量のモニターに関するパラメータは、設定ファイルの\namelist{PARAM_STATISTICS}で与える。

\editboxtwo{
\verb|&PARAM_STATISTICS                    | & \\
\verb| STATISTICS_checktotal     = .false. | & ; 変数の合計量を計算し、ログファイルへ出力するか? \\
\verb| STATISTICS_use_globalcomm = .false. | & ; 全通信を用いて全量を計算するか? \\
\verb|/                                    | & \\
}

\noindent
\nmitem{STATISTICS_checktotal}が\verb|.true.|である場合は, 
デバッグのために、いくつかの変数の領域内の合計量を計算し、ログファイルへ出力する。
%
\nmitem{STATISTICS_use_globalcomm}が\verb|.true.|である場合は, 
全通信を用いて領域全体の合計量を計算するが、
シミュレーション時間が長くなる可能性がある。
このオプションが\verb|.false.|の場合は、各プロセスに対して割り当てられた空間領域内での合計量が計算される。


