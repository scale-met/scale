\section{\SecInputDataSetting} \label{sec:adv_datainput}
%====================================================================================

\begin{table}[htb]
\begin{center}
\caption{\scalelib で対応している外部入力データ}
\begin{tabularx}{150mm}{|l|l|X|} \hline
 \rowcolor[gray]{0.9} データ形式      & \verb|FILETYPE_ORG|  & 備考 \\ \hline
 SCALEデータ形式   & \verb|SCALE-RM|     &  ヒストリファイルのみ対応。latlonカタログを必要とする。 \\ \hline
 バイナリ形式 & \verb|GrADS|        & データ読み込み用のネームリストを別途必要とする。       \\ \hline
% NICAMデータ   & \verb|NICAM-NETCDF| & NetCDF形式の緯度経度格子に変換されたデータに対応する。 \\ \hline
 WRFデータ形式     & \verb|WRF-ARW|      & 「wrfout」、「wrfrst」の両方に対応する。          \\ \hline
\end{tabularx}
\label{tab:inputdata_format}
\end{center}
\end{table}

\scalerm では、表\ref{tab:inputdata_format}に示される様々な種類の外部データを読み込むことによって、初期値データや境界値データを作成できる。
プログラム\verb|scale-rm_init|は、ファイル\verb|init.conf|の設定に従って外部データを初期値・境界値データに変換する。
入力データの形式は、\namelist{PARAM_MKINIT_REAL_***}の\nmitem{FILETYPE_ORG}で指定する。

SCALEデータ形式は主にオフライン・ネスティング実験で使用される。
詳細については、第\ref{subsec:nest_offline}節を参照されたい。

WRFデータ形式も使用でき、WRFによるモデル出力データを直接使用できる。
ただし、ファイルは{\scalerm}の境界値データの作成に必要な全てのデータを含まなければならない。

本書における「バイナリデータ形式」は、Fortran が直接アクセスできる単精度浮動小数点のバイナリデータとして定義される。
GRIB/GRIB2 データは、バイナリデータ形式に変換することで {\scalerm} に読み込ませることができる。
この方法は、第\ref{sec:tutrial_real_data}節で説明される。
その他の任意のデータについても、バイナリデータ形式に変換すれば使用できる。

{\scalelib}の最新版の出力ファイル形式は、バージョン5.2以前の形式とは異なる。
そのため、バージョン5.2以前で作成された初期値/境界値ファイルは{\scalelib}{\version}では使用できない。

%%%---------------------------------------------------------------------------------%%%%
\subsubsection{バイナリ形式データの入力} \label{sec:datainput_grads}

入力データの形式は、設定ファイル\verb|init.conf|の\namelist{PARAM_MKINIT_REAL_***}で以下のように指定する。\\

\editbox{
\verb|&PARAM_RESTART|\\
\verb| RESTART_OUTPUT       = .true.,|\\
\verb| RESTART_OUT_BASENAME = "init_d01",|\\
\verb|/|\\
\\
\verb|&PARAM_MKINIT_REAL_ATMOS|\\
\verb| NUMBER_OF_FILES      = 2,|\\
\verb| FILETYPE_ORG         = "GrADS",|\\
\verb| BASENAME_ORG         = "namelist.grads_boundary.FNL.grib1",|\\
\verb| BASENAME_BOUNDARY    = "boundary_d01",|\\
\verb| BOUNDARY_UPDATE_DT   = 21600.0,|\\
\verb| PARENT_MP_TYPE       = 3,|\\
\verb| USE_FILE_DENSITY     = .false.,|\\
\verb|/|\\
\\
\verb|&PARAM_MKINIT_REAL_OCEAN|\\
\verb| NUMBER_OF_FILES      = 2,|\\
\verb| FILETYPE_ORG         = "GrADS",|\\
\verb| BASENAME_ORG         = "namelist.grads_boundary.FNL.grib1",|\\
\verb| INTRP_OCEAN_SFC_TEMP = "mask",|\\
\verb| INTRP_OCEAN_TEMP     = "mask",|\\
\verb|/|\\
\\
\verb|&PARAM_MKINIT_REAL_LAND|\\
\verb| NUMBER_OF_FILES      = 2,|\\
\verb| FILETYPE_ORG         = "GrADS",|\\
\verb| BASENAME_ORG         = "namelist.grads_boundary.FNL.grib1",|\\
\verb| USE_FILE_LANDWATER   = .true.,|\\
\verb| INTRP_LAND_TEMP      = "mask",|\\
\verb| INTRP_LAND_WATER     = "fill",|\\
\verb| INTRP_LAND_SFC_TEMP  = "fill",|\\
\verb|/|\\
}


バイナリデータを読み込むときは、\nmitem{FILETYPE_ORG}に\verb|"GrADS"|を設定する。
\scalerm では ファイル名やバイナリデータの構造を含むネームリストファイル\verb|namelist.grads_boundary**|を「ctl」ファイルの代わりに準備する。

\nmitem{NUMBER_OF_FILES}は入力ファイルの数である。
単一の入力ファイルの場合は、ファイル「\verb|ファイル名.grd|」のみを準備する。
複数の入力ファイルの場合には、時間の進む方向に「\verb|ファイル名.XXXXX.grd|」と名前をつけたファイルを準備する。
プログラム\verb|scale-rm_init|は、\verb|00000|から\nmitem{NUMBER_OF_FILES}-1 までの数字を付けたファイルを読み込む。
入力ファイルのヘッダー名(つまり「\verb|ファイル名|」)はネームリストファイルで指定されるが、これについては後で説明する。

\nmitem{BOUNDARY_UPDATE_DT}は入力データの時間間隔である。
変換された初期値ファイルのヘッダー名は、\namelist{PARAM_RESTART}の\nmitem{RESTART_OUT_BASENAME}で設定する。
\nmitem{BASENAME_BOUNDARY}は、変換された境界値ファイルのヘッダー名である。
\nmitem{BASENAME_BOUNDARY}を指定しなければ、 境界値ファイルは出力されない。

以上の設定は、\namelist{PARAM_MKINIT_REAL_ATMOS}、
\namelist{PARAM_MKINIT_REAL_OCEAN}、\\
\namelist{PARAM_MKINIT_REAL_LAND}の間で共通である。
\namelist{PARAM_MKINIT_REAL_OCEAN}や \\
\namelist{PARAM_MKINIT_REAL_LAND}を別途指定しない限り、
これらの情報は引き継がれる。
%
\nmitem{USE_FILE_DENSITY}は\verb|FILETYPE_ORG="SCALE-RM"|とした場合のオプションであり、
バイナリデータを選択した場合は\verb|.false.|を与えなければならない。
\nmitem{PARENT_MP_TYPE}は親モデルの水物質カテゴリーの種類であるが、
バイナリデータを読み込む場合は\verb| 3 |を設定しなければならない。

土壌水分の設定は、親モデルからデータを与える方法と、
領域全体で一定値を与える方法の2種類ある。
前者の場合は、3次元の土壌水分データを必要とする。
後者の場合は、\verb|init.conf|の\namelist{PARAM_MKINIT_REAL_LAND}に
\verb|USE_FILE_LANDWATER = .false.| を設定する。
また、土壌水分の条件は、土壌空隙率に対する水が占める割合(飽和度)として
\verb|INIT_LANDWATER_RATIO| で指定する。デフォルト値は 0.5 である。
また、単位体積あたりの土壌の隙間の大きさ(空隙率)は土地利用に応じて変わる。\\
\editboxtwo{
\verb|&PARAM_MKINIT_REAL_LAND| &\\
\verb| USE_FILE_LANDWATER   = .false.| & 土壌水分をファイルから読むかどうか。デフォルトは\verb|.true.| \\
\verb| INIT_LANDWATER_RATIO = 0.5    | & \verb|USE_FILE_LANDWATER=.false.|の場合、 \\
                                       & 空隙率に対する水が占める割合(飽和度)。\\
\verb|  .....略.....                 | & \\
\verb|/| & \\
}

バイナリデータ({\grads}形式)を入力ファイルに用いる場合は、ユーザが用意する。
その形式については、\grads の Web ページ
(\url{http://cola.gmu.edu/grads/gadoc/aboutgriddeddata.html#structure})を参照されたい。
「ctl」ファイルの代わりに、データのファイル名やデータ構造を \scalerm に与えるための
ネームリストファイル(\verb|namelist.grads_boundary**|)の一例を下記に示す。\\

\editbox{
\verb|#| \\
\verb|# Dimension    |  \\
\verb|#|                \\
\verb|&nml_grads_grid|  \\
\verb| outer_nx     = 360,|~~~   ; 大気データのx方向の格子数 \\
\verb| outer_ny     = 181,|~~~   ; 大気データのy方向の格子数 \\
\verb| outer_nz     = 26, |~~~~~ ; 大気データのz方向の層数 \\
\verb| outer_nl     = 4,  |~~~~~~ ; 土壌データの層数 \\
\verb|/|                \\
\\
\verb|#              |  \\
\verb|# Variables    |  \\
\verb|#              |  \\
\verb|&grdvar  item='lon',     dtype='linear',  swpoint=0.0d0,   dd=1.0d0 /  |  \\
\verb|&grdvar  item='lat',     dtype='linear',  swpoint=90.0d0,  dd=-1.0d0 / |  \\
\verb|&grdvar  item='plev',    dtype='levels',  lnum=26,| \\
~~~\verb|      lvars=100000,97500,...(省略)...,2000,1000, /     |  \\
\verb|&grdvar  item='MSLP',    dtype='map',     fname='FNLsfc', startrec=1,  totalrec=6   / |  \\
\verb|&grdvar  item='PSFC',    dtype='map',     fname='FNLsfc', startrec=2,  totalrec=6   / |  \\
\verb|&grdvar  item='U10',     dtype='map',     fname='FNLsfc', startrec=3,  totalrec=6   / |  \\
\verb|&grdvar  item='V10',     dtype='map',     fname='FNLsfc', startrec=4,  totalrec=6   / |  \\
\verb|&grdvar  item='T2',      dtype='map',     fname='FNLsfc', startrec=5,  totalrec=6   / |  \\
\verb|&grdvar  item='RH2',     dtype='map',     fname='FNLsfc', startrec=6,  totalrec=6   / |  \\
\verb|&grdvar  item='HGT',     dtype='map',     fname='FNLatm', startrec=1,  totalrec=125 / |  \\
\verb|&grdvar  item='U',       dtype='map',     fname='FNLatm', startrec=27, totalrec=125 / |  \\
\verb|&grdvar  item='V',       dtype='map',     fname='FNLatm', startrec=53, totalrec=125 / |  \\
\verb|&grdvar  item='T',       dtype='map',     fname='FNLatm', startrec=79, totalrec=125 / |  \\
\verb|&grdvar  item='RH',      dtype='map',     fname='FNLatm', startrec=105,totalrec=125, knum=21 /  |  \\
\verb|&grdvar  item='llev',    dtype='levels',  lnum=4, lvars=0.05,0.25,0.70,1.50, /        |  \\
\verb|&grdvar  item='lsmask',  dtype='map',     fname='FNLland', startrec=1, totalrec=10 /  |  \\
\verb|&grdvar  item='SKINT',   dtype='map',     fname='FNLland', startrec=2, totalrec=10 /  |  \\
\verb|&grdvar  item='STEMP',   dtype='map',     fname='FNLland', startrec=3, totalrec=10,|\\
~~~~~~~~\verb| missval=9.999e+20 /|  \\
\verb|&grdvar  item='SMOISVC', dtype='map',     fname='FNLland', startrec=7, totalrec=10,|\\
~~~~~~~~\verb| missval=9.999e+20 /|  \\
}

大気データの格子数を\verb|outer_nx, outer_ny, outer_nz|で指定し、
土壌データ(STEMP、SMOISVC) の層数を\verb|outer_nl|で指定する。

QVやRHのデータは大気上層でしばしば提供されていない。
そのような場合はデータが存在する層数を\verb|knum|で指定する。
上層での値の与え方として2種類の方法を用意している。
デフォルトでは、以下のように\verb| upper_qv_type = "ZERO"|である。\\
\editboxtwo{
\verb|&PARAM_MKINIT_REAL_GrADS| & \\
\verb| upper_qv_type = "ZERO"| & \verb|"ZERO"|: QV=0 \\
                               & \verb|"COPY"|: 湿度の入力データが存在する最上層のRHを、データが存在しない上層にコピーする\\
\verb|/|\\
}

\namelist{grdvar}の設定は、表\ref{tab:namelist_grdvar}に示すようにデータによって異なる。
表\ref{tab:grdvar_item}において、体積含水率は土の体積$V$の中に占める水の体積$V_w$の割合($V_w / V$)である。
また、飽和度は$V$の中に占める間隙の体積$V_v$に対する水の体積$V_w$の割合($V_w / V_v$)である。
\namelist{PARAM_MKINIT_REAL_LAND}の\nmitem{USE_FILE_LANDWATER}が\verb|.true.|である場合は、
\verb|SMOISVC|か\verb|SMOISDS|のどちらかを用意する必要がある。

{\small
\begin{table}[htb]
\begin{center}
\caption{\namelist{grdvar}の変数}
\label{tab:namelist_grdvar}
\begin{tabularx}{150mm}{llX} \hline
\rowcolor[gray]{0.9} \verb|grdvar|の項目  & 説明 & 備考 \\ \hline
\multicolumn{1}{l}{item}    & \multicolumn{1}{l}{変数名} & 表\ref{tab:grdvar_item}より選択      \\
\multicolumn{1}{l}{dtype}   & \multicolumn{1}{l}{データ形式} & \verb|"linear", "levels", "map"|から選択 \\\hline
\multicolumn{3}{l}{\nmitem{dtype}が\verb|"linear"|の場合のネームリスト (\verb|"lon", "lat"|専用)} \\ \hline
\multicolumn{1}{l}{swpoint}  & \multicolumn{1}{l}{スタートポイントの値} &  \\
\multicolumn{1}{l}{dd}       & \multicolumn{1}{l}{増分}                 &  \\ \hline
\multicolumn{3}{l}{\nmitem{dtype}が\verb|"levels"|の場合のネームリスト (\verb|"plev", "llev"|専用)} \\ \hline
\multicolumn{1}{l}{lnum}     & \multicolumn{1}{l}{レベルの数(層数)}     &  \\
\multicolumn{1}{l}{lvars}    & \multicolumn{1}{l}{各層の値}             &  \\ \hline
\multicolumn{3}{l}{\nmitem{dtype}が\verb|"map"|の場合のネームリスト}           \\ \hline
\multicolumn{1}{l}{fname}    & \multicolumn{1}{l}{ファイル名の頭}       &  \\
\multicolumn{1}{l}{startrec} & \multicolumn{1}{l}{変数\nmitem{item}のレコード番号} &  \multicolumn{1}{l}{t=1 の時刻の値}\\
\multicolumn{1}{l}{totalrec} & \multicolumn{1}{l}{一時刻あたりの全変数のレコード長}  &  \\
\multicolumn{1}{l}{knum}     & \multicolumn{1}{l}{3次元データの層数} & \multicolumn{1}{l}{(オプション) \verb|outer_nz|と異なる場合。}\\
                             &                                  & \multicolumn{1}{l}{~~~~~~~~~~ RHとQVのみ使用可。}\\
\multicolumn{1}{l}{missval}  & \multicolumn{1}{l}{欠陥値の値}        & \multicolumn{1}{l}{(オプション)}\\ \hline
\end{tabularx}
\end{center}
\end{table}
}

{\small
\begin{table}[hbt]
\begin{center}
\caption{\namelist{grdvar}の\nmitem{item}の変数リスト。
アスタリスクは「オプションであるが、可能な限り推奨される」ことを意味する。
二重のアスタリスクは、「利用できるが、推奨されない」ことを意味する。
高高度の場所を対象とする場合は、\texttt{HGT}を用いることを強く推奨する。
}
\label{tab:grdvar_item}
\begin{tabularx}{150mm}{rl|l|l|l} \hline
 \rowcolor[gray]{0.9} & 変数名 & 説明 & 単位 & \nmitem{dtype} \\ \hline
             &\verb|lon|     & 経度データ                 & [deg.]   & \verb|linear, map| \\
             &\verb|lat|     & 緯度データ                 & [deg.]   & \verb|linear, map| \\
             &\verb|plev|    & 気圧データ                 & [Pa]     & \verb|levels, map| \\
      $\ast$ &\verb|HGT|     & 高度(ジオポテンシャル)データ & [m]      & \verb|map| \\
$\ast$$\ast$ &\verb|DENS|    & air density               & [kg/m3]        & \verb|map|         \\
             &\verb|U|       & 東西風速                   & [m/s]    & \verb|map| \\
             &\verb|V|       & 南北風速                   & [m/s]    & \verb|map| \\
$\ast$$\ast$ &\verb|W|       & 鉛直風速                   & [m/s]    & \verb|map| \\
             &\verb|T|       & 気温                      & [K]       & \verb|map| \\
             &\verb|RH|      & 相対湿度 (QVがある場合は省略可) & [\%]    & \verb|map| \\
             &\verb|QV|      & 比湿 (RH がある場合は省略可)   & [kg/kg] & \verb|map| \\
$\ast$$\ast$ &\verb|QC|      & 雲水の質量比    & [kg/kg] & \verb|map| \\
$\ast$$\ast$ &\verb|QR|      & 雨水の質量比    & [kg/kg] & \verb|map| \\
$\ast$$\ast$ &\verb|QI|      & 雲氷の質量比    & [kg/kg] & \verb|map| \\
$\ast$$\ast$ &\verb|QS|      & 雪の質量比      & [kg/kg] & \verb|map| \\
$\ast$$\ast$ &\verb|QG|      & 霰の質量比      & [kg/kg] & \verb|map| \\
      $\ast$ &\verb|MSLP|    & 海面更正気圧     & [Pa]     & \verb|map| \\
      $\ast$ &\verb|PSFC|    & 地上気圧        & [Pa]     & \verb|map| \\
      $\ast$ &\verb|U10|     & 10m 東西風速    & [m/s]    & \verb|map| \\
      $\ast$ &\verb|V10|     & 10m 南北風速    & [m/s]    & \verb|map| \\
      $\ast$ &\verb|T2|      & 2m 気温         & [K]      & \verb|map| \\
      $\ast$ &\verb|RH2|     & 2m 相対湿度 (Q2がある場合は省略可) & [\%]  & \verb|map| \\
      $\ast$ &\verb|Q2|      & 2m 比湿 (RH2がある場合は省略可)   &[kg/kg] & \verb|map| \\
      $\ast$ &\verb|TOPO|    & GCMの地形                      & [m]      & \verb|map| \\
      $\ast$ &\verb|lsmask|  & GCMの海陸分布                   & 0:海1:陸 & \verb|map| \\
             &\verb|SKINT|   & 地表面温度                      & [K]      & \verb|map| \\
             &\verb|llev|    & 土壌の深さ                      & [m]      & \verb|levels| \\
             &\verb|STEMP|   & 土壌温度                        & [K]      & \verb|map| \\
             &\verb|SMOISVC| & 土壌水分(体積含水率)             & [-] & \verb|map| \\
             &               & (SMOISDS がある場合は省略可)     &                &                    \\
             &\verb|SMOISDS| & 土壌水分(飽和度)                & [-] & \verb|map| \\
             &               & (SMOISVC がある場合は省略可)     &                &                    \\
             &\verb|SST|     & 海面温度(SKINTがある場合は省略可) & [K] & \verb|map|\\ \hline
\end{tabularx}
\end{center}
\end{table}
}
