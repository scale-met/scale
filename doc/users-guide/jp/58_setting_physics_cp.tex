\section{積雲パラメタリゼーション} \label{sec:basic_usel_cumulus}

積雲パラメタリゼーションは、設定ファイル\verb|init.conf|と\verb|run.conf|中の
\namelist{PARAM_ATMOS}の\nmitem{ATMOS_PHY_CP_TYPE}で指定する。
積雲パラメタリゼーションを呼び出す時間間隔は、\namelist{PARAM_TIME}で設定する
(詳細は第\ref{sec:timeintiv}節を参照)。

\editboxtwo{
\verb|&PARAM_ATMOS  | & \\
\verb| ATMOS_PHY_CP_TYPE = "KF", | & ; 表\ref{tab:nml_atm_cp}に示すスキームから選択 \\
\verb|/             | & \\
}
\begin{table}[h]
\begin{center}
  \caption{積雲パラメタリゼーションの選択肢}
  \label{tab:nml_atm_cp}
  \begin{tabularx}{150mm}{lXX} \hline
    \rowcolor[gray]{0.9}  スキーム名 & スキームの説明 & 参考文献 \\ \hline
      \verb|OFF|  & 積雲パラメタリゼーションを使用しない &  \\
      \verb|KF|   & Kain-Fritsch 対流パラメタリゼーション & \citet{kain_1990,kain_2004} \\
    \hline
  \end{tabularx}
\end{center}
\end{table}

\scalerm の現版では、積雲パラメタリゼーションとして\verb|KF|のみ対応している。
\verb|KF| は質量フラックス保存型の積雲パラメタリゼーションスキームであり、
サブグリッドスケールの一つの積雲を表現する。
格子間隔が 5 km 以下の場合に、非自然的な強力な深い対流が計算されることを避けるために、
この積雲パラメタリゼーションを使用することを推奨する。
積雲パラメタリゼーションと雲微物理のスキームは、
\verb|RAIN_CP|\verb|RAIN_MP|という名前で別々に降水量を出力する。
\verb|RAIN|と\verb|PREC|は、両者のスキームによる合計の降水量である。
つまり、\verb|RAIN| = \verb|RAIN_CP| + \verb|RAIN_MP|、
\verb|PREC| = \verb|RAIN_CP| + \verb|RAIN_MP|である。
%%%
\verb|KF|は大気中の水蒸気と水物質(雲水・雲氷等)の変化を計算することに注意が必要である。
水物質の変化は、雲微物理の過程でさらに計算される。
\verb|KF|では、雲水や雲氷等の数密度は考慮されない。
したがって、\verb|KF|における水物質の変化と関係した数密度の変化は、指定した関数によって見積もられ、2モーメントの雲微物理スキームへと渡される。

\subsubsection{\texttt{Kain-Fritsch}スキームに対する設定}

\verb|KF|では、以下のチューニングパラメータを設定できる。
\editboxtwo{
\verb|&PARAM_ATMOS_PHY_CP_KF  | & \\
\verb| PARAM_ATMOS_PHY_CP_kf_trigger   = 1,|     & ; Trigger function の種類: 1=Kain, 3=Narita-Ohmori\\
\verb| PARAM_ATMOS_PHY_CP_kf_dlcape    = 0.1,|   & ; Cape decleace rate\\
\verb| PARAM_ATMOS_PHY_CP_kf_dlifetime = 1800,|  & ; 深い対流の生存時間のスケール[sec]\\
\verb| PARAM_ATMOS_PHY_CP_kf_slifetime = 2400,|  & ; 浅い対流の生存時間のスケール[sec]\\
\verb| PARAM_ATMOS_PHY_CP_kf_DEPTH_USL =  300,|  & ; updraft source layer の深さ[hPa]\\
\verb| PARAM_ATMOS_PHY_CP_kf_prec      = 1,|     & ; 降水の種類: 1=Ogura-Cho, 2=Kessler\\
\verb| PARAM_ATMOS_PHY_CP_kf_rate      = 0.03, | & ; Ogura-Cho precipitation function における雲水と降水の比 \\
\verb| PARAM_ATMOS_PHY_CP_kf_thres     = 1.E-3,| & ; Kessler の precipitation function における Autoconversion の比 \\
\verb| PARAM_ATMOS_PHY_CP_kf_LOG       = false,| & ; 警告メッセージを出力するか? \\
\verb|/             | & \\
}\\
ユーザーはtrigger fuctionとして以下の2つから選択できる。
\begin{enumerate}
\item Kain Type \citet{kain_2004} \\
  \scalerm におけるデフォルト。
\item Narita and Ohmori Type \citet{narita_2007} \\
  日本域における KF 積雲パラメタリゼーションスキームに対する trigger function
\end{enumerate}
また、 precipitation fuctionとして以下の2つから選択できる。
\begin{enumerate}
\item Ogura-Cho Type \citet{ogura_1973} \\
  \scalerm におけるデフォルト。この場合には、
  \nmitem{PARAM_ATMOS_PHY_CP_kf_rate}というチューニングパラメータをさらに設定できる。
\item Kessler Type \citet{kessler_1969} \\
  Kessler type の簡単な precipitation fuction.
  この場合には、 \nmitem{PARAM_ATMOS_PHY_CP_kf_thres}というチューニングパラメータをさらに設定できる。
\end{enumerate}

\namelist{PARAM_TIME}内の\nmitem{TIME_DT_ATMOS_PHY_CP}で指定する、
KF を呼び出す時間間隔もまたチューニングパラメータであり、降水量に影響を及す。
\nmitem{TIME_DT_ATMOS_PHY_CP}の初めの設定として 300 秒を推奨する。
上昇流を駆動する層はしきい値(デフォルトでは 50 hPa)よりも厚い必要があるが、
この条件を満たさない場合でも計算は止まらない。
ただし、\nmitem{PALAM_ATMOS_PHY_CP_kf_LOG}を\verb|true|にすれば、
「go off top/bottom of updraft source layer」という警告メッセージを出力する。
