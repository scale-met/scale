\section{ユーザが設定を行うためのモジュール} \label{sec:mod_user}

\scalerm では、計算に対するユーザの要求を満たすために、数多くのオプションを用意してある。
これらは、ネームリストのパラメータによって指定することができる。
しかし、期待するオプションが存在しない場合には、ユーザ用のモジュール(\verb|mod_user|)にプログラムを記述することによって、ユーザが望むようにモデル変数を直接書き換えることができる。
本節では、\verb|mod_user|とは何であるかを説明し、その使い方を記述する。


\subsection{\texttt{mod\_user} モジュールとは?}

デフォルトの\verb|mod_user| モジュールは、 \texttt{scale-{\version}/scale/scale-rm/src/user/mod\_user.F90}に用意されている。
適宜\verb|mod_user.F90|を書き換え、このファイルをデフォルトのファイルに代わってコンパイルする。

\verb|mod_user| モジュールには、以下のサブルーチンを含まなければならない。
\begin{alltt}
  subroutine USER_tracer_setup
  subroutine USER_setup
  subroutine USER_mkinit
  subroutine USER_update
  subroutine USER_calc_tendency
\end{alltt}

\noindent 以下は、\scalerm における各プロセスの実行順序である。
\begin{alltt}
初期設定
  IOの設定
  MPIの設定
  格子の設定
  力学や物理スキームの管理モジュールの設定
  トレーサーの設定
  \textcolor{blue}{USER_tracer_setup}
  地形、陸面の設定
  力学や物理スキームの変数やドライバーの設定
  \textcolor{blue}{USER_setup}
メインルーチン
  時間進展
  海洋/陸面/都市/大気モデルの更新
  \textcolor{blue}{User_update}
  リスタートファイルの出力
  海洋/陸面/都市/大気モデルにおける時間変化率の計算
  \textcolor{blue}{USER_calc_tendency}
  ヒストリファイルの出力
\end{alltt}
\verb|mod_user|の各サブルーチンを呼び出すタイミングを、青色で示している。
\verb|USER_mkinit|は、 初期値作成プログラム\verb|scale-rm_init|で呼び出される。


\verb|mod_user|のサブルーチンは基本的には各プロセスを処理した後に呼び出されるので、
設定や変数を思うように置き換えることができる。
また\verb|USER_tracer_setup|において、パッシブトレーサー等のトレーサをいくつか加えることもできる。
%When you change settings defined in setup process, use \verb|USER_setup|.
%\textcolor{blue}{User update}
\verb|mod_user.F90|の例として、
各テストケース(\texttt{scale-{\version}/scale-rm/test/case}以下)に含まれる
\verb|mod_user.F90|が参考になるだろう。


\subsection{コンパイル}
テストケース用の Makefile を活用することによって、
ユーザが作成した\verb|mod_user.F90|と共に{\scalerm}をコンパイルできる。
その手順の例は以下である。\\
\texttt{ \$ cd scale-\version/scale-rm/test/case}\\
\texttt{ \$ mkdir -p your\_dir/exp\_name}\\
\texttt{ \$ cd your\_dir/exp\_name}\\
\texttt{ \$ cp ../../advection/500m/Makefile .}\\
\noindent ユーザが作成した\verb|mod_user.F90|を本ディレクトリにコピー \\
\texttt{ \$ make}
