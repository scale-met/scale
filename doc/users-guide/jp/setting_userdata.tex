\section{ユーザ定義データの作成方法} \label{sec:userdata}

いくつかのスキームでは、初期値/境界値以外の入力データを必要とする。
そのようなスキームを使用する場合、そのデータはシミュレーション実行の前に準備する必要がある。
入力データは \scalenetcdf ファイルである必要があり、また、実験設定に応じた格子データである必要がある。
そのような入力データを用意するために、\scalerm にはユーザ定義データを変換する方法が用意されている。
データ変換は、\namelist{PARAM_CONVERT} の \nmitem{CNVERT_USER} を \verb|.true.| と設定することで、\verb|scale-rm_init| で変換が行われる。
サポートされているユーザ定義データの種類が表\ref{tab:userdata_type}に記載されている。
一度の実行で一つの変数しか変換されないため、変換が必要なユーザ定義変数の数だけ実行を繰り返す必要がある。
現時点では、空間方向には2次元データのみ変換が可能である。


\begin{table}[tbh]
\begin{center}
\caption{\scalerm でサポートされているユーザ定義データの種類}
\begin{tabularx}{150mm}{l|X} \hline
 \rowcolor[gray]{0.9} データ種類 & 説明 \\ \hline
 \verb|GrADS| & \grads 形式バイナリデータ    \\ \hline
 \verb|TILE|  & 水平方向に分割されたタイルデータ \\ \hline
\end{tabularx}
\label{tab:userdata_type}
\end{center}
\end{table}


ユーザ定義データに関する設定は、以下のように \namelist{PARAM_CNVUSER} で行う。
\editboxtwo{
\verb|&PARAM_CNVUSER| & \\
\verb| CNVUSER_FILE_TYPE = '',        | & ; ユーザ定義データの種類、 表\ref{tab:userdata_type}参照 \\
\verb| CNVUSER_INTERP_TYPE = 'LINEAR',| & ; 水平内挿の種類 \\
\verb| CNVUSER_INTERP_LEVEL = 5,      | & ; 内挿のレベル   \\
\verb| CNVUSER_OUT_TITLE = 'SCALE-RM 2D Boundary',| & ; 出力ファイル中のタイトル \\
\verb| CNVUSER_OUT_VARNAME = '',      | & ; 出力ファイルにおける変数名 \\
\verb| CNVUSER_OUT_VARDESC = '',      | & ; 出力変数の説明 \\
\verb| CNVUSER_OUT_VARUNIT = '',      | & ; 出力変数の単位 \\
\verb| CNVUSER_OUT_DTYPE 'DEFAULT',   | & ; 出力変数のデータタイプ (DEFAULT,INT2,INT4,REAL4,REAL8) \\
\verb| CNVUSER_NSTEPS =  10,          | & ; データの時間ステップ数 \\
\verb| CNVUSER_OUT_DT = -1.0D0,       | & ; 出力データの時間間隔 [秒] \\
\verb| CNVUSER_GrADS_FILENAME = '',   | & ; \grads データ用のネームリストファイル名 \\
\verb| CNVUSER_GrADS_VARNAME = '',    | & ; ネームリスト中の変数名 \\
\verb| CNVUSER_GrADS_LATNAME = 'lat', | & ; ネームリスト中の緯度の名前 \\
\verb| CNVUSER_GrADS_LONNAME = 'lon', | & ; ネームリスト中の経度の名前 \\
\verb| CNVUSER_TILE_DIR = '',         | & ; タイルデータとカタログファイルがあるディレクトリパス \\
\verb| CNVUSER_TILE_CATALOGUE = '',   | & ; カタログファイルの名前 \\
\verb| CNVUSER_TILE_DLAT = 1.0D0,     | & ; タイルデータの緯度間隔 \\
\verb| CNVUSER_TILE_DLON = 1.0D0,     | & ; タイルデータの経度間隔 \\
\verb| CNVUSER_TILE_DTYPE = 'REAL4',  | & ; タイルデータのデータタイプ (INT2,INT4,REAL4,REAL8) \\
\verb|/|\\
}

空間内挿の種類およびそのレベルをそれぞれ \nmitem{CNVUSER_INTERP_TYPE} および \nmitem{CNVUSER_INTERP_LEVEL} で指定する。
空間内挿についての詳細は \ref{sec:datainput_grads}節参照のこと。

出力ファイル中のタイトル、変数名、変数の説明、変数の単位は、\nmitem{CNVUSE_OUT_TITLE}, \\
\nmitem{CNVUSE_OUT_VARNAME}, \nmitem{CNVUSE_OUT_VARDESC}, \nmitem{CNVUSE_OUT_VARUNIT} でそれぞれ設定する。
ユーザ定義データの種類が \verb|GrADS| の場合は, \nmitem{CNVUSE_OUT_VARNAME} が設定されていなければ \nmitem{CNVUSER_GrADS_VARNAME} が出力変数名として用いられる。

\nmitem{CNVUSER_OUT_DTYPE} が \verb|DEFAULT|の場合, シミュレーションにおけるデータタイプが使われる。
シミュレーションのデータタイプのデフォルト値は \verb|REAL8| であり、\scalerm が \\
\verb|SCALE_USE_SINGLEFP = "T"| としてコンパイルされた場合 (\ref{sec:compile}章参照) は \verb|REAL4| となる。


ユーザ定義データの時間ステップ数は \nmitem{CNVUSER_NSTEPS} で指定する。
出力ファイル中の時刻座標の初期値は \namelist{PARAM_TIME} の \nmitem{TIME_STARTDATE} (\ref{sec:timeintiv}章参照) で指定し、時間間隔は \nmitem{CNVUSER_OUT_DT} で秒単位で指定する。


\verb|GrADS| タイプでは、入力のデータ構造を記述したネームリストファイルが必要である。
ネームリストファイルは \nmitem{USERFILE_GrADS_FILENAME} で指定する。
ネームリストの詳細については \ref{sec:datainput_grads} 章を参照のこと。
デフォルトでは、緯度および経度の名前はそれぞれ \verb|lat| および \verb|lon| である。
デフォルト値と異なる場合は、それぞれ \nmitem{CNVUSER_GrADS_LATNAME} および \nmitem{CNVUSER_GrADS_LONNAME} で指定する。
\verb|GrADS| タイプのユーザ定義データの変換を行う2つのサンプルが \\
\verb|scale/scale-rm/test/framework/cnvuser| に用意されている。
以下はネームリストファイルの例である。
\editbox{
\verb|#| \\
\verb|# Dimension| \\
\verb|#| \\
\verb|&GrADS_DIMS| \\
\verb| nx = 361,| \\
\verb| ny = 181,| \\
\verb| nz =   1,| \\
\verb|/| \\
\verb|#| \\
\verb|# Variables | \\
\verb|#| \\
\verb|&GrADS_ITEM name='lon', dtype='linear', swpoint=0.0D0, dd=1.0D0 /| \\
\verb|&GrADS_ITEM name='lat', dtype='linear', swpoint=-90.0D0, dd=1.0D0 /| \\
\verb|&GrADS_ITEM name='var', dtype='map', fname='fname_in', startrec=1, totalrec=1,| \textbackslash \ \\
~~~~~~~~~~~~~ \verb|bintype='int1', yrev=.true. /| \\
}



\verb|TILE| タイプは、空間方向にタイル状に分割されたユーザ定義データを読み込む。
個々のタイルデータは ダイレクトアクセスのバイナリデータでなければならない。
バイナリデータの種類は \nmitem{CNVUSER_TILE_DTYPE} で指定する。
データは、等間隔の緯度経度格子である必要があるり、それぞれの間隔は、\nmitem{CNVUSER_TILE_DLAT} および \nmitem{CNVUSER_TILE_DLON} で指定する。
個々のタイルデータファイルの名前やカバーする領域の情報を記述したカタログファイルが必要となる。
カタログファイルの名前は \nmitem{CNVUSER_TILE_CATALOGUE} で指定する。
カタログファイルおよびタイルデータファイルは \nmitem{CNVUSER_TILE_DIR} で指定するディレクトリに配置する。
カタログファイルの各行には、通し番号、最南端、最北端の緯度、最西端、最東端の経度、タイルデータの名前が記載されている。
カタログファイルについては \verb|$SCALE_DB/topo/DEM50M/Products/DEM50M_catalogue.txt| および \verb|$SCALE_DB/topo/GTOPO30/Products/GTOPO30_catalogue.txt| が参考になる。
以下は、全球のデータが南北それぞれ 2x2 の 4つに分割されたユーザ定義データのためのカタログファイルの例である。
\editboxtwo{
\verb|001 -90.0  0.0 -180.0   0.0 TILE_sw.grd| & ; latitude: -90--0, longitude: -180--0, name: \verb|TILE_sw.grd| \\
\verb|002 -90.0  0.0    0.0 180.0 TILE_se.grd| & ; latitude: -90--0, longitude: 0--180,  name: \verb|TILE_se.grd| \\
\verb|003   0.0 90.0 -180.0   0.0 TILE_nw.grd| & ; latitude: 0--90,  longitude: -180--0, name: \verb|TILE_nw.grd| \\
\verb|004   0.0 00.0    0.0 180.0 TILE_ne.grd| & ; latitude: 0--90,  longitude: 0--180,  name: \verb|TILE_ne.grd| \\
}

