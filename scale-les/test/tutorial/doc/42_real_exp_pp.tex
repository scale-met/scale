
%-------------------------------------------------------%
\section{地形・土地利用データの作成:pp}
%-------------------------------------------------------%

ppディレクトリへ移動し、現実実験のための地形データ、土地利用データを作成する。
\begin{verbatim}
 $ cd ${Tutrial_DIR}/pp
\end{verbatim}
ppディレクトリの中には,\verb|pp.conf|という名前の
コンフィグファイルが準備されている.
ドメインの位置や格子点数など、実験設定に合わせて,
適宜\verb|pp.conf|を編集する必要があるが,
チュートリアルでは,すでに編集済みの\verb|pp.conf|が
与えられているためそのまま利用する.
\verb|pp.conf|の設定の中で特に注意するべき項目は,\verb|PARAM_CONVERT|である.
\begin{verbatim}
 &PARAM_CONVERT
  CONVERT_TOPO = .true.,
  CONVERT_LANDUSE = .true.,
 /
\end{verbatim}
上記のように\verb|CONVERT_TOPO|と\verb|CONVERT_LANDUSE|が
\verb|.true.|となっていることが,
それぞれ地形と土地利用の処理を行うことを意味している.
詳細なコンフィグファイルの内容については,
Appendix \ref{app:namelist}を参照されたい.

次に,コンパイル済みのバイナリと入力データをppディレクトリへリンクする.
\begin{verbatim}
 $ ln -s ${TOPDIR}/scale/scale-les/test/tutorial/bin/scale-les_pp ./
 $ ln -s ${SCALE_DB}/topo    ./
 $ ln -s ${SCALE_DB}/landuse ./
\end{verbatim}
今回は,Table \ref{tab:grids}に示されているように,
4つのMPIプロセスを使用する設定なので次のように実行する.
\begin{verbatim}
 $ mpirun -n 4 ./scale-les_pp pp.conf
\end{verbatim}
ジョブが正常に終了すれば,\verb|topo_d01.pe######.nc|と\verb|landuse_d01.pe######.nc|というファイルがMPIプロセス数だけ,つまり4つずつ生成される(\verb|######|にはMPIプロセスの番号が入る).
それぞれ,ドメインの格子点に内挿された地形と土地利用の情報が入ってる.


実行時のログは、\verb|pp_LOG_d01.pe000000|に出力されるので
内容を確かめておくこと.
gpviewがインストールされていれば,次のコマンドによって
作成された地形と土地利用データを描画してチェックすることができる.
正しく作成されていれば,Fig. \ref{fig:domain}と同じように描かれる.
\begin{verbatim}
  $ gpview topo_d01.pe00000*@TOPO --aspect=1
  $ gpview landuse_d01.pe00000*@FRAC_LAND --aspect=1
\end{verbatim}

