
%-------------------------------------------------------%
\section{初期値・境界値データの作成:init}
%-------------------------------------------------------%

init では,SCALE計算に必要な初期値・境界値データを作成する.
まず、initディレクトリへ移動する.
\begin{verbatim}
 $ cd ${Tutrial_DIR}/real/init
\end{verbatim}

initディレクトリの中には,\verb|init.conf|という名前のコンフィグファイルが準備されている.
\verb|pp.conf|と同様に,実験設定に合わせて、この\verb|init.conf|を書き換える必要があるが、
チュートリアル用の\verb|init.conf|ファイルはTable\ref{tab:grids}の設定に
すでに合わせてある.
初期値・境界値データの作成には前節で作成した地形・土地利用データを利用する.
これは,下記のように,相対PATHを用いて参照するように設定されている.

\begin{verbatim}
  &PARAM_TOPO
   TOPO_IN_BASENAME = "../pp/topo_d01",
  /
  &PARAM_LANDUSE
   LANDUSE_IN_BASENAME  = "../pp/landuse_d01",
  /
\end{verbatim}
その他に\verb|init.conf|の設定の中で特に注意するべき項目は,
\verb|PARAM_MKINIT_REAL|である.

\begin{verbatim}
  &PARAM_MKINIT_REAL
   BASENAME_BOUNDARY   = "boundary_d01",  <- 境界値データの出力名
   FILETYPE_ORG        = "GrADS",
   NUMBER_OF_FILES     = 3,               <- 読み込むファイルの数
   BOUNDARY_UPDATE_DT  = 21600.D0,        <- 入力データの時間間隔
   INTERP_SERC_DIV_NUM = 20,              <- 内挿計算用のチューニングパラメータ
  /
\end{verbatim}

\verb|FILETYPE_ORG|は入力する気象場データのファイルフォーマットに
関するパラメータを設定しており,ここでは
grads形式データのフォーマットで読み込むことを指定している.
詳細なコンフィグファイルの内容については,Appendix \ref{app:namelist}を参照されたい.

次に,コンパイル済みのバイナリをinitディレクトリへリンクする.
\begin{verbatim}
  $ ln -s ../../bin/scale-les_init ./
\end{verbatim}
入力データは、initディレクトリの中に準備されている\verb|"gradsinput-link_FNL.sh"|を用いてリンクをはる.
\begin{verbatim}
  $ sh gradsinput-link_FNL.sh
\end{verbatim}
下記ファイルにリンクが張れれば成功.
{\small
\begin{verbatim}
  FNLatm_00000.grd
  FNLatm_00001.grd
  FNLatm_00002.grd
  FNLatm_00003.grd
  FNLland_00000.grd
  FNLland_00001.grd
  FNLland_00002.grd
  FNLland_00003.grd
  FNLsfc_00000.grd
  FNLsfc_00001.grd
  FNLsfc_00002.grd
  FNLsfc_00003.grd
\end{verbatim} }

次に、陸面の変数を用意するのに必要なパラメータファイルにリンクをはる.
\begin{verbatim}
 $ ln -s ../../../data/land/* ./   <- 陸面スキーム用のパラメータファイル
\end{verbatim}
準備が整ったら,4つのMPIプロセスを使用してinitを実行する.
\begin{verbatim}
 $ mpirun -n 4 ./scale-les_init init.conf
\end{verbatim}

正常にジョブが終了すれば,
\verb|boundary_d01.pe######.nc|と\verb|init_d01_00010713600.000.pe######.nc|というファイルが
MPIプロセス数だけ,つまり4つずつ生成される(\verb|######|にはMPIプロセスの番号が入る).
それぞれ,境界値データと初期値データが入ってるおり,境界値データには複数の時刻のデータが1つのファイルに含まれている.
初期値ファイルの名前のうち\verb|"00019094400.000"|の部分は,モデル内で算出された実験開始時刻を表している.
処理内容のログとして,\verb|init_LOG_d01.pe000000|という名前でログファイルも出力されるので内容を確かめておくこと.


\vspace{1cm}
\noindent {\Large\em OPTION}\\
gpviewがインストールされていれば,次のコマンドによって
作成された地形と土地利用データが、正しく作成されているかどうか、
確認することが出来る.正しく作成されていれば、
Fig. \ref{fig:init}と同じように描かれる.

\begin{verbatim}
$ gpvect --scalar --slice z=1500 --nocont --aspect=1 --range=0.001:0.015          \
         --xintv=10 --yintv=10 --unit_vect init_d01_00010713600.000.pe00*@QV      \
         init_d01_00010713600.000.pe00*@MOMX init_d01_00010713600.000.pe00*@MOMY
\end{verbatim}


\begin{figure}[h]
\begin{center}
  \includegraphics[width=0.7\hsize]{./figure/init_qv-momxy.eps}\\
  \caption{チュートリアル実験の初期場の様子:カラーシェードは高度1.5kmにおける比湿の分布,ベクトルは高度1.5kmにおける水平運動量フラックスを表している.}
  \label{fig:init}
\end{center}
\end{figure}

