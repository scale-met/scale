\section{ドメインネスティング実験と一括実行}

ここではSCALEのネスティング実験の方法について説明する。ネスティング実験は、広領域かつ高解像度の計算領域が必要な現象を
取り扱う場合に利用される計算領域設定方法である。図\ref{fig_nestsample}に示すように、ネスティング実験では水平格子間隔の異なる
複数の計算領域(ドメイン)を設定し、領域が一部重複するように入れ子(ネスト)構造にする。外側のドメインは比較的粗い
水平解像度であるが広い領域を取ることで大きな場の構造を表現することができる。逆に内側のドメインは、比較的狭い領域であるが
細かい水平解像度を取ることで対象とする現象の細かい構造を表現することができる。
入れ子構造のうち、データを渡す側のドメインを「親ドメイン」、データを受ける側のドメインを「子ドメイン」と称する。

SCALEはオフライン・ネスティング実験とオンライン・ネスティング実験の両方をサポートしている。オフライン・ネスティング実験は、
はじめに親ドメインだけで時間積分を行い、その計算結果のhistoryデータを用いて、子ドメイン用の初期値・境界値を作成する。
その後に子ドメインの時間積分を行う。オンライン・ネスティング実験は、親ドメインと子ドメインを同時に実行し、適宜計算途中の
データを親ドメインから子ドメインへMPI通信によって受け渡しすることで、子ドメインの時間積分を行う。
通常はオンライン・ネスティング実験が実行できるだけの計算機リソースがあれば、オンラインで実行することを推奨する。
それは、オンライン・ネスティング実験の場合、子ドメインの境界条件の更新間隔は親ドメインの時間積分間隔に一致するため、
可能な限りで最大限の更新間隔を得ることができる。また、中間ファイル等も作成されずディスクリソースにも優しい。
以降で、まずは実行方法がわかりやすいオフライン・ネスティングから説明し、ついでオンライン・ネスティングの実行方法に
ついて説明する。

また、すこしネスティングとは異なるが、同様のシステムを用いて実行できる機能として「一括実行機能」、いわゆるバルクジョブ機能
について説明する。これは、パラメタスイープ実験、初期値アンサンブル実験や、Time Slice気候実験など多数の実験を行う場合に
便利な機能である。


\begin{figure}[t]
\begin{center}
  \includegraphics[width=1.0\hsize]{./figure/nesting_sample.eps}\\
  \caption{日本の近畿地方を対象領域としたドメインネスティング設定の例. domain 1が最外ドメインでdomain 3が最内ドメインである。
           赤い矩形と線は、それぞれの位置関係を示している。domain 1の水平格子間隔は7.5 km、domain 2は2.5 km、
           そしてdomain 3は0.5 kmである。}
  \label{fig_nestsample}
\end{center}
\end{figure}

\subsection{オフライン・ネスティング}

SCALE-LESモデルは単一の計算領域(Domain)を使用した計算の他に,
複数のDomainを使用したもつDomain nesting計算をサポートしている.
広い領域に渡る環境場の再現と高解像度計算による詳細な構造と過程を再現したい場合に有用な機能である.
ここでは,広い領域をとった比較的粗い解像度のDomainをParent domainと称し,
相対的に狭い領域ではあるが高解像度のDomainをChild domainと称する.
Child domainの領域はParent domainの領域内に完全に包含されていなければならない.



\subsection{オンライン・ネスティング}




現在Online Nestingにおいては,1-way nesting(Parent domainからChild domainへのデータ受け渡し)のみをサポートしている.
Nesting実験を行う場合は,Domainの数だけ各コンフィグファイル(\verb|pp_d##.conf,init_d##.conf,run_d##.conf|)を
用意する必要がある.そして,各Domainについて地形・土地利用の作成,初期値・境界値の作成を事前に行っておく.
Offline Nestingを行う場合は,まずParent domainの時間積分を実行し,そのhistory outputをChild domainへの
入力データとして,初期値・境界値作成を行えばよい.NestingするDomainの数が増えても,この手順を繰り返すだけである.
Online Nestingを行う場合は,\verb|run_d##.conf|の他に,起動用コンフィグファイル(\verb|launch.conf|)が必要になる.

\vspace{0.5cm}
\noindent {\em launch.conf}
\begin{verbatim}
&PARAM_LAUNCHER
 NUM_DOMAIN  = 3,
 CONF_FILES  = run.d01.conf,run.d02.conf,run.d03.conf,
 PRC_DOMAINS = 9,27,72,
/
\end{verbatim}

上記は3つのDomainを使用したOnline Nesting計算における起動用コンフィグファイルの例である.
\verb|NUM_DOMAIN = 3|が「3つのDomainを使用する」ことを表している.
\verb|CONF_FILES|は各Domainで読み込む実行コンフィグファイル(\verb|run_d##.conf|)を指定する.
\verb|PRC_DOMAINS|は各Domainで使用するMPIプロセスの数を指定する.\verb|CONF_FILES|で羅列した
順番で指定しなければならない.従って,この場合,Domain 1(最外)は9プロセス,
Domain 2(中間)は27プロセス,そしてDomain 3(最内)は72プロセスを使用するように指定されている.
ここで指定するプロセス数はrun.confで指定されているプロセス数と合致させなければならない.
この3段のOnline Nesting計算で使用するMPIプロセスの全数は,\verb|9 + 27 + 72 = 108|プロセスである.

実行時には,単一Domainの計算とは異なり,\verb|launch.conf|を引数に指定し,
全体で使用するMPIプロセス数を指定して実行する.
\begin{verbatim}
 $ mpirun -n 108 ./scale-les launch.conf
\end{verbatim}

実行にあたって注意することは,3つのDomainを同時に実行するため,出力されるファイルの名前等をDomain毎に変更しておく必要があることである.たとえば,\verb|history.pe######.nc|は,\verb|history_d01.pe######.nc, history_d02.pe######.nc, history_d03.pe######.nc|といったようにDomain毎に名前を変えながらどのDomainの出力であるか判別がつくようにする.
ほかにLOGファイル,topoファイル,landuseファイル,boundaryファイル,initファイル,restartファイル,そしてmonitorファイルの名前を変更しておく必要がある.


\subsection{一括実行機能の使用}

ここにBulk Job機能を使った実験の実行方法を書く。

pp, init, runすべてについてBulkが可能であること。

Bulk POPSCAの利用が可能であること。

Bulk時のディレクトリ構造。

LOGメッセージの見方(COMMの種類の説明)。

