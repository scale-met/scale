\section{Postprocess : netcdf2grads}
\label{sec:net2g}
%====================================================================================

SCALE-LESの出力ファイル(\verb|history.******.nc|)を
GrADSで図化するためのnetcdf2grads の使用方法について説明する。
並列処理に対応したnetcdf2grads\_hも用意されているが、
ここではシングルノード用のnetcdf2gradsの使用方法を示す。
netcdf2grads\_h については、最後に簡単に説明する。
netcdf2gradsのソースファイルは \verb|scale/scale-les/util/netcdf2grads/|にある。
SCALE本体とは独立なので、ディレクトリを任意の場所に移動して
使用することが可能である.

\subsubsection{コンパイル}
\begin{description}
\item[Intel compiler]\mbox{}\\
 \begin{verbatim}
  ifort -convert big_endian -assume byterecl -I${NETCDF4}/include 
    -L${NETCDF4}/lib -lnetcdff -lnetcdf make_grads_file.f90 -o convine
  \end{verbatim}
\item[gfortran]\mbox{}\\
\begin{verbatim}
  gfortran -frecord-marker=4 --convert=big-endian -I${NETCDF4}/include
    -L${NETCDF4}/lib -lnetcdff -lnetcdf make_grads_file1.f90 -o convine
\end{verbatim}
\end{description}


\subsubsection{使用方法}
実行時に,インタラクティブモードかサイレントモードかを選択することが出来る.
\begin{verbatim}
   Interactive mode :  ''./convine -i''
   Silent mode      :  ''./convine -s''
\end{verbatim}
\begin{description}
\item[インタラクティブモード]\mbox{}\\
\begin{verbatim}
  cd netcdf2grads/
  ./convine -i
\end{verbatim}
と実行すると、下記のメッセージが出るので、指示通り、必要なファイルのパスを打ち込む.
\begin{verbatim}
path to configure file for run with the quotation mark
 '${path to directory of configure file}/run.conf' <- configureファイルへのパス
path to directory of history files with the quotation mark
 '${path to directory of history files}/' <- SCALE-LESの出力ファイルのパス
path to directory of output files with the quotation mark
 './grads/'  <- gradsファイルの出力先
start time of convert data
 1           <- 任意の番号の時間から変換可能. 
end time of convert data
 10          <- 変換を修了する時間
Imput number of variable
0 -> all variable output from model
 X           <- 変換したい変数の数
Imput variable
 VARIABLE(PRECなど)  <- 変換したい変数名
\end{verbatim}
うまく実行されれば、指定した出力先にctlファイルとgrdファイルが作成される。
\item[サイレントモード]\mbox{}\\
サイレントモードの場合は、あらかじめ、\verb|namelist.in|に必要な情報を書き入れておく.
\begin{verbatim}
  cd netcdf2grads/
  ./convine -s
\end{verbatim}
と実行すれば変換が始まる。
\end{description}



\subsubsection{並列処理: netcdf2grads\_h}

スパコンなどの大型計算機で並列計算を行った場合、
出力ファイルの数が多く、それぞれのファイルのデータ容量も大きい。
netcdf2gradsを並列処理したい場合にはこちらを使用する。
ここでは、K上での使用方法を簡単に説明する。

\begin{enumerate}
\item ファイルをコピー\\
 \verb|scale/scale-les/util/netcdf2grads_h| を \verb|/data/GROUP/USER/WORK_directory/| など、作業したいディレクトリにコピー。
\item コンパイル\\
 \verb|.bashrc| などに下記を設定
 \begin{verbatim}
  #--SCALE
   export SCALE_SYS=''K''
   export AGGRESIVE=''F''
   export FAST=''T''
   export DEBUG=''F''
 \end{verbatim}
 そして、コンパイル.\\
 \verb|$ make|\\
 うまくいけば、\verb|net2g|が作成される。
\item \verb|microで実行する| \\
 \verb|/scratch/GROUP/USER/|の下に、作業ディレクトリを用意する。そこに、
 \begin{verbatim}
   net2g         : copy executive file
   net2g.conf    : copy configure file
   output/  : link to directory with scale history files
   bindata/      : create directory for grads file output
   job.sh        : job script for K (option)
 \end{verbatim}
 を用意する。\verb|net2g.conf|の設定と\verb|job.sh|の設定をする。
 使用するノード数は、計算に使用したノード数の約数である必要がある。
\end{enumerate}


%####################################################################################

