%====================================================================================
\chapter{SCALEに実装されている雲モデルの概要とそれぞれの選択方法}
%====================================================================================

SCALEライブラリには5種類の雲モデルが実装されている。氷雲の微物理過程を
含まない1-momentバルクモデル(\cite{kessler_1969})、氷雲の微物理過程を
含む1-momentバルクモデル(\cite{tomita_2008})、氷雲を含む2-momentバルクモデル
(\cite{sn_2014})、1-momentビン法(\cite{suzuki_etal_2010})、
及び、超水滴法(\cite{Shima_etal_2009})である。1-momentビン法はオプションにより
氷を含む場合とそうでない場合を切り替えられる。また超水滴法に関しては
著作権の関係から公開されていない。使用したい場合はSCALEの開発者に連絡をされたい。\\
これらの雲微物理モデルの違いは雲粒の粒径分布関数の表現方法にある。
ここでは超水滴法を除く4種類の雲モデルの利用法について記述する。

\begin{enumerate}
\item {\bf 氷を含まない1-momentバルク法\cite{kessler_1969}}\\
\cite{kessler_1969}は後述の\cite{tomita_2008}と同様に1-momentバルク法は粒径分布関数を粒径分布関数の
3次のモーメント(質量)のみで表現する。\cite{kessler_1969}は雲粒と雨粒のカテゴリを考慮し、
雲粒混合比(QC[kg/kg])と雨粒混合比(QR[kg/kg])を予報する。\\
粒径分布関数はデルタ関数(すなわち雲粒のサイズは全て同じと仮定する)で表現する。\\
考慮する雲粒の成長過程は、雲粒生成(飽和調整)、凝結・蒸発、衝突併合、落下である。
\item {\bf 氷を含む1-momentバルク法\cite{tomita_2008}}\\
氷を含む1-momentバルク法は雲粒、雨粒、氷粒、雪片、あられの5カテゴリを考慮し、\\
それぞれの混合比(QC、QR、QI、QS、QG[kg/kg])をそれぞれ予報する。粒径分布は\\
雲物はデルタ分布(半径10µmで一様)、その他はMarshall-Palmer分布を仮定する。\\
考慮する成長過程は雲粒生成(飽和調整)、凝結・蒸発、衝突併合、落下である。
\item {\bf 氷を含む2-momentバルク法\cite{sn_2014}}\\
氷を含む2-momentバルク法は混合比に加えて、粒径分布関数の0次のモーメント(数濃度)で表現する。\\
考慮する水物質のカテゴリは、雲粒、雨粒、氷粒、雪片、あられの5カテゴリであり、\\
それぞれの質量混合比(QC、QR、QI、QS、QG)と数密度混合比(NC、NR、NI、NS、NG)を予報する。
粒径分布は一般ガンマ関数で近似する。考慮する成長過程は雲粒生成、凝結・蒸発、
衝突併合、分裂、落下である。
\item {\bf 1-momentビン法\cite{suzuki_etal_2010}}\\
1-momentビン法は粒径分布関数を差分化して陽に表現する。差分化された各粒径分布関数を
ビンと呼ぶためビン法と呼ばれる。各サイズのビンを質量のみで予報する(各ビンを
質量及び数密度で表現するビン法を2-momentビン法と呼ぶ)。考慮する水物質のカテゴリは
雲粒、雨粒、氷粒、雪片、あられ、ひょうの6種類であり、考慮する成長過程は雲粒生成、
凝結・蒸発、衝突併合、落下である。
\end{enumerate}

上記の4種類の雲微物理は1→4の順に高精度になるが、その分計算コストも高くなる。次にこれらの
雲物理スキームを用いて実験を行う方法を記述する。


\section{雲微物理スキームの選択方法}
%====================================================================================
雲微物理スキームを選択するには、init.confおよびrun.confの編集が必要である。まずinit.confをエディタで
開き、\verb|TRACER_TYPE|を指定する(以下は氷を含む2-momentバルク法を指定する場合の例である)。\\

\noindent {\gt \ovalbox{
\begin{tabularx}{140mm}{l}
\verb|&PARAM_TRACER| \\
\verb| TRACER_TYPE = 'SN14',| \\
\verb|/| \\
\end{tabularx}
}}\\

それぞれの雲微物理スキームを利用する際の\verb|TRACER_TYPE|は、
次の表 \ref{tab:mp_trc_type}に示すとおりである。

\begin{table}[htb]
\begin{center}
\caption{雲微物理スキームの種類と設定値}
\begin{tabularx}{150mm}{|l|X|} \hline
 \rowcolor[gray]{0.9} 雲微物理スキームの種類 & \verb|TRACER_TYPE|の設定値 \\ \hline
 氷を含まない1-momentバルク法;\cite{kessler_1969}    & \verb|KESSLER| \\ \hline           
 氷を含む1-momentバルク法;\cite{tomita_2008}         & \verb|TOMITA08| \\ \hline
 氷を含む2-momentバルク法;\cite{sn_2014}             & \verb|SN14| \\ \hline
 1-momentビン法;\cite{suzuki_etal_2010}             & \verb|SUZUKI10| \\ \hline
\end{tabularx}
\label{tab:mp_trc_type}
\end{center}
\end{table}

次にrun.confをエディタで開き\verb|TRACER_TYPE|と\verb|ATMOS_PHY_MP_TYPE|を指定する
(以下は氷を含む2-momentバルク法を指定する場合である)\\

\noindent {\gt \ovalbox{
\begin{tabularx}{140mm}{l}
\verb|&PARAM_TRACER| \\
\verb| TRACER_TYPE = 'SN14',| \\
\verb|/| \\
 \\
\verb|&PARAM_ATMOS| \\
\verb| ATMOS_PHY_MP_TYPE = "SN14",| \\
\verb|/| \\
\end{tabularx}
}}\\

この際、run.conf、init.confに記述される\verb|TRACER_TYPE|と\verb|ATMOS_PHY_MP_TYPE|は
{\bf 全て同一でなければならない}。\\
加えて、1-momentビン法を用いる際は、init.confに以下の項目を追加する必要がある。\\

\noindent {\gt \ovalbox{
\begin{tabularx}{140mm}{l}
\verb|&PARAM_MKINIT| \\
\verb| flg_bin = .true.| \\
\verb|/| \\
 \\
\verb|&PARAM_MKINIT_RF01| \\
\verb| PERTURB_AMP=0.02d0,| \\
\verb| RANDOM_LIMIT=5,| \\
\verb| RANDOM_FLAG=1| \\
\verb|/| \\
 \\
\verb|&PARAM_BIN| \\
\verb| nbin   = 33,| \\
\verb| ICEFLG =  0,| \\
\verb| nccn   =  0,| \\
\verb| kphase =  0,| \\
\verb|/| \\
 \\
\verb|&PARAM_SBMAERO| \\
\verb| F0_AERO =   1.D7,| \\
\verb| R0_AERO =   1.D-7,| \\
\verb| R_MAX   =   1.D-6,| \\
\verb| R_MIN   =   1.D-8,| \\
\verb| A_ALPHA =   3.D0,| \\
\verb| rhoa    = 2.25D3,| \\
\verb|/| \\
\end{tabularx}
}}\\

それぞれの意味は「雲微物理スキームのオプションとNamelist」の章を参照されたい。


\section{実行方法}
%====================================================================================
上記を設定して他の実験と同様に実行することで各雲微物理スキームを選択することができる。



\section{各雲物理スキームにのオプションとNamelist}
%====================================================================================
現在執筆中...


