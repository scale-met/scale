\section{表記上の注意} \label{sec:notation}

本書中では、Unix での bash 上での実行を想定して記述している。
異なる環境下では適宜読み替えて対応すること。
また、本書内では特に断りがない限り、下記の表記法に従うものとする。


コマンドラインのシンボル(\verb|$, #|)は、コマンドの実行を示す。
下記の表記の違いは、プログラムを実行する権限の違いを示している。

\begin{verbatim}
 #        <- root権限で実行するコマンド
 $        <- ユーザ権限で実行するコマンド
\end{verbatim}
%権限の一時的な切り替えにはsuコマンドを用いる。
%\verb|{User_Name}|は実際のユーザ名に読み替えること。
%\begin{verbatim}
% $ su {User_Name}   <- {User_Name}のユーザ名でログイン
% $ exit             <- {User_Name}のユーザ名でログインを終了
% $ su -             <- root権限に変更
% #
%\end{verbatim}

%コマンドオプションにハイフンを用いると、そのユーザでのログインを行う。
%用いない場合、権限のみの変更となる。またユーザ名を省略するとrootでのログインを試す。
%ユーザの一時切り替えを終わるには、exitコマンドを用いる。
%各プログラムをインストールするための圧縮ファイルは、/tmpにダウンロードされていると仮定する。
%他のディレクトリにダウンロードしてある場合は、mvコマンド等を用いて/tmpに移動しておくことを勧める。
文章表記のうち、ダブルスラッシュ(//)で始まる行は解説のためのもので、実際に記述する必要はない。

下記のように四角い囲みで区切られた部分は、コマンドラインのメッセージ部分を表す。\\

\noindent {\small {\gt
\fbox{
\begin{tabularx}{150mm}{l}
 -- -- -- -- コマンドラインのメッセージ\\
 -- -- -- -- -- -- -- -- コマンドラインのメッセージ\\
 -- -- -- -- -- -- -- -- -- -- -- -- コマンドラインのメッセージ\\
\end{tabularx}
}}}\\

また、下記のように丸い囲みで区切られた部分は、エディタでファイルを編集する記述内容を表す、
もしくはファイル内の記述を参照している部分である。\\

\noindent {\small {\gt
\ovalbox{
\begin{tabularx}{150mm}{l}
 -- -- -- -- ファイル中の記述\\
 -- -- -- -- -- -- -- -- ファイル中の記述\\
 -- -- -- -- -- -- -- -- -- -- -- -- ファイル中の記述\\
\end{tabularx}
}}}\\

~\\
~\\
また、本書では、ネームリストとその項目についての記述を以下のように表記するものとする。
\begin{description}
\item[ネームリスト] \namelist{namelist}
\item[ネームリストの中のの項目] \nmitem{item_of_namelist}
\end{description}


%\begin{verbatim}
% $ vi
%\end{verbatim}
%上記のコマンドは、汎用テキストエディタプログラムを実行を意味する。
%各自の使いやすいエディタ (gedit、emacsなど) へ適宜読み替えること。


