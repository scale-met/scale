%-------------------------------------------------------%
%section{概要}
%-------------------------------------------------------%

このチュートリアルでは、図 \ref{fig:tutrial_real_domain}に示した
領域を対象とした現実大気実験を行う。
以下のディレクトリーに移動して作業を行う。
\begin{alltt}
 scale-{\version}/scale-rm/test/tutorial/real/
\end{alltt}
これ以降の説明では、\texttt{scale-{\version}/scale-rm/test/tutorial/}の絶対PATHを
\verb|${Tutorial_DIR}|と示すこととする。
\verb|${Tutorial_DIR}|以下のディレクトリ構造は下記の通りである。
\begin{alltt}
  init/
  net2g/
  pp/
  run/
  sample/  : サンプルのconfファイル
  tools/   : tutrial実行のためのツール
\end{alltt}

SCALE-RMの実行過程は、図\ref{fig:howto}に示されるように
\begin{enumerate}
\item pp    : 地形・土地利用データの作成
\item init  : 初期値・境界値データの作成
\item run   : 時間積分を行う(モデル本体の実行)
\item net2g : 出力データのNetCDFからGrADS形式への変換(オプション) 
\end{enumerate}
といった手順で実行する。


\begin{figure}[h]
\begin{center}
  \includegraphics[width=0.9\hsize]{./figure/real_procedure.eps}\\
  \caption{SCALE-RMモデルの実行過程}
  \label{fig:howto}
\end{center}
\end{figure}

計算領域(ドメイン)の設定は表\ref{tab:grids}のようになっている。
チュートリアルは、SCALEの使い方を学ぶことが目的であり、
短い時間で実行可能な設定にしている。
領域モデルの実験設定として必ずしも適切な設定を選択しているとは限らないので
(例えば、15kmの水平解像度で積雲パラメタリゼーションなし)
ご留意頂きたい。

このチュートリアルを実行するには、下記の条件を満たす計算機環境が推奨される。
\begin{itemize}
\item CPU: 2コア/4スレッド以上の演算コアを持つCPU(4コア以上を推奨)
\item Memory容量: 4GB以上をプログラムに割当可能(8GB以上を搭載した計算機を推奨)
\item HDD空き容量: 7GB以上の空き容量
\end{itemize}
本節の説明で使用した環境は次のとおりである。
\begin{itemize}
\item CPU: Intel Xeon E5-2620 2.0GHz 6コア(ただし4コアのみ使用)
\item Memory: DDR3-1333 (4 channel, 32GB搭載)
\item OS: CentOS 6.6 x86-64, CentOS 7.1 x86-64, openSUSE 13.2 x86-64
\end{itemize}

\begin{figure}[tb]
\begin{center}
  \includegraphics[width=0.5\hsize]{./figure/real_domain.eps}\\
  \caption{計算領域.カラーシェードは地形の標高を示す.}
  \label{fig:tutrial_real_domain}
\end{center}
\end{figure}

\begin{table}[h]
\begin{center}
  \caption{実験設定の概略}
  \label{tab:grids}
  \begin{tabularx}{150mm}{|l|X|} \hline
    \rowcolor[gray]{0.9} 項目 & 設定 \\ \hline
    MPIプロセス分割 (東西 x 南北) & 2 x 2 (合計4プロセス) \\ \hline
    水平格子数 (東西 x 南北) & 60格子点 x 60格子点 \\ \hline
    鉛直層数                 & 36層                  \\ \hline
    水平格子間隔             & dx = dy = 15km       \\ \hline
    積分期間 & 2014年8月10日 00UTC~12UTC (12時間積分) \\ \hline
    時間ステップ間隔 & 60 sec (720 steps) \\ \hline
  \end{tabularx}
\end{center}
\end{table}


