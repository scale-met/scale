%-------------------------------------------------------%
\section{初期値・境界値データの作成:init} \label{sec:tutrial_real_init}
%-------------------------------------------------------%

init では、SCALE計算に必要な初期値・境界値データを作成する。
まず、initディレクトリへ移動する。
\begin{verbatim}
 $ cd ${Tutorial_DIR}/real/init
\end{verbatim}
ディレクトリの中には、\verb|init.conf|という名前の設定ファイルが準備されている。
\verb|pp.conf|と同様に、実験設定に合わせて、この\verb|init.conf|を書き換える必要があるが、
チュートリアル用の\verb|init.conf|ファイルは表\ref{tab:grids}の設定に
すでに合わせてある。
初期値・境界値データの作成には前節で作成した地形・土地利用データを利用する。
これは、\verb|init.conf|の中で、下記のように相対PATHを用いて参照するように設定されている。\\

\noindent {\gt
f\ovalbox{
\begin{tabularx}{140mm}{l}
\verb|&PARAM_TOPO| \\
\verb|   TOPO_IN_BASENAME = "../pp/topo_d01",| \\
\verb|/| \\
 \\
\verb|&PARAM_LANDUSE| \\
\verb|   LANDUSE_IN_BASENAME  = "../pp/landuse_d01",| \\
\verb|/| \\
\end{tabularx}
}}\\

\noindent その他に\verb|init.conf|の設定の中で特に確認してほしい項目は、
\namelist{PARAM_MKINIT_REAL}である。\\

\noindent {\gt
\ovalbox{
\begin{tabularx}{140mm}{l}
\verb|&PARAM_MKINIT_REAL| \\
\verb|   BASENAME_BOUNDARY    = "boundary_d01",|   {\small ← 境界値データの出力名} \\
\verb|   FILETYPE_ORG         = "GrADS",| \\
\verb|   NUMBER_OF_FILES      = 3,|                      {\small ← 読み込むファイルの数} \\
\verb|   BOUNDARY_UPDATE_DT   = 21600.D0,|           {\small ← 入力データの時間間隔} \\
\verb|   INTERP_SERC_DIV_NUM  = 20,|                   {\small ← 内挿計算用のチューニングパラメータ} \\
\verb|   PARENT_MP_TYPE       = 3,| \\
\verb|   USE_FILE_DENSITY     = .false.,|          {\small ← 親モデルの大気密度データを使うか} \\
\verb|   USE_FILE_LANDWATER   = .true.,|           {\small ← 親モデルの土壌水分データを使うか} \\
\verb|   INTRP_LAND_SFC_TEMP  = "mask",|           {\small ← 親モデルの欠測値処理方法} \\
\verb|   INTRP_LAND_TEMP      = "fill",| \\
\verb|   INTRP_LAND_WATER     = "fill",| \\
\verb|   INTRP_OCEAN_SFC_TEMP = "mask",| \\
\verb|   INTRP_OCEAN_TEMP     = "mask",| \\
\verb|/| \\
\end{tabularx}
}}\\

\noindent \nmitem{FILETYPE_ORG}は入力する気象場データのファイルフォーマットに
関するパラメータを設定しており、ここでは
grads形式のデータを読み込むことを指定している。
詳細な設定ファイルの内容については、付録\ref{app:namelist}を参照されたい。

次に、コンパイル済みのバイナリをinitディレクトリへリンクする。
\begin{verbatim}
  $ ln -s ../../bin/scale-rm_init ./
\end{verbatim}
第\ref{sec:tutrial_real_prep_fnl}節で作成した入力データに、
initディレクトリの中に準備されている\verb|"gradsinput-link_FNL.sh"|を用いてリンクをはる。
\begin{verbatim}
  $ sh gradsinput-link_FNL.sh
\end{verbatim}
下記のgrads形式のファイルにリンクが張れれば成功である。\\

\noindent {\gt
\fbox{
\begin{tabularx}{140mm}{l}
\verb|FNLatm_00000.grd -> ../tools/FNL_output/200707/FNLatm_2007071418.grd| \\
\verb|FNLatm_00001.grd -> ../tools/FNL_output/200707/FNLatm_2007071500.grd| \\
\verb|FNLatm_00002.grd -> ../tools/FNL_output/200707/FNLatm_2007071506.grd| \\
\verb|FNLatm_00003.grd -> ../tools/FNL_output/200707/FNLatm_2007071512.grd| \\
\verb|FNLland_00000.grd -> ../tools/FNL_output/200707/FNLland_2007071418.grd| \\
\verb|FNLland_00001.grd -> ../tools/FNL_output/200707/FNLland_2007071500.grd| \\
\verb|FNLland_00002.grd -> ../tools/FNL_output/200707/FNLland_2007071506.grd| \\
\verb|FNLland_00003.grd -> ../tools/FNL_output/200707/FNLland_2007071512.grd| \\
\verb|FNLsfc_00000.grd -> ../tools/FNL_output/200707/FNLsfc_2007071418.grd| \\
\verb|FNLsfc_00001.grd -> ../tools/FNL_output/200707/FNLsfc_2007071500.grd| \\
\verb|FNLsfc_00002.grd -> ../tools/FNL_output/200707/FNLsfc_2007071506.grd| \\
\verb|FNLsfc_00003.grd -> ../tools/FNL_output/200707/FNLsfc_2007071512.grd| \\
\end{tabularx}
}}\\






次に、陸面の変数を用意するのに必要なパラメータファイルにリンクをはる。
\begin{verbatim}
 $ ln -s ../../../data/land/* ./   <- 陸面スキーム用のパラメータファイル
\end{verbatim}
準備が整ったら、4つのMPIプロセスを使用してinitを実行する。
\begin{verbatim}
 $ mpirun -n 4 ./scale-rm_init init.conf
\end{verbatim}

正常にジョブが終了すると、
\begin{verbatim}
 $ ls
  boundary_d01.pe000000.nc
  boundary_d01.pe000001.nc
  boundary_d01.pe000002.nc
  boundary_d01.pe000003.nc
  init_d01_20070714-180000.000.pe000000.nc
  init_d01_20070714-180000.000.pe000001.nc
  init_d01_20070714-180000.000.pe000002.nc
  init_d01_20070714-180000.000.pe000003.nc
  init_LOG_d01.pe000000
\end{verbatim}
が作成される。
\verb|boundary_d01.pe######.nc|は境界値データ、
\verb|init_d01_20070714-180000.000.pe######.nc|は初期値データ、
\verb|init_LOG_d01.pe######|はログファイルである。
\verb|######|はMPIプロセス番号を表している。


%% サポート外
%% \vspace{1cm}
%% \noindent {\Large\em OPTION} \hrulefill \\
%% gpviewがインストールされている場合,作成された初期値と境界値が
%% 正しく作成されているかどうかを確認することが出来る。
%% 正しく作成されていれば、図 \ref{fig:init}と同じように描かれる。

%% \begin{verbatim}
%% $ gpvect --scalar --slice z=1500 --nocont --aspect=1 --range=0.002:0.016          \
%%          --xintv=10 --yintv=10 --unit_vect init_d01_20140810-000000.000.pe00*@QV      \
%%          init_d01_20140810-000000.000.pe00*@MOMX init_d01_20140810-000000.000.pe00*@MOMY
%% \end{verbatim}


%% \begin{figure}[h]
%% \begin{center}
%%   \includegraphics[width=0.7\hsize]{./figure/init_qv-momxy.eps}\\
%%   \caption{チュートリアル実験の高さ1500mにおける初期場の様子.カラーシェードは比湿,
%%            ベクトルは水平運動量フラックスを表している.}
%%   \label{fig:init}
%% \end{center}
%% \end{figure}

