\section{バッファー領域の設定}
\label{sec:buffer}
%-----------------------------------------------------------------------
SCALEでは計算領域境界のすぐ内側に「バッファー格子点」を設定することができる。バッファー格子点では、指定された値や
親モデルのデータにダンピングする、いわゆるナッジングが行われる。これは、モデル最上層で重力波が反射することを
緩和するための「スポンジ層」として利用したり、現実大気実験を行う際に側面境界において親モデルと
SCALEで計算される大気場が乖離することを防ぐための「緩和領域」として利用される。

バッファー格子点は、\verb|BUFFER_DX|、\verb|BUFFER_DY|、\verb|BUFFER_DZ|、および\verb|BUFFFACT|の変数によって
設定される。\verb|BUFFER_DX|、\verb|BUFFER_DY|、\verb|BUFFER_DZ|は、それぞれの方向のバッファー領域の幅
(単位はメートル)を示す。バッファー領域は境界から内側に\verb|BUFFER_DX|等で指定された幅で設置され、
バッファー領域内の格子点はバッファー格子点として認識される。したがって、その分だけナッジングの影響を受けないで計算
される自由領域は狭くなることに注意が必要である。水平方向には東西南北どの方向にも同様にバッファー領域が設定されるが、
鉛直方向には計算領域の上端にのみバッファー領域が設定され、下端には設定されない。
\verb|BUFFFACT|はバッファー領域内の格子点の格子間隔に対するストレッチ係数である。
下記の関係式によって、バッファー領域内の格子間隔(BDX)は決定される。
\begin{eqnarray}
BDX_{i+1}&=&{BDX_{i}}^{BUFFFACT} \nonumber \\
BDX_{1}&=&DX_{default} \nonumber
\end{eqnarray}
ここで、$i$はバッファー領域内の格子点番号を表し、
計算領域の内側から外側へ向かって番号が振られる。
$DX_{default}$は\verb|PARAM_GRID|の項目で設定される
元々の\verb|DX|を意味する。
$i$の最大値$n$は、$\sum_{i=1}^n BDX_{i} > BUFFER_{DX}$の関係を達成する最小の数として決定される。
したがって、``\verb|BUFFFACT = 1.0|''ならば、格子間隔をストレッチせずに等間隔であることを意味し、
``\verb|BUFFFACT = 1.2|''ならば、計算領域の内側から境界へ向けて、格子点間距離がもともとの格子間隔から
1.2倍の割合で広がっていくことを意味する。\verb|BUFFFACT|の値を大きくすると、バッファー領域の幅が同じでも
バッファー格子点として使用される格子点数は少なくなる。

例として理想実験のチュートリアルのrun.confファイル(run\_R20kmDX500m.conf;上に表示したもの)を見てみると、
\verb|BUFFER_DZ = 5000.D0|、\verb|BUFFFACT = 1.0D0|と指定されている。\verb|BUFFER_DZ|以外には指定がないので、
水平方向にはバッファー領域が設定されない(この実験は水平方向には周期境界条件になっておりバッファー領域は必要ない)。
\verb|BUFFFACT = 1.0D0|となっているため、もとの格子間隔のままストレッチは行わない。
最上層の高度が20222 mと指定されており、鉛直方向のバッファー領域の幅は5000 mと指定されているため、
モデル計算領域のトップから高度15222 mまでがバッファー領域となる。鉛直層の高度指定によれば、92層目の高度が
15517 mで、91層目の高度が14910 mであるため、92層目まではバッファー領域だが、91層目はバッファー領域ではない。
繰り返しになるが、これらの設定は\verb|pp_***.conf|、\verb|init_***.conf|、\verb|run_***.conf|の
すべてのconfigファイルにおいて共通した設定になっていなければならない。

SCALEでは、バッファー領域の大きさ、バッファー格子点の数について、まだ明確な指標を設定できていないが、
鉛直方向(計算領域トップ)のバッファー格子点は5点以上、水平方向(側面境界付近)のバッファー格子点は
20〜40点程度を推奨している。実験設定や事例によっては、さらにバッファー格子点を増やしたり、ストレッチ係数を
用いてバッファー領域を広げる処理が必要であったり、ここでは説明しなかったが\verb|ATMOS_BOUNDARY_taux|、
\verb|ATMOS_BOUNDARY_tauy|といった設定項目を調整してバッファー領域のナッジング強度を強める必要があったりする。

