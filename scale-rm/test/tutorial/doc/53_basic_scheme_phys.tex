\section{物理スキームの設定} \label{sec:basic_physics}
%------------------------------------------------------

\subsection{雲微物理スキーム} \label{sec:basic_microphys}
%------------------------------------------------------
雲微物理スキームの選択は、init.confとrun.conf中の
\verb|PARAM_TRACER|の\verb|TRACER_TYPE|、
及び、\verb|PARAM_ATMOS|の\verb|ATMOS_PHY_MP_TYPE|で設定する。
このとき{\color{red}{\verb|TRACER_TYPE|と\verb|ATMOS_PHY_MP_TYPE|は普通は同じものを設定し}}
、かつ、\textcolor{red}{init.conf,run.confで同一の設定}とする必要がある。
ただし、\verb|ATMOS_PHY_MP_TYPE|を\verb|OFF|とするときは、\verb|TRACER_TYPE|は何を設定してもよいが、乾燥大気の計算をする場合は\verb|TRACER_TYPE = DRY|とするのがよい。
雲微物理スキームを計算する(callする)タイミングは、
\verb|PARAM_TIME|で設定するが、これについては
第\ref{sec:timeintiv}節を参照のこと。\\



\noindent {\gt
\ovalbox{
\begin{tabularx}{140mm}{ll}
\verb|&PARAM_ATMOS  | & \\
\verb| ATMOS_PHY_MP_TYPE = "TOMITA08", | & ; 表\ref{tab:nml_atm_mp}より選択。\\
\verb|/             | & \\
\\
\verb|&PARAM_TRACER | & \\
\verb| TRACER_TYPE = "TOMITA08", | & \verb|ATMOS_PHY_MP_TYPE|と同じスキーム。\\
\verb|/             | & \\
\end{tabularx}
}}\\

\begin{table}[h]
\begin{center}
  \caption{雲微物理スキームの設定}
  \label{tab:nml_atm_mp}
  \begin{tabularx}{150mm}{lXX} \hline
    \rowcolor[gray]{0.9}  設定名 & スキームの説明 & 文献\\ \hline
     \verb|OFF|      & 雲微物理による相変化を計算しない &  \\
     \verb|KESSLER|  & 水雲のみの1-momentバルク法 & \citet{kessler_1969} \\
     \verb|TOMITA08| & 氷雲を含む1-momentバルク法 & \citet{tomita_2008} \\
     \verb|SN14|     & 氷雲を含む2-momentバルク法 & \citet{sn_2014} \\
     \verb|SUZUKI10| & 1-momentビン法(氷雲を含むか否かはオプションで選択) & \citet{suzuki_etal_2010} \\
    \hline
  \end{tabularx}
\end{center}
\end{table}

\verb|SUZUKI10|以外を選択した場合は、
init.conf、run.confの\verb|TRACER_TYPE|と\verb|ATMOS_PHY_MP_TYPE|を
変更するだけで実行可能である。


\subsubsection{SUZUKI10}
%---------------------------
\verb|ATMOS_PHY_MP_TYPE = "SUZUKI10"|を選択した場合は、init.conf、run.confの双方に
下記を追加する必要がある。\\

\noindent {\gt
\ovalbox{
\begin{tabularx}{140mm}{ll}
\verb|&PARAM_BIN|   &  \\
\verb| nbin   = 33, & (ビンの数)| \\
\verb| ICEFLG =  1, & (氷雲を考慮するか否か,0->水雲のみ,1->氷雲も含む)| \\
\verb|/|            & \\
\end{tabularx}
}}\\

この場合も、
{\color{red}{init.confとrun.confに記載される\verb|PARAM_BIN|は同一にする必要がある}}。
\verb|SUZUKI10|を選択した時には、micpara.datという
雲微物理の計算に必要なファイルが自動生成される。
micpara.datがすでに存在する場合はあるものを利用するが、
nbinが変わると新たに作成しなければならない。
micpara.datの1行目にnbinの情報が記載されているが、
もしrun.confに記載されるnbinと
micpara.datに記載されているnbinが異なれば、\\

\noindent {\gt
\fbox{
\begin{tabularx}{140mm}{l}
\verb|xxx nbin in inc_tracer and nbin in micpara.dat is different check!| \\
\end{tabularx}
}}\\

\noindent というエラーメッセージを標準出力に出力して計算が落ちるようになっている。
そのため、nbinを変更した際は、micpara.datを消去して
新たに作り直す必要がある
(micpara.datを消して再度SCALEをSUZUKI10を用いて実行すれば自動的に新しいmicpara.datが生成される)。



\subsection{乱流スキーム} \label{sec:basic_turbulence}
%------------------------------------------------------

乱流スキームの選択は,init.confとrun.conf中の
\verb|PARAM_ATMOS|の\verb|ATMOS_PHY_TB_TYPE|で設定する。
乱流スキームを計算する(callする)タイミングは、
\verb|PARAM_TIME|で設定するが、これについては
第\ref{sec:timeintiv}節を参照のこと。\\

\noindent {\gt
\ovalbox{
\begin{tabularx}{140mm}{ll}
\verb|&PARAM_ATMOS  | & \\
\verb| ATMOS_PHY_TB_TYPE = "MYNN", | & ; 表\ref{tab:nml_atm_tb}より選択。\\
\verb|/             | & \\
\end{tabularx}
}}\\

\begin{table}[h]
\begin{center}
  \caption{乱流スキームの設定}
  \label{tab:nml_atm_tb}
  \begin{tabularx}{150mm}{lXX} \hline
    \rowcolor[gray]{0.9}  設定名 & スキームの説明 & 文献\\ \hline
      \verb|OFF|          & サブグリッドスケール乱流の寄与を計算しない &  \\
      \verb|SMAGORINSKY|  & Smagorinsky typeのサブグリッドモデル    & \citet{smagorinsky_1963,lilly_1962,Brown_etal_1994,Scotti_1993} \\
      \verb|D1980|        & Deardorff(1980)サブグリットモデル &\citet{Deardorff_1980} \\
      \verb|MYNN|         & MYNN Level 2.5 乱流モデル & \citet{my_1982,nakanishi_2004} \\
      \verb|HYBRID|       & MYNN と SMAGORINSKYのハイブリット &  \\
    \hline
  \end{tabularx}
\end{center}
\end{table}




\subsection{放射スキーム} \label{sec:basic_radiation}
%------------------------------------------------------
放射スキームの選択は、init.confとrun.conf中の
\verb|PARAM_ATMOS|の\verb|ATMOS_PHY_RD_TYPE|で設定する。
放射スキームを計算する(callする)タイミングは、
\verb|PARAM_TIME|で設定するが、これについては
第\ref{sec:timeintiv}節を参照のこと。\\

\noindent {\gt
\ovalbox{
\begin{tabularx}{140mm}{ll}
\verb|&PARAM_ATMOS  | & \\
\verb| ATMOS_PHY_RD_TYPE = "MSTRN", | & ; 表\ref{tab:nml_atm_rd}より選択。\\
\verb|/             | & \\
\end{tabularx}
}}\\

\begin{table}[h]
\begin{center}
  \caption{放射スキームの選択肢}
  \label{tab:nml_atm_rd}
  \begin{tabularx}{150mm}{lXX} \hline
    \rowcolor[gray]{0.9}  設定名 & スキームの説明 & 文献\\ \hline
      \verb|OFF|    & \textcolor{red}{使用しない, どういう状態を想定?} &  \\
      \verb|MSTRN|  & MSTRN-X    & \citet{sekiguchi_2008} \\
    \hline
  \end{tabularx}
\end{center}
\end{table}

\verb|MSTRN|を実行するには、各種外部データとパラメタテーブルが必要である。
オゾンのプロファイルなどの外部データとパラメタテーブルは、
\begin{verbatim}
  scale-rm/test/data/rad/
\end{verbatim}
に用意されている。
放射スキームを利用する場合は、これらのファイルが必要となる。



\subsection{地表面(大気下端境界)} \label{sec:basic_surface}
%------------------------------------------------------
大気下端境界の設定は、init.confとrun.conf中の
\verb|PARAM_ATMOS|の\verb|ATMOS_PHY_SF_TYPE|で設定する。\\

\noindent {\gt
\ovalbox{
\begin{tabularx}{140mm}{ll}
\verb|&PARAM_ATMOS  | & \\
\verb| ATMOS_PHY_SF_TYPE = "COUPLE", | & ; 表\ref{tab:nml_atm_sf}より選択。\\
\verb|/             | & \\
\end{tabularx}
}}\\

\begin{table}[h]
\begin{center}
  \caption{大気下端境界の選択肢}
  \label{tab:nml_atm_sf}
  \begin{tabularx}{150mm}{lX} \hline
    \rowcolor[gray]{0.9}  設定名 & スキームの説明              \\ \hline
      \verb|OFF|     & \textcolor{red}{ゼロフラックスを想定}   \\
      \verb|CONST|   & Surface fluxを任意の値に固定             \\ %詳細は、表\ref{tab:sfcflux}参照。
      \verb|BULK|    & バルクモデル \textcolor{red}{これって陸面バルクとどう違うの? パラメータだけ?海想定?}  \\
      \verb|COUPLE|  & 海・陸面・都市モデル使用する場合に選択   \\
    \hline
  \end{tabularx}
\end{center}
\end{table}

地表面スキームをcallするタイミングは、
\verb|PARAM_TIME|で設定するが、これについては
第\ref{sec:timeintiv}節を参照のこと。


\subsubsection{CONST設定}
%---------------------------
\verb|ATMOS_PHY_SF_TYPE = "CONST"|を選択した場合は、run.confで
下記を設定することにより、任意の値に固定することが可能である。
下記の値はデフォルトの設定を示す。\\

\noindent {\small {\gt
\ovalbox{
\begin{tabularx}{150mm}{lX}
 \\
 \verb|&PARAM_ATMOS_PHY_SF_CONST                | & \\
 \verb| ATMOS_PHY_SF_FLG_MOM_FLUX   =    0      | & 0: Bulk coefficient is constant \\
                                                  & 1: Friction velocity is constant \\
 \verb| ATMOS_PHY_SF_U_minM         =    0.0_DP | & Minimum limit of absolute velocity for momentum [m/s] \\
 \verb| ATMOS_PHY_SF_Const_Cm       = 0.0011_DP | & Constant bulk coefficient for momentum [NIL] \\
                                                  &  (\verb|ATMOS_PHY_SF_FLG_MOM_FLUX = 0| のとき有効) \\
 \verb| ATMOS_PHY_SF_CM_min         = 1.0E-5_DP | & Minimum limit of bulk coefficient for momentum [NIL] \\
                                                  &  (\verb|ATMOS_PHY_SF_FLG_MOM_FLUX = 1| のとき有効) \\
 \verb| ATMOS_PHY_SF_Const_Ustar    =   0.25_DP | & Constant friction velocity [m/s] \\
                                                  &  (\verb|ATMOS_PHY_SF_FLG_MOM_FLUX = 1| のとき有効) \\
 \verb| ATMOS_PHY_SF_Const_SH       =   15.0_DP | & Constant surface sensible heat flux [W/m2] \\
 \verb| ATMOS_PHY_SF_FLG_SH_DIURNAL =  .false.  | & Diurnal modulation for sensible heat flux? [logical]\\
 \verb| ATMOS_PHY_SF_Const_FREQ     =   24.0_DP | & Frequency of sensible heat flux modulation [hour]\\
 \verb| ATMOS_PHY_SF_Const_LH       =  115.0_DP | & Constant surface latent   heat flux [W/m2] \\
 \verb|/|            & \\
 \\
\end{tabularx}
}}}\\


\subsection{海洋スキーム(大気-海面フラックス)} \label{sec:basic_ocean}
%------------------------------------------------------

海洋スキームをONにする場合、
\verb|&PARAM_ATMOS|の\verb|ATMOS_PHY_SF_TYPE = "COUPLE"|とする必要がある。
海洋スキームの選択は、init.confとrun.conf中の
\verb|PARAM_OCEAN|の\verb|OCEAN_TYPE|で設定する。
海洋スキームを計算(フラックスをupdate)するタイミングは、
\verb|PARAM_TIME|で設定するが、これについては
第\ref{sec:timeintiv}節を参照のこと。\\

\noindent {\gt
\ovalbox{
\begin{tabularx}{140mm}{ll}
\verb|&PARAM_ATMOS  | & \\
\verb| ATMOS_PHY_SF_TYPE = "COUPLE", | &\\
\verb|/             | & \\
\\
\verb|&PARAM_OCEAN  | & \\
\verb| OCEAN_TYPE = "CONST", | & ; 表\ref{tab:nml_ocean}より選択。\\
\verb|/             | & \\
\end{tabularx}
}}\\

\begin{table}[h]
\begin{center}
  \caption{海洋スキームの選択肢}
  \label{tab:nml_ocean}
  \begin{tabularx}{150mm}{lX} \hline
    \rowcolor[gray]{0.9}  設定名 & スキームの説明 \\ \hline
      \verb|OFF|   & 土地利用にOCEANがない場合のみ使用可    \\
      \verb|CONST| & 初期値固定                              \\
      \verb|FILE|  & 外部ファイルから与える (時間変化あり)   \\
      \verb|SLAB|  & 海洋スラブモデル                        \\
    \hline
  \end{tabularx}
\end{center}
\end{table}


\subsubsection{海洋スラブモデル}
%---------------------------
\verb|OCEAN_TYPE = "SLAB"|を選択した場合は、init.confとrun.confで
モデルの深さを設定することができる。\\

\noindent {\gt
\ovalbox{
\begin{tabularx}{140mm}{ll}
 \verb|&PARAM_OCEAN_PHY_SLAB                 | & \\
 \verb|  OCEAN_PHY_SLAB_DEPTH   = 10.0_DP,   | & ; デフォルト設定 \\
 \verb|/|            & \\
\end{tabularx}
}}\\


\subsubsection{外部ファイル入力}
%---------------------------
\verb|OCEAN_TYPE = "FILE"|を選択した場合は、init.confとrun.confで
外部入力ファイルの設定が必要である。\\

\noindent {\gt
\ovalbox{
\begin{tabularx}{140mm}{ll}
 \verb|&EXTITEM            |                       & \\
 \verb| basename   = "../init/output/ocean_d01", | & ; 入力ファイル\\
 \verb| varname    = "OCEAN_TEMP",               | & ; \verb|"OCEAN_TEMP"|と書く。\\
 \verb| step_limit = 1800, |                        & \\
 \verb| step_fixed =  -1,  |                        & \\
 \verb| enable_periodic_year  = .false.,|           & \\
 \verb| enable_periodic_month = .false.,|           & \\
 \verb| enable_periodic_day   = .false.,|           & \\
 \verb|/|            & \\
\end{tabularx}
}}\\



\subsection{陸面モデル(大気-陸面フラックス)} \label{sec:basic_land}
%------------------------------------------------------
陸面モデルをONにする場合、
\verb|&PARAM_ATMOS|の\verb|ATMOS_PHY_SF_TYPE = "COUPLE"|である必要がある。
陸面モデルの選択は、init.confとrun.conf中の
\verb|PARAM_LAND|の\verb|LAND_TYPE|で設定する。
陸面モデルを計算する(callする)タイミングは、
\verb|PARAM_TIME|で設定するが、これについては
第\ref{sec:timeintiv}節を参照のこと。\\

\noindent {\gt
\ovalbox{
\begin{tabularx}{140mm}{ll}
\verb|&PARAM_ATMOS  | & \\
\verb| ATMOS_PHY_SF_TYPE = "COUPLE", | & \\
\verb|/             | & \\
\\
\verb|&PARAM_LAND  | & \\
\verb| LAND_TYPE = "SLAB", | & ; 表\ref{tab:nml_land}より選択。\\
\verb|/             | & \\
\end{tabularx}
}}\\

\begin{table}[h]
\begin{center}
  \caption{陸面スキームの選択肢}
  \label{tab:nml_land}
  \begin{tabularx}{150mm}{llX} \hline
    \rowcolor[gray]{0.9}  設定名 & スキームの説明 & 文献\\ \hline
      \verb|OFF|   & 土地利用にLANDがない場合のみ使用可  &  \\
      \verb|CONST| & 土壌温度・土壌水分・地表面温度 初期値固定 &  \\
      \verb|SLAB|  & 熱拡散、バケツモデルモデル   &  \\
    \hline
  \end{tabularx}
\end{center}
\end{table}



\subsubsection{陸面スラブモデル}
%---------------------------
\verb|LAND_TYPE = "SLAB"|を選択した場合は、
土地利用データとそれに対応するパラメタテーブルが必要である。
パラメタテーブルは、
\begin{verbatim}
  scale-rm/test/data/land/param.bucket.conf
\end{verbatim}
に用意されている。
ファイルには、各土地利用区分とそれに対応する粗度長などのパラメタが与えられている。



フラックス計算に使用するバルク交換係数の計算スキームは
run.conf中の\verb|PARAM_BULKFLUX|の\verb|BULKFLUX_TYPE|で設定する。\\

\noindent {\gt
\ovalbox{
\begin{tabularx}{140mm}{ll}
\verb|&PARAM_BULKFLUX  | & \\
\verb| BULKFLUX_TYPE = "B91W01", | & ; 表\ref{tab:nml_bulk}より選択。\\
\verb|/             | & \\
\end{tabularx}
}}\\

\begin{table}[h]
\begin{center}
  \caption{バルクフラックススキームの選択肢}
  \label{tab:nml_bulk}
  \begin{tabularx}{150mm}{llX} \hline
    \rowcolor[gray]{0.9}  設定名 & スキームの説明 & 文献\\ \hline
      \verb|B91W01| & デフォルト & \citet{beljaars_1991,wilson_2001}\\
      \verb|U95|    &            & \citet{uno_1995}\\
    \hline
  \end{tabularx}
\end{center}
\end{table}




\subsection{都市モデル(大気-都市面フラックス)} \label{sec:basic_urban}
%------------------------------------------------------

都市モデルをONにする場合、
\verb|&PARAM_ATMOS|の\verb|ATMOS_PHY_SF_TYPE = "COUPLE"|である必要がある。
都市モデルの選択は、init.confとrun.conf中の
\verb|PARAM_URBAN|の\verb|URBAN_TYPE|で設定する。
都市モデルを計算する(callする)タイミングは、
\verb|PARAM_TIME|で設定するが、これについては
第\ref{sec:timeintiv}節を参照のこと。\\

\noindent {\gt
\ovalbox{
\begin{tabularx}{140mm}{ll}
\verb|&PARAM_ATMOS  | & \\
\verb| ATMOS_PHY_SF_TYPE = "COUPLE", | &\\
\verb|/             | & \\
\\
\verb|&PARAM_URBAN  | & \\
\verb| URBAN_TYPE="SLC", | & ; 表\ref{tab:nml_urban}より選択。\\
\verb|/             | & \\
\end{tabularx}
}}\\

\begin{table}[h]
\begin{center}
  \caption{都市スキームの選択肢}
  \label{tab:nml_urban}
  \begin{tabularx}{150mm}{llX} \hline
    \rowcolor[gray]{0.9}  設定名 & スキームの説明 & 文献\\ \hline
      \verb|OFF|  & 土地利用にURBANがない場合のみ使用可 &   \\
      \verb|SLC|  & 単層キャノピーモデル   & \citet{kusaka_2001} \\
    \hline
  \end{tabularx}
\end{center}
\end{table}
