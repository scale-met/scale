\section{積分時間と積分時間間隔の設定} \label{sec:timeintiv}
%------------------------------------------------------
例えば理想実験チュートリアルでは、積分時間は1時間とし、時間積分間隔として力学過程は1.0秒、雲物理過程は10.0秒で実行したが、
積分時間を伸ばしたい場合や、計算にかかる時間を短くするために積分時間間隔を長くしたり、計算不安定を防ぐために
積分時間間隔を短くすることがあるだろう。

積分時間と積分時間間隔の設定は、configファイル\verb|run_***.conf|の\verb|PARAM_PRC|の項目を編集することで設定できる。
この項目はモデル本体の実行だけで有効である。理想実験チュートリアルで使用した\verb|run_R20kmDX500m.conf|の例を示す。\\

\noindent {\small {\gt
\ovalbox{
\begin{tabularx}{140mm}{l}
\verb|&PARAM_TIME|\\
\verb| TIME_STARTDATE             = 0000, 1, 1, 0, 0, 0,|(計算開始の日付:放射過程を用いる実験等で必要)\\
\verb| TIME_STARTMS               = 0.D0,| (計算開始時刻[mili sec])\\
\verb| TIME_DURATION              = 3600.0D0,| (積分時間[単位はTIME\_DURATION\_UNITで決定])\\
\verb| TIME_DURATION_UNIT         = "SEC",| (積分時間TIME\_DURATIONの単位)\\
\verb| TIME_DT                    = 5.0D0,| (移流のタイムステップ)\\
\verb| TIME_DT_UNIT               = "SEC",| (TIME\_DTの単位)\\
\verb| TIME_DT_ATMOS_DYN          = 1.0D0,| (力学過程のタイムステップ)\\
\verb| TIME_DT_ATMOS_DYN_UNIT     = "SEC",| (TIME\_DT\_ATMOS\_DYNの単位)\\
\verb| TIME_DT_ATMOS_PHY_MP       = 10.0D0,| (雲物理過程のタイムステップ)\\
\verb| TIME_DT_ATMOS_PHY_MP_UNIT  = "SEC",| (TIME\_DT\_ATMOS\_PHY\_MPの単位)\\
\verb|/|\\
\end{tabularx}
}}}\\

上記の各部分を変更することで積分時間や積分時間間隔を変更することができる。


%またrun.confのPARAM\_TIMEに
%\begin{verbatim}
% TIME_DT_ATMOS_PHY_TB       = 0.10D0,  (乱流スキームの時間ステップ)
% TIME_DT_ATMOS_PHY_TB_UNIT  = "SEC", (TIME_DT_ATMOS_PHY_TBの単位)
%\end{verbatim}
%を加える。


