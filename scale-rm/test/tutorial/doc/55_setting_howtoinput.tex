\section{\SecAdvanceInputDataSetting} \label{sec:adv_datainput}
%====================================================================================

現在、\scalerm では初期値・境界値作成おいて、下記のデータの読み込みに対応している。

\begin{table}[htb]
\begin{center}
\caption{\scalelib が読込に対応する外部入力データフォーマット}
\begin{tabularx}{150mm}{|l|l|X|} \hline
 \rowcolor[gray]{0.9} データ形式      & \verb|FILETYPE_ORG|  & 備考 \\ \hline
 SCALEデータ    & \verb|SCALE-RM|     & historyデータのみ対応;latlonカタログを必要とする。 \\ \hline
 バイナリデータ & \verb|GrADS|        & データ読み込み用のnamelistを別途必要とする。 \\ \hline
 NICAMデータ    & \verb|NICAM-NETCDF| & NetCDF形式のLatLonデータに対応する。 \\ \hline
 WRFデータ      & \verb|WRF-ARW|      & ``wrfout''、``wrfrst''の両方に対応する。 \\ \hline
\end{tabularx}
\label{tab:inputdata_format}
\end{center}
\end{table}


初期値・境界値データのフォーマットの指定は、
\verb|scale-rm_init|の設定ファイル(\verb|init.**.conf|)で、
\namelist{PARAM_MKINIT_REAL_***}の\nmitem{FILETYPE_ORG}の項目で指定する。

SCALEデータ形式は主にオフライン・ネスティング実験で使用される。
詳細については、\ref{subsec:nest_offline}節を参照されたい。

NICAMデータは、正20面体格子データではなく、緯度・経度座標に変換されたデータのみ読み込みに対応している。
WRFデータについてはモデル出力データをそのまま使用することができる。

バイナリデータとは、「4バイト単精度浮動小数点のダイレクトアクセス方式、Fortran型バイナリデータ」を指す。
\textcolor{red}{GRIB/GRIB2のデータ形式は、チュートリアルで説明した方法に基づいて、
バイナリデータ形式を経由してSCALEに読み込ませることができる。}
その他にも、任意のデータを境界値に使用したい場合は、バイナリデータ形式に変換することで読み込ませることができる。
下記では、バイナリデータの読み込み手続きについて説明する。

%%%---------------------------------------------------------------------------------%%%%

\subsection{\SecAdvanceInputDataSetting} \label{sec:adv_datainput}






