\chapter{計算領域の変更の練習問題}


\section{理想実験の設定を変更してみる}


理想実験のチュートリアル設定を変更して4-MPI並列で実行できるようにしてみる。注意する点は領域全体の格子点数を
維持するように設定することである。今回は準2次元実験なので、X方向に4分割して4-MPI並列を達成する。
この場合の設定方法は下記のとおりである。\\

\noindent {\small {\gt
\ovalbox{
\begin{tabularx}{140mm}{l}
\verb|&PARAM_INDEX|\\
\verb| KMAX = 97,|\\
\verb| IMAX = 10,|\\
\verb| JMAX = 2,|\\
\verb|/|\\
\\
\verb|&PARAM_PRC|\\
\verb| PRC_NUM_X       = 4,|\\
\verb| PRC_NUM_Y       = 1,|\\
\verb|/|\\
\end{tabularx}
}}}\\

\noindent X方向に4分割を指定するため、\verb|PRC_NUM_X = 4|と記述されている。そして、領域全体で40格子点
とするために、\verb|IMAX = 10|と記述されている。Y方向と鉛直方向には何も変更していない。
\textcolor{red}{この変更を、{\bf init\_R20kmDX500m.conf}と{\bf run\_R20kmDX500m.conf}の両方に施さなければならない。}
そして、つぎのようにMPIコマンドに指定するプロセス数を``4''として、初期値作成、モデル実行の順で
作業を進めれば、4-MPI並列で実行することができる。
\begin{verbatim}
  $ mpirun  -n  4  ./scale-rm_init  init_R20kmDX500m.conf
  $ mpirun  -n  4  ./scale-rm       run_R20kmDX500m.conf
\end{verbatim}

計算領域(総演算量)を維持したままMPIプロセス数を
2倍に増やすことによって、プロセス数に応じて使用するコア数も
2倍となる場合、1つのMPIプロセスあたりの
問題サイズ(演算量 per PRC)が1/2に減る。
したがって、計算にかかる時間も理想的には半分になる
%\footnote{計算科学用語では、この変更、つまり総演算量一定でプロセスあたりの演算量を減らしていくことを``strong scaling''と呼ぶ。}。
実験機では、2-MPI並列のときチュートリアルの時間積分に60 sec かかっていたが、4-MPI並列にすることで同じ計算が32 secで終了できた。
ここで説明したMPIプロセス数の変更を加えたサンプルファイルが、同じディレクトリ下の``sample''ディレクトリ内に
\verb|init_R20kmDX500m.pe4.conf|、\verb|run_R20kmDX500m.pe4.conf|として置いてあるので、うまく実行できない場合は
参考にして欲しい。


\section{現実実験の設定を変更してみる}
ここでは、現実大気実験のチュートリアルのconfigファイル(\verb|run.conf|)を元にしていくつかの変更例を説明する。
以下の変更例をもとに、自分の行いたい実験設定にあったconfigファイルを作り上げてもらいたい。\\


{\bf a. MPIプロセス数はそのままに計算領域を4倍に広げる}\\

この設定はとても簡単である。デフォルト設定は、X方向、Y方向ともに2つのMPIプロセスを使用し、格子点数はともに
30点なので、全領域の格子点数は、$(2 \times 60)_{x} \times (2 \times 60)_{y} = 14400$点である。
従って、計算領域を4倍に広げるためには、\verb|IMAX|、\verb|JMAX|の値をデフォルトに対して2倍の値に変更するだけである。
これで、全領域の格子点数は、$(2 \times 120)_{x} \times (2 \times 120)_{y} = 57600$点となり、14400点の4倍の
計算領域になっている。この変更を施したconfigファイルの例は次のとおりである。
赤文字の部分がデフォルトからの変更点を意味する。\\

\noindent {\small {\gt
\ovalbox{
\begin{tabularx}{140mm}{l}
\verb|&PARAM_PRC| \\
\verb| PRC_NUM_X      = 2,| \\
\verb| PRC_NUM_Y      = 2,| \\
\verb| PRC_PERIODIC_X = .false.,| \\
\verb| PRC_PERIODIC_Y = .false.,| \\
\verb|/| \\
 \\
\verb|&PARAM_INDEX| \\
\verb| KMAX = 36,| \\
\textcolor{red}{\verb| IMAX = 120,|} \\
\textcolor{red}{\verb| JMAX = 120,|} \\
\verb|/| \\
\end{tabularx}
}}}\\

\vspace{5mm}
{\bf b. 1つのMPIプロセスあたりの格子点数はそのままに計算領域を4倍に広げる}\\

先程は、MPIプロセス数を維持して計算領域を広げたが今度は、1つのMPIプロセスあたりの格子点数はそのままに、MPIプロセス数を
増やすことで計算領域を4倍に広げる方法を説明する。1つのMPIプロセスあたりの格子点数(プロセスあたりの演算量)は
そのままにMPIプロセス数を増やすことで計算領域を広げる(総演算量を増やす)。この場合、1つのCPUが担当する問題サイズは
変わらないため、理想的には計算にかかる時間を増やすことなく計算領域を広げることができる
\footnote{計算科学用語では、この変更、つまりプロセスあたりの演算量一定で総演算量を増やしていくことを``weak scaling''と呼ぶ。}。

計算領域を4倍に広げるためには、\verb|PRC_NUM_X|、\verb|PRC_NUM_Y|の値をデフォルトに対して2倍の値に変更するだけである。
これで、全領域の格子点数は、$(4 \times 60)_{x} \times (4 \times 60)_{y} = 57600$点となり、14400点の4倍の
計算領域になっている。このとき必要なMPIプロセス数は、$4 \times 4 = 16$プロセス、つまりデフォルトの4倍のMPIプロセス数が必要になる。
この変更を施したconfigファイルの例は次のとおりである。赤文字の部分がデフォルトからの変更点を意味する。\\

\noindent {\small {\gt
\ovalbox{
\begin{tabularx}{140mm}{l}
\verb|&PARAM_PRC| \\
\textcolor{red}{\verb| PRC_NUM_X      = 4,|} \\
\textcolor{red}{\verb| PRC_NUM_Y      = 4,|} \\
\verb| PRC_PERIODIC_X = .false.,| \\
\verb| PRC_PERIODIC_Y = .false.,| \\
\verb|/| \\
 \\
\verb|&PARAM_INDEX| \\
\verb| KMAX = 36,| \\
\verb| IMAX = 60,| \\
\verb| JMAX = 60,| \\
\verb|/| \\
\end{tabularx}
}}}\\

\vspace{5mm}
{\bf c. 計算領域はそのままに水平格子間隔を3 kmに変更する}\\

現実大気実験チュートリアルのデフォルト設定は、X方向、Y方向の総格子点数は120点で水平格子間隔が15 kmなので、
計算領域は1800 km $\times$ 1800 kmの領域となっている。ここでは、この領域を維持したまま水平格子間隔を3 kmに
狭める設定にトライする。水平格子間隔が15 kmから3 kmへ1/5だけ小さくなるので逆に1方向あたりの総格子点数は5倍、
つまり600点必要になる。この600点をどのようにMPIプロセス数とプロセスあたりの格子点数として割り振るかは
ユーザーの環境や計算機リソース量によって異なる。たとえば、X方向、Y方向ともに10プロセスずつ、合計で100プロセスを
使えば、プロセスあたりの格子点数は60点となり、積分時間間隔が短くなることを無視すれば演算量はデフォルトと変わらない。

しかし、なかなか100プロセスを利用できる計算機を持っている環境にいる人は少ないだろう。そこで、デフォルトから
各方向に1つずつMPIプロセス数を増やし、$3 \times 3 = 9$プロセスを利用した設定を考えてみる。
1方向あたりの総格子点数は600点なので、プロセスあたりの格子点数は$600 \div 3 = 200$点となる。

ここでの変更で気をつけなければならないことは、バッファー領域の幅である。現実大気実験チュートリアルの
デフォルト設定ではバッファー領域は、計算領域トップと東西南北の側面境界に設定されており、側面境界のバッファー領域の
幅は片側300 km、つまり15 km格子間隔で20点のバッファー格子点が確保されている。水平格子間隔を15 kmから3 kmへ変更
したので、このままでは100点もの格子点がバッファー領域に取られてしまう。SCALEでは一般的に20〜40点程度の
バッファー格子点を設定するようにしているので、デフォルトと同じ20点になるように側面境界のバッファー領域の
幅は片側60 kmと設定する。鉛直層設定は変更していないため、計算領域トップのバッファー領域については設定を変更する
必要はない。

この変更を施したconfigファイルの例は次のとおりである。赤文字の部分がデフォルトからの変更点を意味する。\\

\noindent {\small {\gt
\ovalbox{
\begin{tabularx}{140mm}{l}
\verb|&PARAM_PRC| \\
\textcolor{red}{\verb| PRC_NUM_X      = 3,|} \\
\textcolor{red}{\verb| PRC_NUM_Y      = 3,|} \\
\verb| PRC_PERIODIC_X = .false.,| \\
\verb| PRC_PERIODIC_Y = .false.,| \\
\verb|/| \\
 \\
\verb|&PARAM_INDEX| \\
\verb| KMAX = 36,| \\
\textcolor{red}{\verb| IMAX = 200,|} \\
\textcolor{red}{\verb| JMAX = 200,|} \\
\verb|/| \\
 \\
\verb|&PARAM_GRID| \\
\textcolor{red}{\verb| DX = 3000.D0,|} \\
\textcolor{red}{\verb| DY = 3000.D0,|} \\
\verb| FZ(:) =    80.841D0,   248.821D0, ... ... 1062.158D0,| \\
\verb|            1306.565D0,  1570.008D0, ... ... 2845.575D0,| \\
\verb|       〜 中略 〜|\\
\verb|           18387.010D0, 19980.750D0, ... ... 28113.205D0,| \\
\verb| BUFFER_DZ = 5000.D0,| \\
\textcolor{red}{\verb| BUFFER_DX = 60000.D0,   ! 20 buffer|} \\
\textcolor{red}{\verb| BUFFER_DY = 60000.D0,   ! 20 buffer|} \\
\verb|/| \\
\end{tabularx}
}}}\\

