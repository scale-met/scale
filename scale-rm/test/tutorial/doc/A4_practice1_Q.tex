\section*{練習問題}

\begin{enumerate}
\item {\bf 計算領域は変えず、MPI並列数を変更したい}\\
第\ref{chap:tutorial_reall}章の現実実験チュートリアルの設定ファイル\verb|**.conf|について、
4-MPI並列の設定を6-MPI並列に変更する。

\item {\bf MPI並列数は変えず、計算領域を変更したい}\\
第\ref{chap:tutorial_reall}章の現実実験チュートリアルの設定について、
MPI並列数は変更せず、計算領域を$x$方向に4/3倍に拡大、$y$方向は2/3倍に縮小する。

\item {\bf 計算領域は変えず、水平格子間隔を変更したい}\\
第\ref{chap:tutorial_reall}章の現実実験チュートリアルの設定について、
計算領域は変えず、水平格子間隔を5kmに変更する。

\item {\bf 計算領域の位置を変更したい}\\
第\ref{chap:tutorial_reall}章の現実実験チュートリアルの設定について、
計算領域の大きさは変えず、中心位置を経度139度45.4分、緯度35度41.3分に変更する。

\item {\bf 積分時間の変更したい}\\
第\ref{chap:tutorial_reall}章の現実実験チュートリアルの設定について、
6時間積分から12時間積分に変更する。

\item {\bf 出力変数の追加と出力間隔の変更したい}\\
第\ref{chap:tutorial_reall}章の現実実験チュートリアルの計算(scale-rm)の出力の設定で、
下向き短波放射と上向き短波放射の出力を増やし、出力の時間間隔を1時間に変更する。

\item {\bf リスタート計算をしたい}\\
第\ref{chap:tutorial_reall}章の現実実験チュートリアルの計算について、
最初に3時間動かして、リスタートでさらに3時間計算して、
計6時間計算し、リスタートなしで6時間積分した結果(チュートリアルの結果)と比較する。

\item {\bf オンラインネスティング計算をしたい}\\

\item {\bf 雲微物理スキームをを変更したい}\\


\end{enumerate}

