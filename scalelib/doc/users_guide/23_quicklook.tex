\section{Quicklook SCALE-LES output}

scale-lesのファイルはMPIプロセス毎に分割ファイルとして
netcdfフォーマットでアウトプットされる.
ここではgpviewもしくはgradsを使用した描画方法について説明する。

\subsection{gpview}

Dennou系のRubyベースコマンドgpviewによって直接描画することができる.
Ruby DCL / GPhysのインストールにあたってはCentOS6, 7,Fedora,SUSE11.2,
11.3, Vine Linux 6に対しては,リポジトリを用意しています.
どのような変数がhistoryファイルに含まれているかは,netcdfに付属している
ncdumpのほか,GPhysに含まれているgpdumpを用いても調べることが出来る.
以下に簡単な描画例を示す.その他のオプション等については--helpを使って
ヘルプを表示させるか,電脳倶楽部のWebページを参照のこと.

historyの描画例
\begin{itemize}
\item 水平風東西成分を描画する:\\
高度1000m指定,時刻3600秒指定,コンターなし,(画面をクリックするか''q''を打つことで終了する)\\
\verb|>$ gpview history.pe00*@U,z=1000,time=3600 --nocont|
\item 水平風南北成分を描画する:\\
高度1000m指定,カンバスの縦横比=1,変数の描画範囲=4.8~5.2 [m/s],時間軸をアニメーション(クリックで進む)\\
\verb|>$ gpview history.pe00*@V,z=1000 --aspect 1 --range=4.8:5.2 --anim time|
\item 水蒸気混合比の南北-鉛直断面図を描画する:\\
東西方向に50kmの位置を指定,コンターなし,時間軸をアニメーション,Gawのオプションにより自動でアニメが進む,横軸と縦軸を交換 (exch)\\
\verb|>$ gpview history.pe00*@QV,x=50000 --aspect 1 --nocont --anim time --Gaw --exch|
\end{itemize}

下記のように\verb|init_****.pe#####.nc|ファイルなどを指定することで,初期値・境界値や下端境界条件のファイルの中身も見ることが出来る.
\begin{itemize}
\item \verb|>$ gpview init_00000000000.000.pe00*@MOMX,z=1000 --aspect 1 --nocont|
\item \verb|>$ gpview boundary.pe00*@VELX,z=1000 --nocont --anim time|
\item \verb|>$ gpview topo.pe00*@TOPO --aspect 1|
\item \verb|>$ gpview landuse.pe00*@FRAC_LAND --aspect 1|
\end{itemize}

