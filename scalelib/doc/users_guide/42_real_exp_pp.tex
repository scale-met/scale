
%-------------------------------------------------------%
\section{地形・土地利用データの作成:pp}
%-------------------------------------------------------%

ここでは,\ref{sec:source_code}節でダウンロードした\verb|tutorial_data|は,
\verb|${TOPDIR}/data/|の下に展開されていると想定している.

まず,ppディレクトリへ移動する.
ppでは現実実験のための地形データ、土地利用データを作成する.
\begin{verbatim}
 $ cd scale/scale-les/test/tutorial/pp
\end{verbatim}
ppディレクトリの中には,\verb|pp.conf|という名前の
コンフィグファイルが準備されている.
地球上でのドメインの位置や格子点数など、実験設定に合わせて,
適宜\verb|pp.conf|を編集する必要があるが,
チュートリアルでは,すでに編集済みの\verb|pp.conf|が
与えられているためそのまま利用する.
\verb|pp.conf|の設定の中で特に注意するべき項目は,\verb|PARAM_CONVERT|である.
\begin{verbatim}
 &PARAM_CONVERT
  CONVERT_TOPO = .true.,
  CONVERT_LANDUSE = .true.,
 /
\end{verbatim}
上記のように\verb|CONVERT_TOPO|と\verb|CONVERT_LANDUSE|が
\verb|.true.|となっていることが,
それぞれ地形と土地利用の処理を行うことを意味している.
詳細なコンフィグファイルの内容については,
Appendix \ref{app:namelist}を参照されたい.

次に,コンパイル済みのバイナリと入力データをppディレクトリへリンクする.
\begin{verbatim}
 $ ln -s ${TOPDIR}/bin/scale-les_pp ./
 $ ln -s ${TOPDIR}/data/tutorial_data/data/input_topo    ./
 $ ln -s ${TOPDIR}/data/tutorial_data/data/input_landuse ./
\end{verbatim}
今回は,Table \ref{tab:grids}に示されているように,
9つのMPIプロセスを使用する設定なので次のように実行する.
\begin{verbatim}
 $ mpirun -n 9 ./scale-les_pp pp.conf
\end{verbatim}
正常にジョブが終了すれば,\verb|topo_d01.pe######.nc|と\verb|landuse_d01.pe######.nc|というファイルがMPIプロセス数だけ,つまり9つずつ生成される(\verb|######|にはMPIプロセスの番号が入る).
それぞれ,ドメインの格子点に内挿された地形と土地利用の情報が入ってる.

処理内容のログとして,\verb|pp_LOG_d01.pe000000|という名前でログファイルも
出力されるので内容を確かめておくこと.
gpviewがインストールされていれば,次のコマンドによって作成された地形と土地利用データを
描画してチェックすることができる.
正しく作成されていれば,Fig. \ref{fig:domain}と同じように描かれる.
\begin{verbatim}
$ gpview topo_d01.pe00000*@TOPO --aspect=1
$ gpview landuse_d01.pe00000*@FRAC_LAND --aspect=1
\end{verbatim}


