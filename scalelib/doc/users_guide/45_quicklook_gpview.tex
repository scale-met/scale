\section{Quicklook SCALE-LES output}
%####################################################################################

SCALE-LESモデルの出力ファイルはMPIプロセス毎に出力されるため,
計算領域が分割された状態で出力される.
それぞのファイルフォーマットは気候・予報(CF)メタデータ規約に対応したnetcdf4形式である.
ここでは,RubyDCL/Gphysを用いた描画方法について説明する。
RubyDCL/Gphysがインストールされていない場合は,Appendix \ref{sec:env_setting}を参照して
事前にインストールすること.

この他、grads formatに変換するためのポストプロセス(netcdf2grads)が用意されているが、
netcdf2gradsの使用方法については、\ref{sec:net2g}で説明する.


\subsubsection{ファイル内の変数を確認する}
%-----------------------------------------------------------------------------------

まず,gplistコマンドを用いてヒストリーファイル内の変数を確認する.

\begin{verbatim}
$ gplist history_d01.pe000000.nc
\end{verbatim}

\verb|history_d01.pe######.nc|のファイルはMPI並列数分だけ存在するが,
ファイルに収められている変数は基本的に同じであるため,
どれか1つのファイルについて中身を確認すればよい.
上記のコマンドを実行すると,下記のように変数リストが表示される.

{\small \begin{verbatim}
history_d01.pe000000.nc:
  x	[x=62]	'X'	(m)
  y	[y=62]	'Y'	(m)
  z	[z=36]	'Z'	(m)
  xh	[xh=62]	'X (half level)'	(m)
  yh	[yh=62]	'Y (half level)'	(m)
  zh	[zh=36]	'Z (half level)'	(m)

 ~~中略~~

  topo	[x=62,y=62]	'topography'	(m)
  lsmask	[x=62,y=62]	'fraction for land-sea mask'	(0-1)
  time	[time=24]	'time'	(seconds since 1999-01-01 00:00:00)
  time_bnds	[nv=2,time=24]	''	(seconds since 1999-01-01 00:00:00)
  DENS	[x=62,y=62,z=36,time=24]	'density'	(kg/m3)
  MOMZ	[x=62,y=62,zh=36,time=24]	'momentum z'	(kg/m2/s)
  MOMX	[xh=62,y=62,z=36,time=24]	'momentum x'	(kg/m2/s)
  MOMY	[x=62,yh=62,z=36,time=24]	'momentum y'	(kg/m2/s)
  RHOT	[x=62,y=62,z=36,time=24]	'rho * theta'	(kg/m3*K)
  QV	[x=62,y=62,z=36,time=24]	'Water Vapor mixing ratio'	(kg/kg)
  W	[x=62,y=62,z=36,time=24]	'velocity w'	(m/s)
  U	[x=62,y=62,z=36,time=24]	'velocity u'	(m/s)
  V	[x=62,y=62,z=36,time=24]	'velocity v'	(m/s)
  PT	[x=62,y=62,z=36,time=24]	'potential temp.'	(K)
  QHYD	[x=62,y=62,z=36,time=24]	'total hydrometeors'	(kg/kg)
  PRES	[x=62,y=62,z=36,time=24]	'pressure'	(Pa)
  T	[x=62,y=62,z=36,time=24]	'temperature'	(K)
  RH	[x=62,y=62,z=36,time=24]	'relative humidity'	(%)
  PREC	[x=62,y=62,time=24]	'surface precipitation rate'	(kg/m2/s)
  OLR	[x=62,y=62,time=24]	'TOA net longwave  radiation flux'	(W/m2)

 ~~中略~~
\end{verbatim} }

x, y, zなどの変数は格子情報を表しており,\verb|DENS, MOMZ, MOMX|や
\verb|PT, QHYD, PRES|といった変数が気象場のデータである.
この中から1つ,もしくは複数の変数を選んで描画する.

各気象場の変数に対してデータの軸構成が書かれている.
例えば,\verb|DENS|は空間を構成するx軸,y軸,z軸と時間を構成するtime軸によってデータが構成されている.
高度や時刻を指定して描画するためには,このようにデータを構成する軸についてあらかじめ知っておく必要がある.
例えばヒストリー出力されている時刻を調べるには次のようにコマンドを実行する.

\begin{verbatim}
$ gpprint history_d01.pe000000.nc@time
\end{verbatim}

そうすると下記のようにtime変数の値が標準出力へ表示される.

\begin{verbatim}
 1.07154e+07, 1.07172e+07, 1.0719e+07, 1.07208e+07, 1.07226e+07, 1.07244e+07,
 1.07262e+07, 1.0728e+07, 1.07298e+07, 1.07316e+07, 1.07334e+07, 1.07352e+07,
 1.0737e+07, 1.07388e+07, 1.07406e+07, 1.07424e+07, 1.07442e+07, 1.0746e+07,
 1.07478e+07, 1.07496e+07, 1.07514e+07, 1.07532e+07, 1.0755e+07, 1.07568e+07,
\end{verbatim}

SCALE-LESモデル内では日時が全て秒単位で表現されているため,ここで表記される
時刻も日時が全て秒として積算された値が時刻として表示される.
SCALE-LESモデルはデフォルト設定で,一番始めに初期値がヒストリー出力され,
そのあと\verb|HISTORY_DEFAULT_TINTERVAL|に従った時間間隔でヒストリー出力されている.
チュートリアルでは12時間積分する中で,30分ごとに出力したため,
1(初期値)+23ステップ=24ステップの時刻のデータが出力されていることがわかる.


{\small *netcdfに付属している\verb|"ncdump"|を用いてもよい.}


\subsubsection{ファイル内の変数を描画する}
%-----------------------------------------------------------------------------------

gpviewを用いて次のように描画できる.

(例 1)高度1500mにおける温度の水平分布
\begin{itemize}

\item 積分開始3時間後の様子
  \begin{verbatim}
  gpview history_d01.pe00000*@T,z=1500,time=1.07262e+07 --range=270:291 --aspect=1
  \end{verbatim}
  実行結果はFig. \ref{fig:hist_t}aのようになる(画面をクリックするか\verb|"q"|を打つことで終了する).
  オプションの\verb|z=1500,time=1.07262e+07|によって描画する高度,および時刻を指定している.
  \verb|--range=270:291|のオプションは描画する値のレンジを指定している.
  また,\verb|"--wsn=2"|のオプションを付けて実行することで,\verb|"dcl.ps"|というファイル名で
  画像をPSファイルに保存することができる.

\item 積分開始6時間後の様子:実行結果はFig. \ref{fig:hist_t}bのようになる.
  \begin{verbatim}
  gpview history_d01.pe00000*@T,z=1500,time=1.0737e+07 --range=270:291 --aspect=1
  \end{verbatim}
  先との変更点はtimeオプションの引数だけである.

\item 積分開始9時間後の様子:実行結果はFig. \ref{fig:hist_t}cのようになる.
  \begin{verbatim}
  gpview history_d01.pe00000*@T,z=1500,time=1.07478e+07 --range=270:291 --aspect=1
  \end{verbatim}

\end{itemize}


\begin{figure}[t]
\begin{center}
  \includegraphics[width=0.9\hsize]{./figure/gpview_hist_t.eps}\\
  \caption{高度1.5kmにおける気温の分布}
  \label{fig:hist_t}
\end{center}
\end{figure}


(例 2)高度1500mにおける相対湿度と水平風の分布
\begin{itemize}

\item 積分開始3時間後の様子
  \begin{verbatim}
  gpvect --scalar --slice z=1500,time=1.07262e+07 --nocont --range=10:102        \
         --aspect=1 --xintv=10 --yintv=10 --unit_vect history_d01.pe00000*@RH    \
         history_d01.pe00000*@U history_d01.pe00000*@V
  \end{verbatim}
  実行結果はFig. \ref{fig:hist_rh}aのようになる.
  ここでは\verb|gpview|ではなくベクトルを描画することができる\verb|gpvect|を使用して描画する.
  オプションの\verb|--xintv=10 --yintv=10|は,ベクトルを描画する格子点間隔を指定している.

\item 積分開始6時間後の様子:実行結果はFig. \ref{fig:hist_rh}bのようになる.
  \begin{verbatim}
  gpvect --scalar --slice z=1500,time=1.0737e+07 --nocont --range=10:102        \
         --aspect=1 --xintv=10 --yintv=10 --unit_vect history_d01.pe00000*@RH    \
         history_d01.pe00000*@U history_d01.pe00000*@V
  \end{verbatim}

\item 積分開始9時間後の様子:実行結果はFig. \ref{fig:hist_rh}cのようになる.
  \begin{verbatim}
  gpvect --scalar --slice z=1500,time=1.07478e+07 --nocont --range=10:102        \
         --aspect=1 --xintv=10 --yintv=10 --unit_vect history_d01.pe00000*@RH    \
         history_d01.pe00000*@U history_d01.pe00000*@V
  \end{verbatim}
\end{itemize}


\begin{figure}[t]
\begin{center}
  \includegraphics[width=0.9\hsize]{./figure/gpview_hist_rh.eps}\\
  \caption{高度1.5kmにおける相対湿度と水平風の分布:カラーシェードは相対湿度,ベクトルは水平風を示す}
  \label{fig:hist_rh}
\end{center}
\end{figure}


(例 3)QHYD(凝結物の混合比)についての計算領域西端から800kmの地点における鉛直-南北断面図

\begin{verbatim}
gpview history_d01.pe00000*@QHYD,x=800000,y=100000:600000,z=0:16000,time=1.0737e+07
\end{verbatim}
積分開始6時間後の様子で,実行結果はFig. \ref{fig:hist_qhyd}のようになる.
オプションの\verb|x=800000,y=100000:600000,z=0:16000|によって,xは800kmの地点,
yは100kmから600kmの範囲,zは0kmから16kmの範囲を描画するように指定している.


その他の描画例を下記に挙げる.
\begin{itemize}

\item 水平風南北成分を描画する:\\
  高度1000m指定,カンバスの縦横比=1,コンターなし,変数の描画範囲=4.8~5.2 [m/s],
  時間軸をアニメーション(クリックで進む)\\
  \verb|$ gpview history.pe00*@V,z=1000 --aspect 1 --nocont --range=4.8:5.2 --anim time|

\item 水蒸気混合比の南北-鉛直断面図を描画する:\\
  東西方向に50kmの位置を指定,コンターなし,時間軸をアニメーション,\verb|"--Gaw"|のオプションに
 より自動でアニメが進む,横軸と縦軸を交換 (exch)\\
  \verb|$ gpview history.pe00*@QV,x=50000 --aspect 1 --nocont --anim time --Gaw --exch|

\end{itemize}


\begin{figure}[t]
\begin{center}
  \includegraphics[width=0.7\hsize]{./figure/gpview_hist_qhyd.eps}\\
  \caption{x=800kmにおける鉛直-南北断面図}
  \label{fig:hist_qhyd}
\end{center}
\end{figure}


また,チュートリアルの中でも使用してきたが,下記のように\verb|init_****.pe#####.nc|ファイルなどを
指定することで,初期値・境界値や下端境界条件のファイルの中身も見ることが出来る.

\begin{itemize}
 \item \verb|$ gpview init_00000000000.000.pe00*@MOMX,z=1000 --aspect 1 --nocont|
 \item \verb|$ gpview boundary.pe00*@VELX,z=1000 --nocont --anim time|
 \item \verb|$ gpview topo.pe00*@TOPO --aspect 1|
 \item \verb|$ gpview landuse.pe00*@FRAC_LAND --aspect 1|
\end{itemize}


その他のオプション等については--helpを使ってヘルプを表示させるか,電脳倶楽部のWebページ
(\url{http://ruby.gfd-dennou.org/products/gphys/doc/gpview.html})を参照すること.

