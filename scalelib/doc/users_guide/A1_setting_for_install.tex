\chapter{ライブラリ環境のインストール}
\label{sec:env_setting}

\subsubsection{インストールに関する基礎知識}

Linuxをインストール後、各種プログラムのインストールにはコマンドライン端末を使う。
コマンドラインのシンボル(\verb|$, #|)は、コマンドの実行を示す。
下記の表記の違いは、プログラムを実行する権限の違いを示している。

\begin{verbatim}
 #        <- root権限で実行するコマンド
 $        <- ユーザ権限で実行するコマンド
\end{verbatim}
権限の一時的な切り替えにはsuコマンドを用いる。
\verb|{User_Name}|は実際のユーザ名に読み替えること。
\begin{verbatim}
 $ su {User_Name}   <- {User_Name}のユーザー名でログイン
 $ exit             <- {User_Name}のユーザー名でログインを修了
 $ su -             <- root権限に変更
 #
\end{verbatim}

コマンドオプションにハイフンを用いると、そのユーザでのログインを行う。
用いない場合、権限のみの変更となる。またユーザ名を省略するとrootでのログインを試す。
ユーザの一時切り替えを終わるには、exitコマンドを用いる。
各プログラムをインストールするための圧縮ファイルは、/tmpにダウンロードされていると仮定する。
他のディレクトリにダウンロードしてある場合は、mvコマンド等を用いて/tmpに移動しておくことを勧める。
コード表記のうち、ダブルスラッシュ(//)で始まる行は解説のためのもので、実際に記述する必要はない。


\begin{verbatim}
 $ vi
\end{verbatim}

上記のコマンドは、vi(汎用テキストエディタ)プログラムを実行する。
他に、gedit、emacsなどのテキストエディタがある。
マニュアルではviコマンドの実行指示があるが、各自の使いやすいエディタへ適宜読み替えること。

\subsubsection{各種インストールの事前準備}

SCALEのインストールに必要な開発ツールやライブラリのインストール方法について説明する。
ここでは簡単のため、Red Had Enterpise LinuxのクローンOSとして有名なCentOSを例に、
パッケージマネージャを利用した方法を紹介する。
他のOSでは適宜読み替えること。例えばUbuntuなどではyumではなくapt系コマンドを用いる。

CentOSではいくつかの開発ツール、ライブラリを外部リポジトリから持ってくる必要がある。
具体的にはepelリポジトリ。ルート権限になり,外部リポジトリを認識させる。
\begin{verbatim}
 # yum install epel-release
\end{verbatim}
yumのグループインストール機能を用いて,開発ツールをまとめてインストールする。
\begin{verbatim}
 # yum groupinstall "development tools"
\end{verbatim}
グループインストールではインストールされないパッケージを個別に追加する。
\begin{verbatim}
 # yum install hdf5-devel hdf5-static
 # yum install netcdf-devel netcdf-static
 # yum install openmpi-devel
\end{verbatim}


\subsubsection{MPI設定}

SCALEをマルチプロセッサで走らせるため、OpenMPIの設定を行う。
ユーザ権限に移動して.bashrcをエディタで開き,
\begin{verbatim}
 $ vi ~/.bashrc
\end{verbatim}
下記をファイルの最後に追加して,環境変数の設定を記述する。
\begin{verbatim}
 // ---------------- Add to end of the file ----------------
 # OpenMPI
 export MPI="/usr/lib64/openmpi"
 export PATH="$PATH:$MPI/bin"
 export LD_LIBRARY_PATH="$LD_LIBRARY_PATH:$MPI/lib"
\end{verbatim}
編集が終わったら、環境設定を有効にする。
\begin{verbatim}
 $ . ~/.bashrc
\end{verbatim}


\subsubsection{Installation of Gphys}

CentOSの場合、yumリポジトリに地球電脳倶楽部のGFD-Dennouリポジトリを登録することで、
簡単にGphysをインストールできる。
root権限で、GFD-Dennouリポジトリを次のような内容で登録する。

\begin{verbatim}
 # vi /etc/yum.repos.d/GFD-Dennou.repo
\end{verbatim}

\begin{verbatim}
 // ---------------- Edit the file ----------------
 [gfd-dennou]
 name=GFD DENNOU Club RPMS for CentOS $releasever - $basearch
 baseurl=http://www.gfd-dennou.org/library/cc-env/rpm-dennou/CentOS/$releasever/$basearch/
 enabled=1
 gpgcheck=0
\end{verbatim}
編集が終わったら、yumでGphysをインストールする。
\begin{verbatim}
 # yum install gphys
\end{verbatim}

