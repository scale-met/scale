%Appendix
\chapter{Namelist in run.conf}

\subsubsection{PARAM\_IO}
\begin{tabularx}{150mm}{|l|c|c|X|} \hline
 \rowcolor[gray]{0.9} 名称 & 種類 & 初期値 & 説明 \\ \hline
 \verb|IO_LOG_BASENAME| & 文字列 & "LOG" & ログファイルの接頭辞。 \\ \hline
 \verb|IO_LOG_ALLNODE| & 論理値 & .false. & 全ノードログ出力するかどうか。 \\ \hline
\end{tabularx}


\subsubsection{PARAM\_TIME}
\begin{tabularx}{150mm}{|l|c|c|X|} \hline
 \rowcolor[gray]{0.9} 名称 & 種類 & 初期値 & 説明 \\ \hline
 \verb|TIME_STARTDATE| & 整数配列 & 0000, 1, 1, 0, 0, 0 & 積分実行時の初期時刻。 \\ \hline
 \verb|TIME_STARTMS| & 実数 & 0.0D0 & 初期時刻マイクロ秒。 \\ \hline
 \verb|TIME_DURATION| & 実数 & 0.0D0 & 実行する積分時間。 \\ \hline
 \verb|TIME_DT| & 実数 & 0.0D0 & 積分1STEPに要する時間。 \\ \hline
 \verb|TIME_DT_ATMOS_DYN| & 実数 & \verb|TIME_DT| & 力学スキームの時間差分値。\verb|TIME_DT|の約数である必要がある。 \\ \hline
 \verb|TIME_DT_ATMOS_PHY_MP| & 実数 & \verb|TIME_DT| & 雲微物理スキームの時間差分値。\verb|TIME_DT|の倍数である必要がある。 \\ \hline
 \verb|TIME_DT_ATMOS_PHY_RD| & 実数 & \verb|TIME_DT| & 放射スキームの時間差分値。\verb|TIME_DT|の倍数である必要がある。 \\ \hline
 \verb|TIME_DT_ATMOS_PHY_SF| & 実数 & \verb|TIME_DT| & 地表面スキームの時間差分値。\verb|TIME_DT|の倍数である必要がある。 \\ \hline
 \verb|TIME_DT_ATMOS_PHY_TB| & 実数 & \verb|TIME_DT| & 乱流スキームの時間差分値。\verb|TIME_DT|の倍数である必要がある。 \\ \hline
 \verb|TIME_DT_OCEAN| & 実数 & \verb|TIME_DT| & 海洋スキームの時間差分値。\verb|TIME_DT|の倍数である必要がある。 \\ \hline
 \verb|TIME_DT_LAND| & 実数 & \verb|TIME_DT| & 陸面スキームの時間差分値。\verb|TIME_DT|の倍数である必要がある。 \\ \hline
 \verb|TIME_DT_URBAN| & 実数 & \verb|TIME_DT| & 都市スキームの時間差分値。\verb|TIME_DT|の倍数である必要がある。 \\ \hline
 \verb|TIME_DURATION_UNIT| & 文字列 & "SEC" & 積分時間単位。 \\ \hline
 \verb|TIME_DT_UNIT| & 文字列 & "SEC" & 積分1STEPの時間単位。 \\ \hline
 \verb|TIME_DT_ATMOS_DYN_UNIT| & 文字列 & \verb|TIME_DT_UNIT| & 力学スキームの時間単位。 \\ \hline
 \verb|TIME_DT_ATMOS_PHY_MP_UNIT| & 文字列 & \verb|TIME_DT_UNIT| & 雲微物理スキームの時間単位。 \\ \hline
 \verb|TIME_DT_ATMOS_PHY_RD_UNIT| & 文字列 & \verb|TIME_DT_UNIT| & 放射スキームの時間単位。 \\ \hline
 \verb|TIME_DT_ATMOS_PHY_SF_UNIT| & 文字列 & \verb|TIME_DT_UNIT| & 地表面スキームの時間単位。 \\ \hline
 \verb|TIME_DT_ATMOS_PHY_TB_UNIT| & 文字列 & \verb|TIME_DT_UNIT| & 乱流スキームの時間単位。 \\ \hline
 \verb|TIME_DT_OCEAN_UNIT| & 文字列 & \verb|TIME_DT_UNIT| & 海洋スキームの時間単位。 \\ \hline
 \verb|TIME_DT_LAND_UNIT| & 文字列 & \verb|TIME_DT_UNIT| & 陸面スキームの時間単位。 \\ \hline
 \verb|TIME_DT_URBAN_UNIT| & 文字列 & \verb|TIME_DT_UNIT| & 都市スキームの時間単位。 \\ \hline
\end{tabularx}


\subsubsection{PARAM\_NEST}
\begin{tabularx}{150mm}{|l|c|c|X|} \hline
 \rowcolor[gray]{0.9} 名称 & 種類 & 初期値 & 説明 \\ \hline
 \verb|USE_NESTING| & 論理値 & .false. & Nestingを使うかどうか。 \\ \hline
 \verb|OFFLINE| & 論理値 & .true. & Online Nestingかどうか。\verb|USE_NESTING|が真のときのみ有効。 \\ \hline
 \verb|ONLINE_DOMAIN_NUM| & 整数 &  & ドメイン番号。\verb|USE_NESTING|が真, \verb|OFFLINE|が偽のときのみ有効。 \\ \hline
 \verb|ONLINE_IAM_PARENT| & 論理値 &  & 親ドメインをもつかどうか。\verb|USE_NESTING|が真, \verb|OFFLINE|が偽のときのみ有効。 \\ \hline
 \verb|ONLINE_IAM_DAUGHTER| & 論理値 &  & 娘ドメインをもつかどうか。\verb|USE_NESTING|が真, \verb|OFFLINE|が偽のときのみ有効。 \\ \hline
 \verb|ONLINE_BOUNDARY_USE_QHYD| & 論理値 & .false. & 娘ドメインにQHYDを渡すかどうか。\verb|USE_NESTING|が真, \verb|OFFLINE|が偽のときのみ有効。 \\ \hline
 \verb|ONLINE_AGGRESSIVE_COMM| & 論理値 & .false. & 安全な同期通信を行うかどうか。\verb|USE_NESTING|が真, \verb|OFFLINE|が偽のときのみ有効。 \\ \hline
\end{tabularx}


\subsubsection{PARAM\_STATISTICS}
\begin{tabularx}{150mm}{|l|c|c|X|} \hline
 \rowcolor[gray]{0.9} 名称 & 種類 & 初期値 & 説明 \\ \hline
 \verb|STATISTICS_checktotal| & 論理値 & .false. & 値のチェックを行うかどうか。 \\ \hline
 \verb|STATISTICS_use_globalcomm| & 論理値 & .false. & 全ノード通信を行うかどうか。 \\ \hline
\end{tabularx}


\subsubsection{PARAM\_RESTRAT}
\begin{tabularx}{150mm}{|l|c|c|X|} \hline
 \rowcolor[gray]{0.9} 名称 & 種類 & 初期値 & 説明 \\ \hline
 \verb|RESTART_OUTPUT| & 論理値 & .false. & restartファイルを出力するかどうか。 \\ \hline
 \verb|RESTART_OUT_BASENAME| & 文字列 &  & 書き出すrestartファイルの接頭辞。\verb|RESTART_OUTPUT|が真のときに有効。 \\ \hline
 \verb|RESTART_IN_BASENAME| & 文字列 &  & 読み込むrestartファイルの接頭辞。 \\ \hline
\end{tabularx}


\subsubsection{PARAM\_TOPO}
\begin{tabularx}{150mm}{|l|c|c|X|} \hline
 \rowcolor[gray]{0.9} 名称 & 種類 & 初期値 & 説明 \\ \hline
 \verb|TOPO_IN_BASENAME| & 文字列 &  & 読み込む地形ファイルの接頭辞。 \\ \hline
\end{tabularx}


\subsubsection{PARAM\_LANDUSE}
\begin{tabularx}{150mm}{|l|c|c|X|} \hline
 \rowcolor[gray]{0.9} 名称 & 種類 & 初期値 & 説明 \\ \hline
 \verb|LANDUSE_IN_BASENAME| & 文字列 &  & 読み込む土地利用ファイルの接頭辞。 \\ \hline
\end{tabularx}


\subsubsection{PARAM\_LAND\_PROPERTY}
\begin{tabularx}{150mm}{|l|c|c|X|} \hline
 \rowcolor[gray]{0.9} 名称 & 種類 & 初期値 & 説明 \\ \hline
 \verb|LAND_PROPERTY_IN_FILENAME| & 文字列 &  & 読み込む土壌パラメータファイル名。 \\ \hline
\end{tabularx}


\subsubsection{PARAM\_PRC}
\begin{tabularx}{150mm}{|l|c|c|X|} \hline
 \rowcolor[gray]{0.9} 名称 & 種類 & 初期値 & 説明 \\ \hline
 \verb|PRC_NUM_X| & 整数 &  & X方向に割り当てるプロセス数。 \\ \hline
 \verb|PRC_NUM_Y| & 整数 &  & Y方向に割り当てるプロセス数。 \\ \hline
 \verb|PRC_PERIODIC_X| & 論理値 &  & X方向に周期境界とするかどうか。 \\ \hline
 \verb|PRC_PERIODIC_Y| & 論理値 &  & Y方向に周期境界とするかどうか。 \\ \hline
\end{tabularx}


\subsubsection{PARAM\_INDEX}
\begin{tabularx}{150mm}{|l|c|c|X|} \hline
 \rowcolor[gray]{0.9} 名称 & 種類 & 初期値 & 説明 \\ \hline
 \verb|KMAX| & 整数 &  & 大気の鉛直層数。 \\ \hline
 \verb|IMAX| & 整数 &  & プロセスあたりのX方向の格子数。 \\ \hline
 \verb|JMAX| & 整数 &  & プロセスあたりのY方向の格子数。 \\ \hline
\end{tabularx}


\subsubsection{PARAM\_LAND\_INDEX}
\begin{tabularx}{150mm}{|l|c|c|X|} \hline
 \rowcolor[gray]{0.9} 名称 & 種類 & 初期値 & 説明 \\ \hline
 \verb|LKMAX| & 整数 &  & 陸面の鉛直層数。 \\ \hline
\end{tabularx}


\subsubsection{PARAM\_URBAN\_INDEX}
\begin{tabularx}{150mm}{|l|c|c|X|} \hline
 \rowcolor[gray]{0.9} 名称 & 種類 & 初期値 & 説明 \\ \hline
 \verb|UKMAX| & 整数 &  & 都市の鉛直層数。 \\ \hline
\end{tabularx}


\subsubsection{PARAM\_LAND\_GRID}
\begin{tabularx}{150mm}{|l|c|c|X|} \hline
 \rowcolor[gray]{0.9} 名称 & 種類 & 初期値 & 説明 \\ \hline
 \verb|LDZ| & 実数配列 &  & 陸面の鉛直層の層厚。鉛直層数分の設定が必要。 \\ \hline
\end{tabularx}


\subsubsection{PARAM\_URBAN\_GRID}
\begin{tabularx}{150mm}{|l|c|c|X|} \hline
 \rowcolor[gray]{0.9} 名称 & 種類 & 初期値 & 説明 \\ \hline
 \verb|UDZ| & 実数配列 &  & 都市の鉛直層の層厚。鉛直層数分の設定が必要。 \\ \hline
\end{tabularx}


\subsubsection{PARAM\_GRID}
\begin{tabularx}{150mm}{|l|c|c|X|} \hline
 \rowcolor[gray]{0.9} 名称 & 種類 & 初期値 & 説明 \\ \hline
 \verb|DZ| & 実数 &  & 大気の鉛直層の層厚。FZと排他的設定。 \\ \hline
 \verb|DX| & 実数 &  & 大気のX方向の格子間隔。\\ \hline
 \verb|DY| & 実数 &  & 大気のY方向の格子間隔。\\ \hline
 \verb|FZ| & 実数配列 &  & 大気の鉛直層の面高度。鉛直層数分の設定が必要。DZと排他的設定。 \\ \hline
 \verb|BUFFER_DZ| & 実数 &  & 大気の最上層の緩和領域幅。 \\ \hline
 \verb|BUFFER_DX| & 実数 &  & 大気のX方向の緩和領域幅。\\ \hline
 \verb|BUFFER_DY| & 実数 &  & 大気のY方向の緩和領域幅。\\ \hline
\end{tabularx}


\subsubsection{PARAM\_MAPPROJ}
\begin{tabularx}{150mm}{|l|c|c|X|} \hline
 \rowcolor[gray]{0.9} 名称 & 種類 & 初期値 & 説明 \\ \hline
 \verb|MPRJ_basepoint_lon| & 実数 &  & 計算領域の中心経度。 \\ \hline
 \verb|MPRJ_basepoint_lat| & 実数 &  & 計算領域の中心緯度。 \\ \hline
 \verb|MPRJ_type| & 文字列 &  & 計算領域の投影図法。 \\ \hline
 \verb|MPRJ_LC_lat1| & 実数 &  & 投影図法がLCの場合の参照緯度1。 \\ \hline
 \verb|MPRJ_LC_lat2| & 実数 &  & 投影図法がLCの場合の参照緯度2。 \\ \hline
\end{tabularx}


\subsubsection{PARAM\_CONST}
\begin{tabularx}{150mm}{|l|c|c|X|} \hline
 \rowcolor[gray]{0.9} 名称 & 種類 & 初期値 & 説明 \\ \hline
 \verb|CONST_THERMODYN_TYPE| & 文字列 & "EXACT" & 内部エネルギーの定義種類。SIMPLEは定数。EXACTは温度依存。 \\ \hline
\end{tabularx}


\subsubsection{PARAM\_TRACER}
\begin{tabularx}{150mm}{|l|c|c|X|} \hline
 \rowcolor[gray]{0.9} 名称 & 種類 & 初期値 & 説明 \\ \hline
 \verb|TRACER_TYPE| & 文字列 & "OFF" & トレーサーの種類。通常、\verb|ATMOS_PHY_MP_TYPE|と同じ。 \\ \hline
\end{tabularx}


\subsubsection{PARAM\_ATMOS}
\begin{tabularx}{150mm}{|l|c|c|X|} \hline
 \rowcolor[gray]{0.9} 名称 & 種類 & 初期値 & 説明 \\ \hline
 \verb|ATMOS_DYN_TYPE| & 文字列 & "OFF" & 力学スキームの種類。 \\ \hline
 \verb|ATMOS_PHY_MP_TYPE| & 文字列 & "OFF" & 雲微物理スキームの種類。 \\ \hline
 \verb|ATMOS_PHY_RD_TYPE| & 文字列 & "OFF" & 放射スキームの種類。 \\ \hline
 \verb|ATMOS_PHY_SF_TYPE| & 文字列 & "OFF" & 地表面スキームの種類。 \\ \hline
 \verb|ATMOS_PHY_TB_TYPE| & 文字列 & "OFF" & 乱流スキームの種類。 \\ \hline
\end{tabularx}


\subsubsection{PARAM\_OCEAN}
\begin{tabularx}{150mm}{|l|c|c|X|} \hline
 \rowcolor[gray]{0.9} 名称 & 種類 & 初期値 & 説明 \\ \hline
 \verb|OCEAN_TYPE| & 文字列 & "OFF" & 海洋スキームの種類。 \\ \hline
\end{tabularx}


\subsubsection{PARAM\_LAND}
\begin{tabularx}{150mm}{|l|c|c|X|} \hline
 \rowcolor[gray]{0.9} 名称 & 種類 & 初期値 & 説明 \\ \hline
 \verb|LAND_TYPE| & 文字列 & "OFF" & 陸面スキームの種類。 \\ \hline
\end{tabularx}


\subsubsection{PARAM\_URBAN}
\begin{tabularx}{150mm}{|l|c|c|X|} \hline
 \rowcolor[gray]{0.9} 名称 & 種類 & 初期値 & 説明 \\ \hline
 \verb|URBAN_TYPE| & 文字列 & "OFF" & 都市スキームの種類。 \\ \hline
\end{tabularx}


%&PARAM_ATMOS_VARS
% ATMOS_VARS_CHECKRANGE = .false.,
%/
%
%&PARAM_ATMOS_REFSTATE
% ATMOS_REFSTATE_TYPE        = "INIT",
% ATMOS_REFSTATE_UPDATE_FLAG = .true.,
% ATMOS_REFSTATE_UPDATE_DT   = 10800.D0,
%/
%
%&PARAM_ATMOS_BOUNDARY
% ATMOS_BOUNDARY_TYPE        = "REAL",
% ATMOS_BOUNDARY_IN_BASENAME = "boundary_d01",
% ATMOS_BOUNDARY_UPDATE_DT   = 21600.D0,
% ATMOS_BOUNDARY_USE_VELZ    = .true.,
% ATMOS_BOUNDARY_USE_QHYD    = .false.,
% ATMOS_BOUNDARY_VALUE_VELZ  = 0.0D0,
% ATMOS_BOUNDARY_LINEAR_H    = .false.,
% ATMOS_BOUNDARY_EXP_H       = 2.d0,
%/
%
%&PARAM_ATMOS_DYN
% ATMOS_DYN_NUMERICAL_DIFF_COEF   = 1.D-2,
% ATMOS_DYN_NUMERICAL_DIFF_COEF_Q = 1.D-2,
% ATMOS_DYN_enable_coriolis       = .true.,
%/
%
%&PARAM_ATMOS_PHY_RD_MSTRN
% ATMOS_PHY_RD_MSTRN_KADD                  = 30,
% ATMOS_PHY_RD_MSTRN_GASPARA_IN_FILENAME   = "PARAG.29",
% ATMOS_PHY_RD_MSTRN_AEROPARA_IN_FILENAME  = "PARAPC.29",
% ATMOS_PHY_RD_MSTRN_HYGROPARA_IN_FILENAME = "VARDATA.RM29",
% ATMOS_PHY_RD_MSTRN_NBAND                 = 29,
%/
%
%&PARAM_ATMOS_PHY_RD_PROFILE
% ATMOS_PHY_RD_PROFILE_TOA                   = 100.D0,
% ATMOS_PHY_RD_PROFILE_CIRA86_IN_FILENAME    = "cira.nc",
% ATMOS_PHY_RD_PROFILE_MIPAS2001_IN_BASENAME = "MIPAS",
%/
%
%#################################################
%#
%# model configuration: ocean
%#
%#################################################
%
%&PARAM_OCEAN_VARS
% OCEAN_VARS_CHECKRANGE = .false.,
%/
%
%&PARAM_OCEAN_PHY_SLAB
% OCEAN_PHY_SLAB_DEPTH = 10.D0,
%/
%
%#################################################
%#
%# model configuration: land
%#
%#################################################
%
%&PARAM_LAND_VARS
% LAND_VARS_CHECKRANGE = .false.,
%/
%
%&PARAM_LAND_PHY_SLAB
% LAND_PHY_UPDATE_BOTTOM_TEMP  = .false.,
% LAND_PHY_UPDATE_BOTTOM_WATER = .true.,
%/
%
%#################################################
%#
%# model configuration: urban
%#
%#################################################
%
%&PARAM_URBAN_VARS
% URBAN_VARS_CHECKRANGE = .false.,
%/
%
%&PARAM_URBAN_PHY_SLC
% ZR         = 15.0D0,
% roof_width = 7.5D0,
% road_width = 22.5D0,
% AH         = 0.0D0,
% ALH        = 0.0D0,
% STRGR      = 0.24D0,
% STRGB      = 0.009D0,
% STRGG      = 0.24D0,
% AKSR       = 2.28D0,
% AKSB       = 2.28D0,
% AKSG       = 2.28D0,
% ALBR       = 0.20D0,
% ALBB       = 0.20D0,
% ALBG       = 0.20D0,
% EPSR       = 0.97D0,
% EPSB       = 0.97D0,
% EPSG       = 0.97D0,
% Z0R        = 0.005D0,
% Z0B        = 0.005D0,
% Z0G        = 0.005D0,
% CAPR       = 2.01D6,
% CAPB       = 2.01D6,
% CAPG       = 2.01D6,
%/
